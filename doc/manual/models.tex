\section{Geodynamic Models and Numerical Methods in \aspect{}}
\label{sec:models}

\subsection{Basic equations}
\label{sec:equations}

\aspect{} solves a system of equations in a $d=2$- or $d=3$-dimensional
domain $\Omega$ that describes the motion of a highly viscous fluid driven
by differences in the gravitational force due to a density that depends on
the temperature. In the following, we largely follow the exposition of this
material in Schubert, Turcotte and Olson \cite{STO01}.

Specifically, we consider the following set of equations for velocity $\mathbf
u$, pressure $p$ and temperature $T$, as well as a set of advected quantities
$c_i$ that we call \textit{compositional fields}:
\begin{align}
  \label{eq:stokes-1}
  -\nabla \cdot \left[2\eta \left(\varepsilon(\mathbf u)
                                  - \frac{1}{3}(\nabla \cdot \mathbf u)\mathbf 1\right)
                \right] + \nabla p &=
  \rho \mathbf g
  &
  & \textrm{in $\Omega$},
  \\
  \label{eq:stokes-2}
  \nabla \cdot (\rho \mathbf u) &= 0
  &
  & \textrm{in $\Omega$},
  \\
  \label{eq:temperature}
  \rho C_p \left(\frac{\partial T}{\partial t} + \mathbf u\cdot\nabla T\right)
  - \nabla\cdot k\nabla T
  &=
  \rho H
  \notag
  \\
  &\quad
  +
  2\eta
  \left(\varepsilon(\mathbf u) - \frac{1}{3}(\nabla \cdot \mathbf u)\mathbf 1\right)
  :
  \left(\varepsilon(\mathbf u) - \frac{1}{3}(\nabla \cdot \mathbf u)\mathbf 1\right)
  \\
  &\quad
  +\alpha T \left( \mathbf u \cdot \nabla p \right)
  \notag
  \\
  &\quad
  + \rho T \Delta S \left(\frac{\partial X}{\partial t} + \mathbf u\cdot\nabla X\right)
  &
  & \textrm{in $\Omega$},
  \notag
  \\
  \label{eq:compositional}
  \frac{\partial c_i}{\partial t} + \mathbf u\cdot\nabla c_i
  &=
  q_i
  &
  & \textrm{in $\Omega$},
  i=1\ldots C
\end{align}
where $\varepsilon(\mathbf u) = \frac{1}{2}(\nabla \mathbf u + \nabla\mathbf
u^T)$ is the symmetric gradient of the velocity (often called the
\textit{strain rate}).%
\footnote{There is no consensus in the sciences on the notation used
  for strain and strain rate. The symbols $\varepsilon$,
  $\dot\varepsilon$,  $\varepsilon(\mathbf u)$, and
  $\dot\varepsilon(\mathbf u)$, can all be found. In this manual, and
  in the code, we will consistently use $\varepsilon$ as an
  \textit{operator}, i.e., the symbol is not used on its own but only
  as applied to a field. In other words, if $\mathbf u$ is the
  velocity field, then $\varepsilon(\mathbf u) = \frac{1}{2}(\nabla
  \mathbf u + \nabla\mathbf u^T)$ will denote the strain rate. On the
  other hand, if $\mathbf d$ is the
  displacement field, then $\varepsilon(\mathbf d) = \frac{1}{2}(\nabla
  \mathbf d + \nabla\mathbf d^T)$ will denote the strain.}


In this set of equations, \eqref{eq:stokes-1} and \eqref{eq:stokes-2}
represent the compressible Stokes equations in which $\mathbf u=\mathbf
u(\mathbf x,t)$ is the velocity field and $p=p(\mathbf x,t)$ the pressure
field. Both fields depend on space $\mathbf x$ and time $t$. Fluid flow is
driven by the gravity force that acts on the fluid and that is proportional to
both the density of the fluid and the strength of the gravitational pull.

Coupled to this Stokes system is equation \eqref{eq:temperature} for the
temperature field $T=T(\mathbf x,t)$ that contains heat conduction terms as
well as advection with the flow velocity $\mathbf u$. The right hand side
terms of this equation correspond to
\begin{itemize}
\item internal heat production for example due to radioactive
  decay;
\item friction heating;
\item adiabatic compression of material;
\item phase change.
\end{itemize}
The last term of the temperature equation corresponds to
the latent heat generated or consumed in the process of phase change of material. The latent heat release
is proportional to changes in the fraction of material $X$ that has already
undergone the phase transition (also called phase function) and the change
of entropy $\Delta S$. This process applies both
to solid-state phase transitions and to melting/solidification.
Here, $\Delta S$ is positive for exothermic phase
transitions. As the phase of the material, for a given composition, depends
on the temperature and pressure, the latent heat term can be reformulated:
\begin{gather*}
\frac{\partial X}{\partial t} + \mathbf u\cdot\nabla X
=
\frac{DX}{Dt} 
= 
\frac{\partial X}{\partial T} \frac{DT}{Dt}
 + \frac{\partial X}{\partial p} \frac{Dp}{Dt}
= 
\frac{\partial X}{\partial T} 
\left(\frac{\partial T}{\partial t} + \mathbf u\cdot\nabla T
\right)
 + \frac{\partial X}{\partial p} \mathbf u\cdot\nabla p.
\end{gather*}
The last transformation results from the assumption that the flow field is
always in equilibrium and consequently $\partial p/\partial t=0$ (this is the
same assumption that underlies the fact that equation \eqref{eq:stokes-1}
does not have a term $\partial \mathbf u / \partial t$). With this
reformulation, we can rewrite \eqref{eq:temperature} in the following way in
which it is in fact implemented:
\begin{align}
  \label{eq:temperature-reformulated}
  \left(\rho C_p - \rho T \Delta S \frac{\partial X}{\partial T}\right) 
  \left(\frac{\partial T}{\partial t} + \mathbf u\cdot\nabla
  T\right) - \nabla\cdot k\nabla T
  &=
  \rho H
  \notag
  \\
  &\quad
  +
  2\eta
  \left(\varepsilon(\mathbf u) - \frac{1}{3}(\nabla \cdot \mathbf u)\mathbf 1\right)
  :
  \left(\varepsilon(\mathbf u) - \frac{1}{3}(\nabla \cdot \mathbf u)\mathbf 1\right)
  \\
  &\quad
  +\alpha T \left( \mathbf u \cdot \nabla p \right)
  \notag
  \\
  &\quad
  + \rho T \Delta S \frac{\partial X}{\partial p} \mathbf u\cdot\nabla p
  & \quad & \textrm{in $\Omega$}.
  \notag
\end{align}

The last of the equations above, equation~\eqref{eq:compositional}, describes
the evolution of additional fields that are transported along with the
velocity field $\mathbf u$ and may react with each other and react to other
features of the solution, but that do not diffuse. We call these fields $c_i$
\textit{compositional fields}, although they can also be used for other
purposes than just tracking chemical compositions. We will discuss this
equation in more detail in Section~\ref{sec:compositional}.

\subsubsection{A comment on adiabatic heating}
Other codes and texts sometimes make a simplification to the adiabatic heating
term in the previous equation. If you assume the vertical component of the
gradient of the \textit{dynamic} pressure to be small compared to the gradient
of the \textit{total} pressure (in other words, the gradient is dominated by
the gradient of the hydrostatic pressure), then $ -\rho \mathbf g \approx
\nabla \mathbf{p} $, and we have the following relation (the negative sign is
due to $\mathbf g$ pointing downwards) 
\begin{align*}
\alpha T \left( \mathbf u \cdot \nabla \mathbf p \right)
  & \approx -\alpha \rho T \mathbf u \cdot \mathbf g.
\end{align*}
While this simplification is possible, it is not necessary if you have access
to the total pressure. \aspect{} therefore by default implements the original 
term without this simplification, but allows to simplify this term by setting
the ``\texttt{Use simplified adiabatic heating}'' 
\index[prmindex]{Use simplified adiabatic heating}
\index[prmindexfull]{Heating model!Adiabatic heating!Use simplified adiabatic heating}
parameter in section~\ref{parameters:Heating_20model/Adiabatic_20heating}.

\subsubsection{Boundary conditions}
Having discussed \eqref{eq:temperature}, let us come to the last one of the
original set of equations, \eqref{eq:compositional}. It describes the
motion of a set of advected quantities $c_i(\mathbf x,t),i=1\ldots C$. We call these
\textit{compositional fields} because we think of them as spatially and
temporally varying concentrations of different elements, minerals, or other
constituents of the composition of the material that convects. As such, these
fields participate actively in determining the values of the various
coefficients of these equations. On the other hand, \aspect{} also allows the
definition of material models that are independent of these compositional
fields, making them passively advected quantities. Several of the cookbooks in
Section~\ref{sec:cookbooks} consider compositional fields in this way, i.e.,
essentially as tracer quantities that only keep track of where material came
from.

These equations are
augmented by boundary conditions that can either be of Dirichlet, Neumann, or
tangential type on subsets of the boundary $\Gamma=\partial\Omega$:
\begin{align}
  \mathbf u &= 0 & \qquad &\textrm{on $\Gamma_{0,\mathbf u}$},
  \\
  \mathbf u &= \mathbf u_{\text{prescribed}} & \qquad &\textrm{on
  $\Gamma_{\text{prescribed},\mathbf u}$},
  \\
  \mathbf n \cdot \mathbf u &= 0 & \qquad &\textrm{on $\Gamma_{\parallel,\mathbf
  u}$},
  \\
  (2\eta \varepsilon(\mathbf u) -p I)\mathbf n  &= \mathbf t & \qquad
  &\textrm{on $\Gamma_{\text{traction},\mathbf u}$},
  \\
  T &= T_{\text{prescribed}}
   & \qquad &\textrm{on $\Gamma_{D,T}$},
  \\
  \mathbf n \cdot k\nabla T &= 0
   & \qquad &\textrm{on $\Gamma_{N,T}$}.
  \\
  \label{eq:gamma-in-composition}
  c_i &= c_{i,\text{prescribed}}
   & \qquad &\textrm{on $\Gamma_{\text{in}}=\{\mathbf x: \mathbf
   u\cdot\mathbf n<0\}$}.
\end{align}
Here, the boundary conditions for velocity and temperature are subdivided into
disjoint parts:
\begin{itemize}
  \item $\Gamma_{0,\mathbf u}$ corresponds to parts of the boundary on
which the velocity is fixed to be zero.
  \item $\Gamma_{\text{prescribed},\mathbf u}$ corresponds to parts of the
  boundary on which the velocity is prescribed to some value (which could also
  be zero). It is possible to restrict prescribing the velocity to only certain
  components of the velocity vector.
  \item $\Gamma_{\parallel,\mathbf u}$ corresponds to parts of the boundary on
  which the velocity may be nonzero but must be parallel to the boundary, with the
tangential component undetermined.
  \item $\Gamma_{\text{traction},\mathbf u}$ corresponds to parts of the
  boundary on which the traction is prescribed to some surface force density (a
  common application being $\mathbf t=-p\mathbf n$ if one
  just wants to prescribe a pressure component). It is possible to restrict
  prescribing the traction to only certain vector components.
  \item $\Gamma_{D,T}$ corresponds to places where the temperature is prescribed
  (for example at the inner and outer boundaries of the earth mantle).
  \item $\Gamma_{N,T}$ corresponds to places where the temperature is unknown
  but the heat flux across the boundary is zero (for example on symmetry surfaces if only a part
of the shell that constitutes the domain the Earth mantle occupies is
simulated).
\end{itemize}
We require that one of these boundary conditions hold at each
point for both velocity and temperature, i.e.,
$\Gamma_{0,\mathbf u}\cup\Gamma_{{\text{prescribed}}\mathbf
  u}\cup\Gamma_{\parallel,\mathbf u}\cup\Gamma_{{\text{traction}}\mathbf
  u}=\Gamma$ and
$\Gamma_{D,T}\cup\Gamma_{N,T}=\Gamma$. 

Boundary conditions have to be imposed for the compositional fields only
at those parts of the boundary where flow points inward, see equation
\eqref{eq:gamma-in-composition}, but not where it is either tangential
to the boundary or points outward. The difference in treatment between
temperature and compositional boundary conditions is due to the fact
that the temperature equation contains a (possibly small) diffusion
component, whereas the compositional equations do not.

There are other equations that \aspect{} can optionally solve. For example, it
can deal with free surfaces (see Section~\ref{sec:freesurface}), melt generation and
transport (see Section~\ref{sec:melt_transport}), and it can advect along
particles (see Section~\ref{sec:particles}). These optional models
are discussed in more detail in the indicated sections.


\subsubsection{Two-dimensional models}
\label{sec:meaning-of-2d}
\aspect{} allows solving both two- and three-dimensional
models via a parameter in the input files, see also Section~\ref{sec:2d-vs-3d}.
\index[prmindex]{Dimension} \index[prmindexfull]{Dimension}
At the same time, the world is unambiguously three-dimensional. This raises the
question what exactly we mean when we say that we want to solve two-dimensional
problems.

The notion we adopt here -- in agreement with that chosen by many other codes --
is to think of two-dimensional models in the following way: We assume that the
domain we want to solve on is a two-dimensional cross section (parameterized by
$x$ and $y$ coordinates) that extends infinitely far in both negative and
positive $z$ direction. Further, we assume that the velocity is zero in $z$
direction and that all variables have no variation in $z$ direction. As a
consequence, we ought to really think of these two-dimensional models as
three-dimensional ones in which the $z$ component of the velocity is zero and so
are all $z$ derivatives.

If one adopts this point of view, the Stokes equations
\eqref{eq:stokes-1}--\eqref{eq:stokes-2} naturally simplify in a way that allows
us to reduce the $3+1$ equations to only $2+1$, but it makes clear that the
correct description of the compressible strain rate is still 
$\varepsilon(\mathbf u) - \frac{1}{3}(\nabla \cdot \mathbf u)\mathbf 1$, rather
than using a factor of $\frac{1}{2}$ for the second term. (A derivation of why
the compressible strain rate tensor has this form can be found in \cite[Section
6.5]{STO01}.)

It is interesting to realize that this compressible strain rate indeed requires
a $3\times 3$ tensor: While under the assumptions above we have
\begin{align*}
  \varepsilon(\mathbf u) = 
  \begin{pmatrix}
    \tfrac{\partial u_x}{\partial x}
    &
    \tfrac 12 \tfrac{\partial u_x}{\partial y} + 
    \tfrac 12 \tfrac{\partial u_y}{\partial x}
    &
    0
    \\
    \tfrac 12 \tfrac{\partial u_x}{\partial y} + 
    \tfrac 12 \tfrac{\partial u_y}{\partial x}
    &
    \tfrac{\partial u_y}{\partial y}
    &
    0
    \\
    0 & 0 & 0
  \end{pmatrix}
\end{align*}
with the expected zeros in the last row and column, the full compressible strain
rate tensor reads
\begin{align*}
  \varepsilon(\mathbf u) - \frac{1}{3}(\nabla \cdot \mathbf u)\mathbf 1 = 
  \begin{pmatrix}
    \tfrac 23 \tfrac{\partial u_x}{\partial x}
    - \tfrac 13 \tfrac{\partial u_y}{\partial y}
    &
    \tfrac 12 \tfrac{\partial u_x}{\partial y} + 
    \tfrac 12 \tfrac{\partial u_y}{\partial x}
    &
    0
    \\
    \tfrac 12 \tfrac{\partial u_x}{\partial y} + 
    \tfrac 12 \tfrac{\partial u_y}{\partial x}
    &
    \tfrac 23 \tfrac{\partial u_y}{\partial y}
    - \tfrac 13 \tfrac{\partial u_x}{\partial x}
    &
    0
    \\
    0 & 0 &
    - \tfrac 13 \tfrac{\partial u_y}{\partial y}
    - \tfrac 13 \tfrac{\partial u_x}{\partial x}
  \end{pmatrix}.
\end{align*}
The entry in the $(3,3)$ position of this tensor may be surprising. It
disappears, however, when taking the (three-dimensional) divergence of the
stress, as is done in \eqref{eq:stokes-1}, because the divergence applies the $z$ derivative to all
elements of the last row -- and the assumption above was that all $z$
derivatives are zero; consequently whatever lives in the third row of the
strain rate tensor does not matter.



\subsubsection{Comments on the final set of equations}
\aspect{} solves these equations in essentially the form stated. In
particular, the form given in \eqref{eq:stokes-1} implies that the pressure
$p$ we compute is in fact the \textit{total pressure}, i.e., the sum of
hydrostatic pressure and dynamic pressure (however, see
Section~\ref{sec:pressure-static-dyn} for more information on this, as well as
the extensive discussion of this issue in \cite{KHB12}).
Consequently, it allows the direct use of this pressure when looking up
pressure dependent material parameters.


\subsection{Coefficients}
\label{sec:coefficients}

The equations above contain a significant number of coefficients that we will
discuss in the following. In the most general form, many of these coefficients
depend nonlinearly on the solution variables pressure $p$, temperature $T$
and, in the case of the viscosity, on the strain rate $\varepsilon(\mathbf
u)$. If compositional fields $\mathfrak c=\{c_1,\ldots,c_C\}$ are present (i.e.,
if $C>0$), coefficients may also depend on them. Alternatively, they may be
parameterized as a function
of the spatial variable $\mathbf x$. \aspect{} allows both kinds of
parameterizations.

Note that below we will discuss examples of the dependence of coefficients on
other quantities; which dependence is actually implemented in the code is a
different matter. As we will discuss in Sections~\ref{sec:parameters} and
\ref{sec:extending}, some versions of these models are already implemented and
can be selected from the input parameter file; others are easy to add to
\aspect{} by providing self-contained descriptions of a set of coefficients
that the rest of the code can then use without a need for further
modifications.

Concretely, we consider the following coefficients and dependencies:
\begin{itemize}
\item \textit{The viscosity $\eta=\eta(p,T,\varepsilon(\mathbf u),\mathfrak
c,\mathbf x)$:} Units $\textrm{Pa}\cdot \textrm{s} =
  \textrm{kg}\frac{1}{\textrm{m}\cdot\textrm{s}}$.

  The viscosity is the proportionality factor that relates total forces
  (external gravity minus pressure gradients) and fluid velocities $\mathbf
  u$. The simplest models assume that $\eta$ is constant, with the constant
  often chosen to be on the order of $10^{21} \textrm{Pa}\;\textrm{s}$.

  More complex (and more realistic) models assume that the viscosity depends
  on pressure, temperature and strain rate. Since this dependence is often
  difficult to quantify, one modeling approach is to make $\eta$ spatially
  dependent.

\item \textit{The density $\rho=\rho(p,T,\mathfrak c,\mathbf x)$:} Units
  $\frac{\textrm{kg}}{\textrm{m}^3}$.

  In general, the density depends on pressure and temperature, both through
  pressure compression, thermal expansion, and phase changes the material may
  undergo as it moves through the pressure-temperature phase diagram.

  The simplest parameterization for the density is to assume a linear
  dependence on temperature, yielding the form
  $\rho(T)=\rho_{\text{ref}}[1-\alpha (T-T_{\text{ref}})]$ where
  $\rho_{\text{ref}}$ is the reference density at temperature $T_{\text{ref}}$
  and $\alpha$ is the linear thermal expansion coefficient. For the earth's
  mantle, typical values for this parameterization would be
  $\rho_{\text{ref}}=3300\frac{\textrm{kg}}{\textrm{m}^3}$,
  $T_{\text{ref}}=293 \textrm{K}$, $\alpha=2\cdot 10^{-5}
  \frac{1}{\mathrm{K}}$.

\item \textit{The gravity vector $\mathbf g=\mathbf g(\mathbf x)$:} Units
  $\frac{\textrm{m}}{\textrm{s}^2}$.

  Simple models assume a radially inward gravity vector of constant magnitude
  (e.g., the surface gravity of Earth, $9.81 \frac{\textrm{m}}{\textrm{s}^2}$),
  or one that can be computed analytically assuming a homogeneous mantle
  density.

  A physically self-consistent model would compute the gravity vector as
  $\mathbf g = -\nabla \varphi$ with a gravity potential $\varphi$ that
  satisfies $-\Delta\varphi=4\pi G\rho$ with the density $\rho$ from above and
  $G$ the universal constant of gravity. This would provide a gravity vector
  that changes as a function of time. Such a model is not currently
  implemented.

\item \textit{The specific heat capacity $C_p=C_p(p,T,\mathfrak c,\mathbf x)$:}
Units $\frac{\textrm{J}}{\textrm{kg}\cdot\textrm{K}} =
  \frac{\textrm{m}^2}{\textrm{s}^2\cdot\textrm{K}}$.

  The specific heat capacity denotes the amount of energy needed to increase
  the temperature of one kilogram of material by one degree. Wikipedia lists a
  value of 790 $\frac{\textrm{J}}{\textrm{kg}\cdot\textrm{K}}$ for granite%
  \footnote{See \url{http://en.wikipedia.org/wiki/Specific_heat}.}
  For the earth mantle, a value of 1250
  $\frac{\textrm{J}}{\textrm{kg}\cdot\textrm{K}}$ is within the range
  suggested by the literature.


\item \textit{The thermal conductivity $k=k(p,T,\mathfrak c,\mathbf x)$:} Units
  $\frac{\textrm{W}}{\textrm{m}\cdot\textrm{K}}=\frac{\textrm{kg}\cdot\textrm{m}}{\textrm{s}^3\cdot\textrm{K}}$.

  The thermal conductivity denotes the amount of thermal energy flowing
  through a unit area for a given temperature gradient. It depends on the
  material and as such will from a physical perspective depend on pressure and
  temperature due to phase changes of the material as well as through
  different mechanisms for heat transport (see, for example, the partial
  transparency of perovskite, the most abundant
  material in the earth mantle, at pressures above around 120 GPa
  \cite{BRVMFG04}).

  As a rule of thumb for its
  order of magnitude, Wikipedia quotes values of
  $1.83$--$2.90\frac{\textrm{W}}{\textrm{m}\cdot\textrm{K}}$ for sandstone and
  $1.73$--$3.98\frac{\textrm{W}}{\textrm{m}\cdot\textrm{K}}$ for granite.%
  \footnote{See \url{http://en.wikipedia.org/wiki/Thermal_conductivity} and
    \url{http://en.wikipedia.org/wiki/List_of_thermal_conductivities}.} The
  values in the mantle are almost certainly higher than this though probably
  not by much. The exact value is not really all that important: heat
  transport through convection is several orders of magnitude more important
  than through thermal conduction.

  The thermal conductivity $k$ is often expressed in terms of the
  \textit{thermal diffusivity} $\kappa$ using the relation $k = \rho C_p \kappa$.

\item \textit{The intrinsic specific heat production $H=H(\mathbf x)$:} Units
  $\frac{\textrm{W}}{\textrm{kg}}=\frac{\textrm{m}^2}{\textrm{s}^3}$.

  This term denotes the intrinsic heating of the material, for example due to
  the decay of radioactive material. As such, it depends not on pressure or
  temperature, but may depend on the location due to different chemical
  composition of material in the earth mantle. The literature suggests a value
  of $\gamma=7.4\cdot 10^{-12}\frac{\textrm{W}}{\textrm{kg}}$.

\item \textit{The thermal expansion coefficient $\alpha=\alpha(p,T,\mathfrak c ,\mathbf x)$:} Units
  $\frac{1}{\textrm{K}}$.

  This term denotes by how much the material under consideration
  expands due to temperature increases. This coefficient is defined as
  $\alpha = -\frac{1}{\rho}\frac{\partial \rho}{\partial T}$, where
  the negative sign is due the fact that the density
  \textit{decreases} as a function of temperature. Alternatively, if
  one considers the \textit{volume} $V=V(T)$ a piece of material of mass $M$
  occupies, $V=\frac{M}{\rho}$, then the thermal expansion coefficient
  is defined as the relative increase in volume,
  $\alpha=\frac{1}{V}\frac{\partial V(T)}{\partial T}$, because 
  $\frac{\partial V(T)}{\partial T} =
   \frac{\partial \frac{M}{\rho}}{\partial T} =
   -\frac{M}{\rho^2} \frac{\partial \rho}{\partial T} =
   -\frac{V}{\rho} \frac{\partial \rho}{\partial T}$.

   The literature suggests that values of $\alpha=1\cdot
   10^{-5}\frac{1}{\textrm{K}}$ at the core-mantle boundary and $\alpha=4\cdot
   10^{-5}\frac{1}{\textrm{K}}$ are appropriate for Earth.

\item \textit{The change of entropy $\Delta S$ at a
  phase transition together with the derivatives of the phase function
  $X=X(p,T,\mathfrak c,\mathbf x)$ with regard to temperature and pressure:} Units
  $\frac{\textrm{J}}{\textrm{kg}\textrm{K}^2}$ ($-\Delta S \frac{\partial X}{\partial T}$) and
  $\frac{\textrm{m}^3}{\textrm{kg}\textrm{K}}$ ($\Delta S \frac{\partial X}{\partial p}$).

  When material undergoes a phase transition, the entropy changes due to
  release or consumption of latent heat. However, phase transitions occur
  gradually and for a given chemical composition it depends on temperature
  and pressure which phase prevails. Thus, the latent heat release can
  be calculated from the change of entropy $\Delta S$ and the derivatives
  of the phase function $\frac{\partial X}{\partial T}$ and
  $\frac{\partial X}{\partial p}$. These values have to be provided by
  the material model, separately for the coefficient
  $-\Delta S \frac{\partial X}{\partial T}$ on the left-hand side and
  $\Delta S \frac{\partial X}{\partial p}$ on the right-hand side of the
  temperature equation. However, they may be either approximated with the help
  of an analytic phase function, employing data from a thermodynamic database
  or in any other way that seems appropriate to the user.
\end{itemize}


\subsection{Dimensional or non-dimensionalized equations?}
\label{sec:non-dimensional}

Equations \eqref{eq:stokes-1}--\eqref{eq:temperature} are stated in the
physically correct form. One would usually interpret them in a way that the
various coefficients such as the viscosity, density and thermal conductivity
$\eta,\rho,\kappa$ are given in their correct physical units, typically
expressed in a system such as the meter, kilogram, second (MKS) system that is
part of the \href{http://en.wikipedia.org/wiki/SI}{SI} system.
This is certainly how we envision \aspect{} to be used: with geometries,
material models, boundary conditions and initial values to be given in their correct
physical units. As a consequence, when \aspect{} prints information about the
simulation onto the screen, it typically does so by using a postfix such as
\texttt{m/s} to indicate a velocity or \texttt{W/m\^{}2} to indicate a heat
flux.

\note{For convenience, output quantities are sometimes provided
  in units meters per \textit{year} instead of meters per \textit{second}
  (velocities) or in \textit{years} instead of \textit{seconds} (the current
  time, the time step size); this
  conversion happens at the time output is generated, and is not part of the
  solution process. Whether this conversion should happen is determined by the
  flag ``\texttt{Use years in output instead of seconds}'' in the input file,
\index[prmindex]{Use years in output instead of seconds}
\index[prmindexfull]{Use years in output instead of seconds}
  see Section~\ref{parameters:global}. Obviously, this conversion from seconds
  to years only makes sense if the model is described in physical units rather
  than in non-dimensionalized form, see below.}

That said, in reality, \aspect{} has no preferred system of
units as long as every material constant, geometry, time, etc., are all
expressed in the same system. In other words, it is entirely legitimate to
implement geometry and material models in which the dimension of the domain is
one, density and viscosity are one, and the density variation as a function of
temperature is scaled by the Rayleigh number -- i.e., to use the usual
non-dimensionalization of the equations~\eqref{eq:stokes-1}--\eqref{eq:temperature}. Some of the cookbooks in
Section~\ref{sec:cookbooks} use this non-dimensional form; for example,
the simplest cookbook in Section~\ref{sec:cookbooks-simple-box} as well as
the SolCx, SolKz and inclusion benchmarks in Sections~\ref{sec:benchmark-solcx},
are such cases. Whenever this is the case, output showing units \texttt{m/s} or
\texttt{W/m\^{}2} clearly no longer have a literal meaning. Rather, the unit postfix must in this case simply
be interpreted to mean that the number that precedes the first is a velocity and
a heat flux in the second case.

In other words, whether a computation uses physical or non-dimensional units
really depends on the geometry, material, initial and boundary condition
description of the particular case under consideration -- \aspect{} will simply
use whatever it is given. Whether one or the other is the more appropriate
description is a decision we purposefully leave to the user. There are of
course good reasons to use non-dimensional descriptions of realistic problems,
rather than to use the original form in which all coefficients remain in their
physical units. On the other hand, there are also downsides:
\begin{itemize}
  \item Non-dimensional descriptions, such as when using the
  \href{http://en.wikipedia.org/wiki/Rayleigh_number}{Rayleigh} number to
  indicate the relative strength of convective to diffusive thermal transport,
  have the advantage that they allow to reduce a system to its essence. For
  example, it is clear that we get the same behavior if one increases both the
  viscosity and the thermal expansion coefficient by a factor of two because the
  resulting Rayleigh number; similarly, if we were to increase the size of the
  domain by a factor of 2 and thermal diffusion coefficient by a factor of 8. In both of
  these cases, the non-dimensional equations are exactly the same. On the other
  hand, the equations in their physical unit form are different and one may not
  see that the result of this variations in coefficients will be exactly the
  same as before. Using non-dimensional variables therefore reduces the space of
  independent parameters one may have to consider when doing parameter studies.

  \item From a practical perspective, equations
  \eqref{eq:stokes-1}--\eqref{eq:temperature} are often ill-conditioned in
  their original form: the two sides of each equation have physical units
  different from those of the other equations, and their numerical values are
  often vastly different.%
  \footnote{To illustrate this, consider convection in the Earth as a
  back-of-the-envelope example.
  With the length scale of the mantle $L=3\cdot 10^6\;\text{m}$, viscosity
  $\eta=10^{24} \; \text{kg}/\text{m}/\text{s}$, density $\rho=3\cdot 10^3 \; \text{kg}/\text{m}^3$ and a typical
  velocity of $U=0.1\;\text{m}/\text{year}=3\cdot 10^{-9}\; \text{m}/\text{s}$, we get that the friction
  term in \eqref{eq:stokes-1} has size $\eta U/L^2 \approx 3\cdot 10^2 \;
  \text{kg}/\text{m}^2/\text{s}^2$. On the other hand, the term $\nabla\cdot(\rho u)$ in the
  continuity equation \eqref{eq:stokes-2} has size $\rho U/L\approx 3\cdot
  10^{-12} \; \text{kg}/\text{s}/\text{m}^3$. In other words, their \textit{numerical values} are 14
  orders of magnitude apart.}
  Of course, these values can not be compared: they have different physical
  units, and the ratios between these values depends on whether we choose to
  measure lengths in meters or kilometers, for example. Nevertheless, when
  implementing these equations in software, at one point or another, we have to
  work with numbers and at this point the physical units are lost. If one does
  not take care at this point, it is easy to get software in which all accuracy
  is lost due to round-off errors. On the other hand, non-dimensionalization
  typically avoids this since it normalizes all quantities so that values that
  appear in computations are typically on the order of one.

  \item On the downside, the numbers non-dimensionalized equations produce are
  not immediately comparable to ones we know from physical experiments. This is
  of little concern if all we have to do is convert every output number of our
  program back to physical units. On the other hand, it is more difficult and a
  source of many errors if this has to be done inside the program, for example,
  when looking up the viscosity as a pressure-, temperature- and
  strain-rate-dependent function: one first has to convert pressure,
  temperature and strain rate from non-dimensional to physical units, look up
  the corresponding viscosity in a table, and then convert the viscosity back to
  non-dimensional quantities. Getting this right at every one of the dozens or
  hundreds of places inside a program and using the correct (but distinct)
  conversion factors for each of these quantities is both a challenge and a possible source
  of errors.

  \item From a mathematical viewpoint, it is typically clear how an equation
  needs to be non-dimensionalized if all coefficients are constant. However, how
  is one to normalize the equations if, as is the case in the earth mantle, the
  viscosity varies by several orders of magnitude? In cases like these, one has
  to choose a reference viscosity, density, etc. While the resulting
  non-dimensionalization retains the universality of parameters in the
  equations, as discussed above, it is not entirely clear that this would also
  retain the numerical stability if the reference values are poorly chosen.
\end{itemize}

As a consequence of such considerations, most codes in the past have used
non-dimensionalized models. This was aided by the fact that until recently and
with notable exceptions, many models had constant coefficients and the
difficulties associated with variable coefficients were not a concern. On the
other hand, our goal with \aspect{} is for it to be a code that solves realistic
problems using complex models and that is easy to use. Thus, we allow users to
input models in physical or non-dimensional units, at their discretion. We
believe that this makes the description of realistic models simpler. On
the other hand, ensuring numerical stability is not something users should have
to be concerned about, and is taken care of in the implementation of \aspect{}'s
core (see the corresponding section in \cite{KHB12}).



\subsection{Static or dynamic pressure?}
\label{sec:pressure-static-dyn}

One could reformulate equation \eqref{eq:stokes-1} somewhat. To this end, let us
say that we would want to represent the pressure $p$ as the sum of two parts
that we will call static and dynamic, $p=p_s+p_d$. If we assume that $p_s$ is
already given, then we can replace \eqref{eq:stokes-1} by
\begin{gather*}
  -\nabla \cdot 2\eta
  \nabla \mathbf u + \nabla p_d =
  \rho\mathbf g - \nabla p_s.
\end{gather*}
One typically chooses $p_s$ as the pressure one would get if the whole medium
were at rest -- i.e., as the hydrostatic pressure. This pressure can be
computed noting that \eqref{eq:stokes-1} reduces to
\begin{gather*}
  \nabla p_s = \rho(p_s,T_s,\mathbf x)\mathbf g = \bar\rho \mathbf g
\end{gather*}
in the absence of any motion where $T_s$ is some static temperature field (see
also Section~\ref{sec:adiabatic}). This, our rewritten version of
\eqref{eq:stokes-1} would look like this:
\begin{gather*}
  -\nabla \cdot 2\eta
  \nabla \mathbf u + \nabla p_d =
  \left[\rho(p,T,\mathbf x)-\rho(p_s,T_s,\mathbf x)\right]\mathbf g.
\end{gather*}
In this
formulation, it is clear that the quantity that drives the fluid flow is in
fact the \textit{buoyancy} caused by the \textit{variation} of densities,
not the density itself.

This reformulation has a number of advantages and disadvantages:
\begin{itemize}
\item One can notice that in many realistic cases, the dynamic component $p_d$
  of the pressure is orders of magnitude smaller than the static component
  $p_s$. For example, in the earth, the two are separated by around 6 orders
  of magnitude at the bottom of the earth mantle. Consequently, if one wants
  to solve the linear system that arises from discretization of the original
  equations, one has to solve it a significant degree of accuracy (6--7
  digits) to get the dynamic part of the pressure correct to even one
  digit. This entails a very significant numerical effort, and one that is not
  necessary if we can split the pressure in a way so that the pre-computed
  static pressure $p_s$ (or, rather, the density using the static pressure and
  temperature from which $p_s$ results) absorbs the dominant part and one only
  has to compute the remaining, dynamic pressure to 2 or 3 digits of accuracy,
  rather than the corresponding 7--8 for the total pressure.

\item On the other hand, the pressure $p_d$ one computes this way is not immediately
  comparable to quantities that we use to look up pressure-dependent
  quantities such as the density. Rather, one needs to first find the static
  pressure as well (see Section~\ref{sec:adiabatic}) and add the two together
  before they can be used to look up material properties or to compare them with
  experimental results. Consequently, if the pressure a program outputs
  (either for visualization, or in the internal interfaces to parts of the
  code where users can implement pressure- and temperature-dependent material
  properties) is only the dynamic component, then all of the consumers of this
  information need to convert it into the total pressure when comparing with
  physical experiments. Since any code implementing realistic material models
  has a great many of these places, there is a large potential for inadvertent
  errors and bugs.

\item Finally, the definition of a reference density $\rho(p_s,T_s,\mathbf x)$
  derived from static pressures and temperatures
  is only simple if we have incompressible models and under the assumption
  that the temperature-induced density variations are small compared to the
  overall density. In this case, we can choose $\rho(p_s,T_s,\mathbf
  x)=\rho_0$ with a constant reference density $\rho_0$. On the other hand,
  for more complicated models, it is not a priori
  clear which density to choose since we first need to compute static
  pressures and temperatures -- quantities that satisfy equations that
  introduce boundary layers, may include phase changes releasing latent heat,
  and where the density may have discontinuities at certain depths, see
  Section~\ref{sec:adiabatic}.

  Thus, if we compute adiabatic pressures and
  temperatures $\bar p_s,\bar T_s$ under the assumption of a thermal boundary layer
  worth 900 Kelvin at the top, and we get a corresponding density profile
  $\bar\rho=\rho(\bar p_s,\bar T_s, \mathbf x)$, but after running for a few
  million years the temperature turns out to be so that the top boundary layer
  has a jump of only 800 Kelvin with corresponding adiabatic pressures and
  temperatures $\hat p_s,\hat T_s$, then a more appropriate density profile
  would be $\hat\rho=\rho(\hat p_s,\hat T_s, \mathbf x)$.

  The problem is that it may well be that the erroneously computed density
  profile $\hat \rho$ does \textit{not} lead to a separation where
  $|p_d|\ll|p_s|$ because, especially if the material undergoes phase changes,
  there will be entire areas of the computational domain in which $|\rho-\hat
  \rho_s|\ll |\rho|$ but $|\rho-\bar
  \rho_s|\not\ll |\rho|$. Consequently the benefits of lesser requirements on the
  iterative linear solver would not be realized.
\end{itemize}

We do note that most of the codes available today and that we are aware of
split the pressure into static and dynamic parts nevertheless, either
internally or require the user to specify the density profile as the
difference between the true and the hydrostatic density. This may, in part, be
due to the fact that historically most codes were written to solve problems
in which the medium was considered incompressible, i.e., where the definition
of a static density was simple.

On the other hand, we intend \aspect{} to be a code that can solve more
general models for which this definition is not as simple. As a consequence, we
have chosen to solve the equations as stated originally -- i.e., we solve for
the \textit{full} pressure rather than just its \textit{dynamic} component. With
most traditional methods, this would lead to a catastrophic loss of accuracy in the
dynamic pressure since it is many orders of magnitude smaller than the total
pressure at the bottom of the earth mantle. We avoid this problem in \aspect{}
by using a cleverly chosen iterative solver that ensures that the full pressure
we compute is accurate enough so that the dynamic pressure can be extracted from
it with the same accuracy one would get if one were to solve for only the
dynamic component. The methods that ensure this are described in detail in
\cite{KHB12} and in particular in the appendix of that paper.

\note{By default, \aspect{} uses the full pressure in the equations, and only prescribing 
density deviations from a reference state on the right-hand side of \eqref{eq:stokes-1} 
would lead to negative densities in the energy equation \eqref{eq:temperature}.  
However, when using one of the approximations described in Section \ref{sec:approximate-equations}, 
the energy balance uses the reference density $\bar\rho$ instead of the full density, 
which makes it possible to formulate the Stokes system in terms of the dynamic instead of
the full pressure. In order to do this, one would have to use a material model 
(see Section~\ref{sec:material-models}) in which the density is in fact a density variation, 
and then the pressure solution variable would only be the dynamic pressure.}

\subsection{Pressure normalization}
\label{sec:pressure}

The equations described above, \eqref{eq:stokes-1}--\eqref{eq:temperature},
only determine the pressure $p$ up to an additive constant. On the other hand,
since the pressure appears in the definition of many of the coefficients, we
need a pressure that has some sort of \textit{absolute} definition. A
physically useful definition would be to normalize the pressure in such a way
that the average pressure along the ``surface'' has a prescribed value where
the geometry description (see Section~\ref{sec:geometry-models}) has to
determine which part of the boundary of the domain is the ``surface'' (we call
a part of the boundary the ``surface'' if its depth is ``close to zero'').

Typically, one will choose this average pressure to be zero, but there is a
parameter ``\texttt{Surface pressure}''
\index[prmindex]{Surface pressure}
\index[prmindexfull]{Surface pressure}
in the input file (see Section~\ref{parameters:global}) to set it to
a different value. One may want to do that, for example, if one wants to
simulate the earth mantle without the overlying lithosphere. In that case, the
``surface'' would be the interface between mantle and lithosphere, and the
average pressure at the surface to which the solution of the equations will be
normalized should in this case be the hydrostatic pressure at the bottom of
the lithosphere.

An alternative is to normalize the pressure in such a way that the
\textit{average} pressure throughout the domain is zero or some constant
value. This is not a useful approach for most geodynamics applications but is
common in benchmarks for which analytic solutions are available. Which kind of
normalization is chosen is determined by the ``\texttt{Pressure
  normalization}'' flag in the input file,
\index[prmindex]{Pressure normalization}
\index[prmindexfull]{Pressure normalization}
see Section~\ref{parameters:global}.


\subsection{Initial conditions and the adiabatic pressure/temperature}
\label{sec:adiabatic}

Equations \eqref{eq:stokes-1}--\eqref{eq:temperature} require us to
pose initial conditions for the temperature, and this is done by
selecting one of the existing models for initial conditions in the
input parameter file, see
Section~\ref{parameters:Initial_20temperature_20model}. The equations
themselves do not require that initial conditions are specified for
the velocity and pressure variables (since there are no time
derivatives on these variables in the model).

Nevertheless, a nonlinear solver will have difficulty converging to
the correct solution if we start with a completely unphysical pressure
for models in which coefficients such as density $\rho$ and viscosity
$\eta$ depend on the pressure and temperature. To this end, \aspect{} 
uses pressure and temperature fields $p_{\textrm{ad}}(z),
T_{\textrm{ad}}(z)$ computed in the adiabatic conditions model 
(see Section~\ref{parameters:Adiabatic_20conditions_20model}).
By default, these fields satisfy adiabatic conditions:
\begin{align}
  \rho C_p \frac{\textrm{d}}{\textrm{d}z} T_{\textrm{ad}}(z)
  &=
  \frac{\partial\rho}{\partial T} T_{\textrm{ad}}(z) g_z,
\\
  \frac{\textrm{d}}{\textrm{d}z} p_{\textrm{ad}}(z)
  &=
  \rho g_z,
\end{align}
where strictly speaking $g_z$ is the magnitude of the vertical
component of the gravity vector field, but in practice we take the
magnitude of the entire gravity vector.

These equations can be integrated numerically starting at $z=0$, using
the depth dependent gravity field and values of the coefficients
$\rho=\rho(p,T,z), C_p=C_p(p,T,z)$. As starting conditions at $z=0$ we
choose a pressure $p_{\textrm{ad}}(0)$ equal to the average surface
pressure (often chosen to be zero, see Section~\ref{sec:pressure}),
and an adiabatic surface temperature $T_{\textrm{ad}}(0)$ that is
\index[prmindex]{Adiabatic surface temperature}
\index[prmindexfull]{Adiabatic surface temperature}
also selected in the input parameter file.

However, users can also supply their own adiabatic conditions models or 
define an arbitrary profile using the ``function'' plugin.

\note{The adiabatic surface temperature is often chosen significantly
  higher than the actual surface temperature. For example, on earth,
  the actual surface temperature is on the order of 290 K, whereas a
  reasonable adiabatic surface temperature is maybe 1600 K. The reason
  is that the bulk of the mantle is more or less in thermal equilibrium
  with a thermal profile that corresponds to the latter temperature,
  whereas the very low actual surface temperature and the very high
  bottom temperature at the core-mantle boundary simply induce a
  thermal boundary layer. Since the temperature and pressure profile
  we compute using the equations above are simply meant to be good
  starting points for nonlinear solvers, it is important to choose
  this profile in such a way that it covers most of the mantle well;
  choosing an adiabatic surface temperature of 290 K would yield a
  temperature and pressure profile that is wrong almost throughout the
  entire mantle.}



\subsection{Compositional fields}
\label{sec:compositional}

The last of the basic equations, \eqref{eq:compositional}, describes the
evolution of a set of variables $c_i(\mathbf x, t), i=1\ldots C$ that we
typically call \textit{compositional fields} and that we often aggregate into
a vector $\mathfrak c$.

Compositional fields were originally intended to track what their name
suggest, namely the chemical composition of the convecting medium. In this
interpretation, the composition is a non-diffusive quantity that is simply advected along
passively, i.e., it would satisfy the equation
\begin{align*}
  \frac{\partial \mathfrak c}{\partial t} + \mathbf u \cdot \nabla \mathfrak c
  = 0.
\end{align*}
However, the compositional fields may also participate in determining the values of
the various coefficients as discussed in
Section~\ref{sec:coefficients}, and in this sense the equation above
describes a composition that is \textit{passively advected}, but an
\textit{active participant} in the equations.

That said, over time compositional fields have shown to be a much more useful
tool than originally intended. For example, they can be used to track where
material comes from and goes to (see Section~\ref{sec:cookbooks-composition})
and, if one allows for a reaction rate $\mathfrak q$ on the right hand side,
\begin{align*}
  \frac{\partial \mathfrak c}{\partial t} + \mathbf u \cdot \nabla \mathfrak c
  = \mathfrak q,
\end{align*}
then one can also model interaction between species -- for example to simulate
phase changes where one compositional field, indicating a particular phase,
transforms into another phase depending on pressure and temperature, or where
several phases combine to other phases. Another example of using a
right hand side -- quite outside what the original term
\textit{compositional field} was supposed to indicate -- is to track
the accumulation of finite strain, see Section~\ref{sec:finite-strain}.

In actual practice, one finds that it is often useful to allow
$\mathfrak q$ to be a function that has both a smooth (say,
continuous) in time component, and one that is singular in time (i.e.,
contains Dirac delta, or ``impulse'' functions). Typical time
integrators require the evaluation of the right hand side at specific
points in time, but this would preclude the use of delta
functions. Consequently, the integrators in \aspect{} only require
material models to provide an \textit{integrated} value
$\int_t^{t+\Delta t} \mathfrak q(\tau) \;
\text{d}\tau$ through the {\tt reaction\_term} output
variable. Implementations often approximate this as $\triangle t \cdot
\mathfrak q(t)$, or similar formulas.

A second application for only providing integrated right hand sides
comes from the fact that
modeling reactions between different compositional fields often involves
finding an equilibrium state between different fields because
chemical reactions happen on a much faster time scale than transport. In other
words, one then often assumes that there is a $\mathfrak c^\ast(p,T)$ so that
\begin{align*}
  \mathfrak q(p,T,\varepsilon(\mathbf u),\mathfrak c^\ast(p,T)) = 0.
\end{align*}
Consequently, the material model methods that deal with source terms for the
compositional fields need to compute an \textit{increment} $\Delta\mathfrak c$
to the previous value of the compositional fields so that the sum of the
previous values and the increment equals $\mathfrak c^\ast$. This
corresponds to an \textit{impulse change} in the compositions at every
time step, as opposed
to the usual approach of evaluating the right hand side term
$\mathfrak q$ as a continuous function in time,
which corresponds to a \textit{rate}.

On the other hand, there are other uses of compositional fields that do not
actually have anything to do with quantities that can be considered related to
compositions. For example, one may define a field that tracks the grain size
of rocks. If the strain rate is high, then the grain size decreases as the
rocks break. If the temperature is high enough, then grains heal and their size
increases again. Such ``damage'' models would then call for an equation of the
form (assuming one uses only a single compositional field)
\begin{align*}
  \frac{\partial c}{\partial t} + \mathbf u \cdot \nabla c
  = q(T,c),
\end{align*}
where in the simplest case one could postulate
\begin{align*}
  q(T,c) = -A c + B \max\{T-T_{\text{healing}},0\} c.
\end{align*}
One would then use this compositional field in the definition of the viscosity
of the material: more damage means lower viscosity because the rocks are weaker.

In cases like this, there is only a single compositional field and it is not
in permanent equilibrium. Consequently, the increment implementations of
material models in \aspect{} need to compute is typically the rate $q(T,c)$
times the time step.  In other words, if you compute a reaction rate inside the material model you need to multiply it by the time step size before returning the value.

Compositional fields have proven to be surprisingly versatile tools to model
all sorts of components of models that go beyond the simple Stokes plus
temperature set of equations. Play with them!


\subsection{Constitutive laws}

Equation \eqref{eq:stokes-1} describes buoyancy-driven flow in an isotropic
fluid where strain rate is related to stress by a scalar (possibly spatially variable)
multiplier, $\eta$. For some material models it is useful to generalize this
relationship to anisotropic materials, or other exotic constitutive laws.
For these cases \aspect{} can optionally include a generalized, fourth-order
tensor field as a material model state variable which changes equation
\eqref{eq:stokes-1} to
\begin{align}
  \label{eq:stokes-1-anisotropic}
  -\nabla \cdot \left[2\eta \left(C \varepsilon(\mathbf u)
                                  - \frac{1}{3}(tr(C \varepsilon(\mathbf u)))\mathbf 1\right)
                \right] + \nabla p &=
  \rho \mathbf g
  & \qquad
  & \textrm{in $\Omega$}
\end{align}
and the shear heating term in equation \eqref{eq:temperature} to
\begin{align}
  \label {eq:temperature-anisotropic}
  \dots
  \notag
  \\
  + 2 \eta
  \left(C \varepsilon(\mathbf u) - \frac{1}{3}(tr(C \varepsilon(\mathbf u)))\mathbf 1\right)
  :
  \left(\varepsilon(\mathbf u) - \frac{1}{3}(\nabla \cdot \mathbf u)\mathbf 1\right)
  \\
  \dots
  \notag
\end{align}
where $C = C_{ijkl}$ is defined by the material model. For physical reasons, $C$ needs
to be a symmetric rank-4 tensor: i.e., when multiplied by a symmetric (strain rate)
tensor of rank 2 it needs to return another symmetric tensor of rank 2. In mathematical
terms, this means that $C_{ijkl}=C_{jikl}=C_{ijlk}=C_{jilk}$. Energy considerations
also require that $C$ is positive definite: i.e., for any $\varepsilon \neq 0$, the
scalar $\varepsilon : (C \varepsilon)$ must be positive.

This functionality can be optionally invoked by any material model that chooses to
define a $C$ field, and falls back to the default case ($C=\mathbb I$) if no such
field is defined. It should be noted that $\eta$ still appears in equations
\eqref{eq:stokes-1-anisotropic} and \eqref{eq:temperature-anisotropic}. $C$ is
therefore intended to be thought of as a ``director'' tensor rather than a
replacement for the viscosity field, although in practice either interpretation
is okay.


\subsection{Numerical methods}

There is no shortage in the literature for methods to solve the equations
outlined above. The methods used by \aspect{} use the following,
interconnected set of strategies in the implementation of numerical
algorithms:
\begin{itemize}
\item \textit{Mesh adaptation:} Mantle convection problems are characterized
  by widely disparate length scales (from plate boundaries on the order of
  kilometers or even smaller, to the scale of the entire earth). Uniform
  meshes can not resolve the smallest length scale without an intractable
  number of unknowns.  Fully adaptive meshes allow resolving local features of
  the flow field without the need to refine the mesh globally. Since the
  location of plumes that require high resolution change and move with time,
  meshes also need to be adapted every few time steps.
\item \textit{Accurate discretizations:} The equations upon which
  most models for the earth mantle are based
  have a number of intricacies that make the choice of discretization
  non-trivial. In particular, the finite elements chosen for velocity and
  pressure need to satisfy the usual compatibility condition for saddle point
  problems. This can be worked around using pressure stabilization schemes for
  low-order discretizations, but high-order methods can yield better accuracy
  with fewer unknowns and offer more reliability. Equally important is the choice of
  a stabilization method for the highly advection-dominated temperature
  equation. \aspect{} uses a nonlinear artificial diffusion method for the latter.
\item \textit{Efficient linear solvers:} The major obstacle in solving the
  system of linear equations that results from discretization is the
  saddle-point nature of the Stokes equations.
  Simple linear solvers and preconditioners can not efficiently solve this system in
  the presence of strong heterogeneities or when the size of the system
  becomes very large. \aspect{} uses an efficient solution strategy based on a
  block triangular preconditioner utilizing an algebraic multigrid that
  provides optimal complexity even up to problems with hundreds of millions of
  unknowns.
\item \textit{Parallelization of all of the steps above:} Global mantle convection
  problems frequently require extremely large numbers of unknowns for
  adequate resolution in three dimensional simulations. The only realistic way to solve such problems lies in
  parallelizing computations over hundreds or thousands of processors. This is
  made more complicated by the use of dynamically changing meshes, and it
  needs to take into account that we want to retain the optimal complexity of
  linear solvers and all other operations in the program.
\item \textit{Modularity of the code:} A code that implements all of these
  methods from \textit{scratch} will be unwieldy, unreadable and unusable as a community
  resource. To avoid this, we build our implementation on widely used and well
  tested libraries that can provide researchers interested in extending it
  with the support of a large user community. Specifically, we use the
  \dealii{} library \cite{BHK07,BK99m} for meshes, finite
  elements and everything discretization related; the \trilinos{} library
  \cite{trilinos,trilinos-web-page} for scalable and parallel linear algebra;
  and \pfrst{} \cite{p4est} for distributed, adaptive meshes. As a
  consequence, our code is freed of the mundane tasks of defining finite
  element shape functions or dealing with the data structures of linear algebra,
  can focus on the high-level description of what is supposed to happen, and
  remains relatively compact. The code will also
  automatically benefit from improvements to the underlying libraries with
  their much larger development communities. \aspect{} is extensively
  documented to enable other researchers to understand, test, use, and extend it.
\end{itemize}

Rather than detailing the various techniques upon which \aspect{} is built, we
refer to the paper by Kronbichler, Heister and Bangerth \cite{KHB12} that
gives a detailed description and rationale for the various building blocks.


\subsection{Approximate equations}
\label{sec:approximate-equations}

There are a number of common variations to equations
\eqref{eq:stokes-1}--\eqref{eq:temperature} that are frequently used in the
geosciences. For example, one frequently finds references to the anelastic liquid
approximation (ALA), truncated anelastic liquid approximation (TALA), and the
Boussinesq approximation (BA). These can all be derived from the basic
equations~\eqref{eq:stokes-1}--\eqref{eq:temperature} via various approximations, 
and we will discuss them in the following. Since they are typically only provided
considering velocity, pressure and temperature, we will in the following omit
the dependence on the compositional fields used in previous sections, though
this dependence can easily be added back into the equations stated below. A
detailed discussion of the approximations introduced below can also be found in
\cite{KLKLZTTK10}.

The three approximations mentioned all start by writing the pressure and
temperature as the sum of a (possibly depth dependent)
reference state plus a perturbation, i.e., we will write
\begin{align*}
  p(\mathbf x,t) &= \bar p(z) + p'(\mathbf x,t),
  \\
  T(\mathbf x,t) &= \bar T(z) + T'(\mathbf x,t).
\end{align*}
Here, barred quantities are reference states and may depend on the depth $z$
(not necessarily the third component of $\mathbf x$) whereas primed quantities
are the spatially and temporally variable deviations of the temperature and
pressure fields from this reference state. In particular, the reference pressure
is given by solving the hydrostatic equation,
\begin{align}
\label{eq:hydrostatic-pressure}
  \nabla \bar p = \bar\rho \mathbf g,
\end{align}
where $\bar\rho=\rho(\bar p,\bar T)$ is a \textit{reference density} that
depends on depth and represents a typical change of material parameters and solution 
variables with depth. $\bar T(z)$ is chosen as an adiabatic profile accounting for the
fact that the temperature increases as the pressure increases.
With these definitions, equations \eqref{eq:stokes-1}--\eqref{eq:stokes-2} can equivalently be written as follows:
\begin{align}
  \label{eq:stokes-decomposed-1}
  -\nabla \cdot \left[2\eta \left(\varepsilon(\mathbf u)
                                  - \frac{1}{3}(\nabla \cdot \mathbf u)\mathbf 1\right)
                \right] + \nabla p' &=
  (\rho-\bar\rho) \mathbf g
  & \qquad
  & \textrm{in $\Omega$},
  \\
  \label{eq:stokes-decomposed-2}
  \nabla \cdot (\rho \mathbf u) &= 0
  & \qquad
  & \textrm{in $\Omega$}.
\end{align}
The temperature equation, when omitting entropic effects, still reads as
\begin{multline}
  \label{eq:temperature-decomposed}
  \rho C_p \left(\frac{\partial T}{\partial t} + \mathbf u\cdot\nabla T\right)
  - \nabla\cdot k\nabla T
  \\
  =
  \rho H
  +
  2\eta
  \left(\varepsilon(\mathbf u) - \frac{1}{3}(\nabla \cdot \mathbf u)\mathbf 1\right)
  :
  \left(\varepsilon(\mathbf u) - \frac{1}{3}(\nabla \cdot \mathbf u)\mathbf 1\right)
  +\alpha T \left( \mathbf u \cdot \nabla p \right)
  \quad
  \textrm{in $\Omega$},
\end{multline}
where the right-hand side includes radiogenic heat production, shear heating and adiabatic heating (in that order).
  
Starting from these equations, the approximations discussed in the next few
subsections make use of the fact that for the flows for which these approximations are valid, the
perturbations $p'$, $T'$ are much smaller than typical values of the reference
quantities $\bar p$, $\bar T$. 
The terms influenced by these approximations are $\nabla \cdot (\rho u) =0$ in the 
continuity equation, and all occurrences of $\rho(p,T)$ in the temperature equation, 
and we will discuss them separately below. The equations for these approximations are
almost always given in terms of non-dimensionalized quantities. We will for
now stick with the dimensional form because it expresses in a clearer way the
approximations that are made. The non-dimensionalization can then be done on
each of the forms below separately.

\subsubsection{The anelastic liquid approximation (ALA)}
\label{sec:ala}

The \textit{anelastic liquid approximation (ALA)} is based on two assumptions.
First, that the density variations $\rho(p,T)-\bar\rho$ are small and in
particular can be accurately described by a Taylor expansion:
\begin{align*}
  \rho(p,T) \approx 
  \bar\rho 
  + \frac{\partial \rho(\bar p,\bar T)}{\partial p} p'
  + \frac{\partial \rho(\bar p,\bar T)}{\partial T} T'.
\end{align*}
Here, $\frac{\partial \rho(\bar p,\bar T)}{\partial T}$ is related to the
thermal expansion coefficient $\alpha = -\frac{1}{\rho}\frac{\partial \rho}{\partial T}$, 
and $\frac{\partial \rho(\bar p,\bar T)}{\partial p}$ to the compressibility
$\kappa = \frac{1}{\rho}\frac{\partial \rho}{\partial p}$.

The second assumption is that the variation of the density
from the reference density can be neglected in the mass balance and
temperature equations.
This yields the following system of equations for the velocity and pressure
equations:
\begin{align}
  \label{eq:stokes-ALA-1}
  -\nabla \cdot \left[2\eta \left(\varepsilon(\mathbf u)
                                  - \frac{1}{3}(\nabla \cdot \mathbf u)\mathbf 1\right)
                \right] + \nabla p' &=
  \left(\frac{\partial \rho(\bar p,\bar T)}{\partial p} p'
  + \frac{\partial \rho(\bar p,\bar T)}{\partial T} T' \right) \mathbf g
  & \qquad
  & \textrm{in $\Omega$},
  \\
  \label{eq:stokes-ALA-2}
  \nabla \cdot (\bar\rho \mathbf u) &= 0
  & \qquad
  & \textrm{in $\Omega$}.
\end{align}

For the temperature equation, using the definition of the hydrostatic pressure gradient \eqref{eq:hydrostatic-pressure}, we arrive at the following:

\begin{multline}
  \label{eq:temperature-ala}
  \bar\rho C_p \left(\frac{\partial T}{\partial t} + \mathbf u\cdot\nabla
  T\right) - \nabla\cdot k\nabla T
  \\
  =
  \bar\rho H
  +
  2\eta
  \left(\varepsilon(\mathbf u) - \frac{1}{3}(\nabla \cdot \mathbf u)\mathbf 1\right)
  :
  \left(\varepsilon(\mathbf u) - \frac{1}{3}(\nabla \cdot \mathbf u)\mathbf 1\right)
  +\alpha \bar\rho T (\mathbf u \cdot \mathbf g)
  \quad
  \textrm{in $\Omega$}.
\end{multline}

\note{Our energy equation is formulated in terms of $T$, while in the literature, the equation
has sometimes been formulated in terms of $T'$, which yields additional terms
containing $\bar T$ on the right-hand side.
Both ways of writing the equation are equivalent.}

\subsubsection{The truncated anelastic liquid approximation (TALA)}
\label{sec:tala}

The \textit{truncated anelastic liquid approximation (TALA)} further simplifies
the ALA by assuming that the variation of the density due to pressure variations
is small, i.e., that
\begin{align*}
  \rho(p,T) \approx 
  \bar\rho 
  + \frac{\partial \rho(\bar p,\bar T)}{\partial T} T'.
\end{align*}
This does not mean that the density is not pressure dependent -- it will, for
example, continue to be depth dependent because the hydrostatic pressure grows
with depth. It simply means that the deviations from the reference pressure are
assumed to be so small that they do not matter in describing the density.
Because the pressure variation $p'$ is induced by the flow field (the static
component pressure is already taken care of by the hydrostatic pressure), this
assumption in essence means that we assume the flow to be very slow, even beyond
the earlier assumption that we can neglect inertial terms when
deriving~\eqref{eq:stokes-1}--\eqref{eq:stokes-2}.

This further assumption then
transforms~\eqref{eq:stokes-ALA-1}--\eqref{eq:stokes-ALA-2} into the following
equations:
\begin{align}
  \label{eq:stokes-TALA-1}
  -\nabla \cdot \left[2\eta \left(\varepsilon(\mathbf u)
                                  - \frac{1}{3}(\nabla \cdot \mathbf u)\mathbf 1\right)
                \right] + \nabla p' &=
  \frac{\partial \rho(\bar p,\bar T)}{\partial T} T' \mathbf g
  & \qquad
  & \textrm{in $\Omega$},
  \\
  \label{eq:stokes-TALA-2}
  \nabla \cdot (\bar\rho \mathbf u) &= 0
  & \qquad
  & \textrm{in $\Omega$}.
\end{align}
The energy equation is the same as in the ALA case. 

\subsubsection{The Boussinesq approximation (BA)}
\label{sec:Boussinesq}

If we further assume that the reference temperature and the reference density are constant, 
$\bar T(z)=T_0$, $\bar\rho(\bar p,\bar T)=\rho_0$, 
-- in other words, density variations are so small that 
they are negligible everywhere except for in the right-hand side of the velocity 
equation (the buoyancy term), which describes the driving force of the flow, 
then we can further simplify the mass conservation equations of the
TALA to $\nabla \cdot \mathbf u=0$. 
This means that the density in all other parts of the equations is not only independent of
the pressure variations $p'$ as assumed in the TALA, but also does not depend on
the much larger hydrostatic pressure $\bar p$ nor on the reference temperature
$\bar T$. We then obtain the following set of
equations that also uses the incompressibility in the definition of the strain rate:
\begin{align}
  \label{eq:stokes-BA-1}
  -\nabla \cdot \left[2\eta \varepsilon(\mathbf u)
                \right] + \nabla p' &=
  \frac{\partial \rho(\bar p,\bar T)}{\partial T} T' \mathbf g
  & \qquad
  & \textrm{in $\Omega$},
  \\
  \label{eq:stokes-BA-2}
  \nabla \cdot \mathbf u &= 0
  & \qquad
  & \textrm{in $\Omega$}.
\end{align}
In addition, as the reference temperature is constant, one needs to neglect the
adiabatic and shear heating in the energy equation
\begin{equation}
  \label{eq:temperature-BA}
  \bar\rho C_p \left(\frac{\partial T}{\partial t} + \mathbf u\cdot\nabla
  T\right) - \nabla\cdot k\nabla T
  =
  \bar\rho H
  \quad
  \textrm{in $\Omega$}.
\end{equation}

\paragraph*{On incompressibility.}

The Boussinesq approximation assumes that the density can be
considered constant in all occurrences in the equations with the exception of
the buoyancy term on the right hand side of \eqref{eq:stokes-1}. The primary
result of this assumption is that the continuity equation \eqref{eq:stokes-2}
will now read
\begin{gather*}
  \nabla \cdot \mathbf u = 0.
\end{gather*}
This makes the equations \textit{much} simpler to solve: First, because the
divergence operation in this equation is the transpose of the gradient of the
pressure in the momentum equation \eqref{eq:stokes-1}, making the system of
these two equations symmetric. And secondly, because the two equations are now
linear in pressure and velocity (assuming that the viscosity $\eta$ and the
density $\rho$ are considered fixed). In addition, one can drop all terms
involving $\nabla \cdot \mathbf u$ from the left hand side of the momentum
equation \eqref{eq:stokes-1}; while dropping these terms does not
affect the solution of the equations, it makes assembly of linear systems
faster. 

From a physical perspective, the assumption that the density is constant in
the continuity equation but variable in the momentum equation is of course
inconsistent. However, it is justified if the variation is small since the
momentum equation can be rewritten to read
\begin{gather*}
  -\nabla \cdot 2\eta \varepsilon(\mathbf u) + \nabla p' =
  (\rho-\rho_0) \mathbf g,
\end{gather*}
where $p'$ is the \textit{dynamic} pressure and $\rho_0$ is the constant
reference density. This makes it clear that the true driver of motion is in
fact the \textit{deviation} of the density from its background value, however
small this value is: the resulting velocities are simply proportional to the
density variation, not to the absolute magnitude of the density.

As such, the Boussinesq approximation can be justified. On the other hand,
given the real pressures and temperatures at the bottom of the Earth's mantle,
it is arguable whether the density can be considered to be almost
constant. Most realistic models predict that the density of mantle rocks
increases from somewhere around 3300 at the surface to over 5000 kilogram per
cubic meters at the core mantle boundary, due to the increasing lithostatic
pressure. While this appears to be a large variability, if the density changes
slowly with depth, this is not in itself an indication that the Boussinesq
approximation will be wrong. To this end, consider that the continuity
equation can be rewritten as $\frac 1\rho \nabla \cdot (\rho \mathbf u)=0$,
which we can multiply out to obtain
\begin{gather*}
  \nabla \cdot \mathbf u
  +
  \frac 1\rho \mathbf u \cdot \nabla \rho
  = 0.
\end{gather*}
The question whether the Boussinesq approximation is valid is then whether the
second term (the one omitted in the Boussinesq model) is small compared to the
first. To this end, consider that the velocity can change completely over length
scales of maybe 10 km, so that $\nabla \cdot\mathbf u \approx \|u\| /
10\text{km}$. On the other hand, given a smooth dependence of density on pressure,
the length scale for variation of the density is the entire earth mantle,
i.e., $\frac 1\rho \mathbf u \cdot \nabla\rho \approx \|u\| 0.5 / 3000 \text{km}$
(given a variation between minimal and maximal density of 0.5 times the
density itself). In other words, for a smooth variation, the contribution of
the compressibility to the continuity equation is very small. This may be
different, however, for models in which the density changes rather abruptly,
for example due to phase changes at mantle discontinuities.

\paragraph{On almost linear models.}

A further simplification can be obtained if one assumes that all coefficients
with the exception of the density do not depend on the solution variables but
are, in fact, constant. In such models, one typically assumes that the density
satisfies a relationship of the form $\rho=\rho(T)=\rho_0(1-\alpha(T-T_0))$
with a small thermal expansion coefficient $\alpha$ and a reference density
$\rho_0$ that is attained at temperature $T_0$. Since the thermal expansion is
considered small, this naturally leads to the following variant of the Boussinesq
model discussed above:
\begin{align}
  \label{eq:stokes-1-Boussinesq-linear}
  -\nabla \cdot \left[2\eta \varepsilon(\mathbf u)
                \right] + \nabla p' &=
  -\alpha\rho_0 T \mathbf g
  & \qquad
  & \textrm{in $\Omega$},
  \\
  \label{eq:stokes-2-Boussinesq-linear}
  \nabla \cdot \mathbf u &= 0
  & \qquad
  & \textrm{in $\Omega$},
  \\
  \label{eq:temperature-Boussinesq-linear}
  \rho_0 C_p \left(\frac{\partial T}{\partial t} + \mathbf u\cdot\nabla T\right)
  - \nabla\cdot k\nabla T
  &=
  \rho H
  & \quad
  & \textrm{in $\Omega$}.
\end{align}
Note that the right hand side forcing term
in \eqref{eq:stokes-1-Boussinesq-linear} is now only the deviation of the
gravitational force from the force that would act if the material were at
temperature $T_0$.

Under the assumption that all other coefficients are constant, one then
arrives at equations in which the only nonlinear term is the advection term,
$\mathbf u \cdot \nabla T$ in the temperature equation
\eqref{eq:temperature-Boussinesq-linear}. This facilitates the use of a
particular class of time stepping schemes in which one does not solve the whole
set of equations at once, iterating out nonlinearities as necessary, but
instead in each time step solves first the Stokes system with the previous
time step's temperature, and then uses the so-computed velocity to solve the
temperature equation. These kind of time stepping schemes are often referred
to as \textit{operator splitting} methods. 

\note{\aspect{} does not solve the equations in the way described in this paragraph, 
however, a particular operator splitting method was used in
earlier \aspect{} versions. It first solves the Stokes equations and then
uses a semi-explicit time stepping method for the temperature equation
where diffusion is handled implicitly and advection explicitly.
This algorithm is often called \textit{IMPES} (it originated in the
porous media flow 
community, where the acronym stands for \textit{Im}plicit \textit{P}ressure,
\textit{E}xplicit \textit{S}aturation) and is explained in more detail
in \cite{KHB12}. Since then the algorithm in \aspect{} has
been rewritten to use an implicit time stepping algorithm also for the
temperature equation because this allows to use larger time steps.}


\subsubsection{The isothermal compression approximation (ICA)}
\label{sec:ica}

In the compressible case and without the assumption of a reference state, 
the conservation of mass equation in equation~\eqref{eq:stokes-2} is $\nabla 
\cdot \left( \rho \textbf{u} \right)= 0$, which is nonlinear and not symmetric to the $\nabla p$ term in the
force balance equation \eqref{eq:stokes-1}, making solving and preconditioning
the resulting linear and nonlinear systems difficult. To make this work in
\aspect{}, we consequently reformulate this equation. Dividing by $\rho$ and
applying the product rule of differentiation gives
\begin{equation*}
\frac{1}{\rho} \nabla \cdot \left( \rho \textbf{u} \right) = \nabla \cdot \textbf{u} + \frac{1}{\rho} \nabla \rho \cdot  \textbf{u}.
\end{equation*}
We will now make two basic assumptions: First, the variation of the density
$\rho(p,T,\mathbf x, \mathfrak c)$ is dominated by the dependence on the
(total) pressure; in other words, $\nabla \rho \approx \frac{\partial \rho}{\partial
  p}\nabla p$. This assumption is primarily justified by the fact that, in the
Earth's mantle, the density increases by at least 50\% between Earth's crust and
the core-mantle boundary due to larger pressure there. Secondly, we assume
that the pressure is dominated by the static pressure, which implies that
$\nabla p \approx \nabla p_s \approx \rho \textbf{g}$. This is justified, 
because the viscosity in the Earth is large and velocities are small, 
hence $\nabla p' \ll \nabla p_s$.
This finally allows us to write
\begin{equation*}
\frac{1}{\rho} \nabla \rho \cdot \textbf{u} \approx \frac{1}{\rho} \frac{\partial \rho}{\partial p} \nabla p \cdot \textbf{u} \approx \frac{1}{\rho} \frac{\partial \rho}{\partial p} \nabla p_s \cdot \textbf{u} \approx \frac{1}{\rho} \frac{\partial \rho}{\partial p} \rho \textbf{g} \cdot \textbf{u} 
\end{equation*}
so we get
\begin{equation}
\label{eq:stokes-2-compressible}
\nabla \cdot \textbf{u} = -\frac{1}{\rho} \frac{\partial \rho}{\partial p} \rho \textbf{g} \cdot \textbf{u}
\end{equation}
where $\frac{1}{\rho} \frac{\partial \rho}{\partial p}$ is often referred to
as the compressibility.

For this approximation, Equation \eqref{eq:stokes-2-compressible} replaces
Equation \eqref{eq:stokes-2}. It has the advantage that it retains the symmetry of the
Stokes equations if we can treat the right hand side of
\eqref{eq:stokes-2-compressible} as known. We do so by evaluating $\rho$ and
$\mathbf u$ using the solution from the last time step (or values extrapolated
from previous time steps), or using a nonlinear solver scheme.

\note{This is the default approximation \aspect{} uses to model compressible convection,
see Section~\ref{sec:combined_formulations}.}


\subsection{Choosing a formulation in \aspect{}}

After discussing different reasonable approximations for modeling compressible or 
incompressible mantle convection, we will now describe the different steps one has 
to take to use one of these approximations in a computation.  
This includes 
\begin{enumerate}
\item Choosing an approximation for the mass conservation equation; 
\item Choosing an approximation for the density in the energy balance, and deciding which heating terms should be included; 
\item Formulating the buoyancy term in the material model to be used on the right-hand side of the momentum equation; 
\item Prescribing a suitable reference state for the temperature, pressure, and density; i.e. the adiabatic profile,
if necessary for the approximations chosen in the first three steps.
\end{enumerate}

All of these choices can be made in the input file by selecting the corresponding parameters (see Sections~\ref{parameters:Formulation} and \ref{parameters:Adiabatic_20conditions_20model}).
A description of how to run \aspect{} and the basic structure of the input file can be found in
Section~\ref{sec:running}.

\subsubsection{Mass conservation approximation}
\label{sec:mass-conservation-approximation}

First, we have to choose how to approximate the conservation of mass: $\nabla \cdot (\rho \mathbf u) = 0$ (see Equation~\eqref{eq:stokes-2}). 
We provide the following options, which can be selected in the parameter file in the subsection 
\texttt{Formulation/Mass conservation} (see also \ref{parameters:Formulation/Mass conservation}):

\begin{itemize}

\item
``incompressible'':
\[
 \nabla \cdot \textbf{u} = 0,
\]

\item
``isothermal compression'':
\[
 \nabla \cdot \textbf{u} = -\frac{1}{\rho} \frac{\partial \rho}{\partial p} \rho \textbf{g} \cdot \textbf{u},
\]
where $\frac{1}{\rho} \frac{\partial \rho}{\partial p} = \kappa$ is the compressibility, and 
is defined in the material model. This is the explicit compressible mass equation where 
the velocity $\textbf{u}$ on the right-hand side is an extrapolated velocity from the last timesteps.

\item
``hydrostatic compression'':
\[
 \nabla \cdot \textbf{u}
= - \left( \frac{1}{\rho} \frac{\partial \rho}{\partial p} \rho \textbf{g} + \frac{1}{\rho} \frac{\partial \rho}{\partial T} \nabla T \right) \cdot \textbf{u}
= - \left( \kappa \rho \textbf{g} - \alpha \nabla T \right) \cdot \textbf{u}
\]
where $\frac{1}{\rho} \frac{\partial \rho}{\partial p} = \kappa$ is the compressibility,
$- \frac{1}{\rho}\frac{\partial \rho}{\partial T} = \alpha$ is the thermal expansion coefficient, 
and both are defined in the material model. 

\item
``reference density profile'':
\[
 \nabla \cdot \textbf{u} = -\frac{1}{\bar{\rho}} \frac{\partial \bar{\rho}}{\partial z} \frac{\textbf{g}}{\|\textbf{g}\|} \cdot \textbf{u},
\]
where the reference profiles for the density $\bar{\rho}$ and the density gradient $\frac{\partial \bar{\rho}}{\partial z}$
provided by the adiabatic conditions model (\ref{sec:adiabatic})
are used. Note that the gravity is assumed to point downwards in depth direction.
This is the explicit mass equation where the velocity $\textbf{u}$ on the right-hand 
side is an extrapolated velocity from the last timesteps.

\item
``implicit reference density profile'':
\[
 \nabla \cdot \textbf{u} + \frac{1}{\bar{\rho}} \frac{\partial \bar{\rho}}{\partial z} \frac{\textbf{g}}{\|\textbf{g}\|} \cdot \textbf{u} = 0,
\]
which uses the same approximation for the density as ``reference density profile'', 
but implements this term on the left-hand side instead of the right-hand side of the mass
conservation equation. This effectively uses the current velocity $\textbf{u}$ instead
of an explicitly extrapolated velocity from the last timesteps.

 \item 
``ask material model'', which uses ``isothermal compression'' if the material model reports 
that it is compressible and ``incompressible'' otherwise.
\end{itemize}


\note{\textbf{The stress tensor approximation.}

Incompressibility in the mass conservation equation automatically simplifies the shear strain rate in the momentum and temperature equation from
\[
 \nabla \cdot \tau = 
 \nabla \cdot \left[
 2\eta \left(\varepsilon(\mathbf u)
                                  - \frac{1}{3}(\nabla \cdot \mathbf u)\mathbf 1\right)
                                  \right]
\]
to
\[
  \nabla \cdot \tau =  \nabla \cdot \left[
 2\eta \varepsilon(\mathbf u) \right], 
\]
as $\nabla \cdot \mathbf u = 0$.
}

\subsubsection{Temperature equation approximation}

The density occurs multiple times in the temperature equation. Depending on the selected approximation it is computed in one of two different ways. Which of these options is used can be chosen in the parameter file in the subsection 
\texttt{Formulation/Temperature equation} (see also \ref{parameters:Formulation/Temperature equation}):

\begin{itemize}
 \item 
``real density'': Use the full density $\rho(p,T)$ that equals the one also used in the buoyancy term of the force balance equation; this is also the value that is computed by the material models when asked for the density,

\item
``reference density profile'': Use the density as computed for the reference profile (which can be constant, an adiabatic profile, or an entirely different function, and is determined by the adiabatic conditions model).
\end{itemize}

\subsubsection{Approximation of the buoyancy term}
The buoyancy term (right-hand side of the momentum equation) always uses the 
density that is provided by the material model (see Section~\ref{sec:material-models}). 
Depending on the material model, this density could for example depend on temperature 
and pressure (such as in ALA), or on temperature and depth (as in TALA); and the model can also be set up in a way 
that it uses density deviations from a reference state instead of a full density 
(see Section~\ref{sec:pressure-static-dyn}).

\note{In the current version of \aspect{}, it is the responsibility of the user to select a 
material model that is consistent with the formulation they want to use in their model. 
In the future, we plan to make it more obvious which approximations are supported by a 
particular material model.}

\subsubsection{Reference state: The adiabatic profile}

The reference temperature profile $\bar{T}$, reference density profile $\bar{\rho}$
and the reference pressure $\bar{p}$ are computed in the adiabatic conditions model 
(provided by the class \texttt{AdiabaticConditions}, see Section~\ref{sec:adiabatic}). 
By default, these fields satisfy adiabatic conditions (if adiabatic heating is included 
in the model, see Section~\ref{parameters:Heating_20model/Adiabatic_20heating}):
\begin{align}
  \frac{\textrm{d} \bar{T}(z)}{\textrm{d}z} 
  &=
  \frac{\alpha \bar{T}(z) g_z}{C_p},
\\
  \frac{\textrm{d} \bar{p}(z)}{\textrm{d}z} 
  &=
  \bar\rho g_z,
\\
  \bar{\rho} &= \bar\rho (\bar{p}, \bar{T}, z) \qquad \text{(as defined by the material model)},
\end{align}
where strictly speaking $g_z$ is the magnitude of the vertical
component of the gravity vector field, but in practice we take the
magnitude of the entire gravity vector.
If there is no adiabatic heating in the model, $\bar{T}$ is constant 
by default and set to the adiabatic surface temperature.
The density gradient is always computed by a simple finite difference approximation 
of the depth derivative of $\bar{\rho}$.

However, users can also supply their own adiabatic conditions models or 
define an arbitrary profile using the ``function'' plugin, 
which allows the user to define arbitrary functions for 
$\bar{T}(z)$, $\bar{p}(z)$ and $\bar{\rho}(z)$, see Section~\ref{parameters:Adiabatic_20conditions_20model}.

\subsubsection{Combined formulations}
\label{sec:combined_formulations}
Not all combinations of the different approximations discussed above 
are physically reasonable, and to help users choose between these options, 
we provide a number of combined ``Formulations'' that are equivalent to the 
approximate equations discussed above (Section~\ref{sec:approximate-equations}).
They can be selected in the subsection \texttt{Formulation/Formulation} 
(see also \ref{parameters:Formulation/Formulation}):

\begin{itemize}
 \item 
``anelastic liquid approximation'': This formulation sets the mass conservation approximation to ``reference density profile'', 
the temperature equation approximation to ``reference density profile'' and checks that both
adiabatic and shear heating are included in the list of heating plugins used in the model, using the 
simplified version of the adiabatic heating term 
(see Section~\ref{parameters:Heating_20model/Adiabatic_20heating}).
The default setting for the adiabatic conditions is an adiabatic temperature profile, and hydrostatic
pressure and density profiles. This option should be chosen together with a material model 
that defines a density that depends on temperature and pressure (and potentially depth), 
which would be equivalent to the anelastic liquid approximation (Section~\ref{sec:ala}), 
or with a material model that defines a density that depends on temperature and depth 
(and not on the pressure), which would be equivalent to the truncated anelastic liquid approximation 
(Section~\ref{sec:tala}).

\item
``Boussinesq approximation'': This formulation sets the mass conservation approximation to ``incompressible'', 
the temperature equation approximation to ``reference density profile'' and checks that neither
adiabatic nor shear heating are included in the list of heating plugins used in the model. 
The default setting for the adiabatic conditions is a constant temperature, and hydrostatic
pressure and density profiles. This option should be chosen together with a material model 
that defines a density that only depends on temperature and depth (and not on the pressure).
This is equivalent to the Boussinesq approximation (Section~\ref{sec:Boussinesq}).

\item
``isothermal compression'': This formulation sets the mass conservation approximation to 
``isothermal compression'', the temperature equation approximation to ``real density''
and checks that both
adiabatic and shear heating are included in the list of heating plugins used in the model.
The default setting for the adiabatic conditions is an adiabatic temperature profile, and hydrostatic
pressure and density profiles. The density can depend on any of the solution variables. 
This is equivalent to the isothermal compression approximation (Section~\ref{sec:ica}).

\item
``custom'': By default, this formulation sets the mass conservation approximation 
to ``ask material model'' and the temperature equation approximation to ``real density''.
The adiabatic conditions model uses an adiabatic temperature profile if adiabatic heating
is included in the model, and a constant temperature if adiabatic heating is not included.
Pressure and density profiles are hydrostatic. The density can depend on any of the solution variables.
However, this option can also be used to arbitrarily combine the different approximations 
described in this section. Users should be careful when using this option, as some combinations
may lead to unphysical model behaviour. 
\end{itemize}

An example cookbook that shows a comparison between different approximations is discussed in Section~\ref{sec:cookbooks-burnman}.

\subsection{Free surface calculations}
\label{sec:freesurface}

In reality the boundary conditions of a convecting Earth are not no-slip or 
free slip (i.e., no normal velocity).  Instead, we expect that a free surface
is a more realistic approximation, since air and water should not prevent the
flow of rock upward or downward.  This means that we require zero stress on the 
boundary, or $\sigma \cdot \textbf{n} = 0$, where $\sigma = 2 \eta \varepsilon (\textbf{u})$. 
In general there will be flow across the boundary with this boundary condition.  
To conserve mass we must then advect the boundary of the domain in the direction 
of fluid flow.  Thus, using a free surface necessitates that the mesh be dynamically deformable.  

\subsubsection{Arbitrary Lagrangian-Eulerian implementation}

The question of how to handle the motion of the mesh with a free surface is
challenging.  Eulerian meshes are well behaved, but they do not move with the 
fluid motions, which makes them difficult for use with free surfaces. 
Lagrangian meshes do move with the fluid, but they quickly become so 
distorted that remeshing is required. \aspect{} implements an Arbitrary 
Lagrangian-Eulerian (ALE) framework for handling motion of the mesh.  The ALE 
approach tries to retain the benefits of both the Lagrangian and the Eulerian
approaches by allowing the mesh motion $\textbf{u}_m$ to be largely independent of 
the fluid. The mass conservation condition requires that 
$\textbf{u}_m \cdot \textbf{n} = \textbf{u} \cdot \textbf{n}$ on the free 
surface, but otherwise the mesh motion is unconstrained, and should be chosen
to keep the mesh as well behaved as possible.

\aspect{} uses a Laplacian scheme for calculating the mesh velocity.  The mesh
velocity is calculated by solving

\begin{align}
-\Delta \textbf{u}_m &= 0 & \qquad & \textrm{in } \Omega, \\ 
\textbf{u}_m &= \left( \textbf{u} \cdot \textbf{n} \right) \textbf{n} & \qquad & \textrm{on } \partial \Omega_{\textrm{free surface}}, \\
\textbf{u}_m \cdot \textbf{n} &= 0 & \qquad & \textrm{on } \partial \Omega_{\textrm{free slip}}, \\
\textbf{u}_m &= 0 & \qquad & \textrm{on } \partial \Omega_{\textrm{Dirichlet}}.
\end{align}
After this mesh velocity is calculated, the mesh vertices are time-stepped explicitly.
This scheme has the effect of choosing a minimally distorting perturbation to the mesh.
Because the mesh velocity is no longer zero in the ALE approach, we must then correct
the Eulerian advection terms in the advection system with the mesh velocity (see, e.g.
\cite{DHPR2004}).  For instance, the temperature equation \eqref{eq:temperature-Boussinesq-linear}
becomes

\begin{equation*}
  \rho C_p \left(\frac{\partial T}{\partial t} + \left(\mathbf u - \mathbf u_m \right) \cdot\nabla T\right)
  - \nabla\cdot k\nabla T
  =
  \rho H
   \quad
   \textrm{in $\Omega$}.
\end{equation*}

\subsubsection{Free surface stabilization}

Small disequilibria in the location of a free surface can cause instabilities in
the surface position and result in a ``sloshing'' instability.  This may be countered with a
quasi-implicit free surface integration scheme described in \cite{KMM2010}.
This scheme enters the governing equations as a small stabilizing surface
traction that prevents the free surface advection from overshooting its
true position at the next time step.  \aspect{} implements this stabilization,
the details of which may be found in \cite{KMM2010}.

An example of a simple model which uses a free surface may be found in Section \ref{sec:cookbooks-freesurface}.

\subsection{Calculations with melt transport}
\label{sec:melt_transport}

The original formulation of the equations in Section~\ref{sec:equations} describes the movement of solid mantle material. These computations also allow for taking into account how partially molten material changes the material properties and the energy balance through the release of latent heat. However, this will not consider melt extraction or any relative movement between melt and solid and there might be problems where the transport of melt is of interest. Thus, \aspect{} allows for solving additional equations describing the behavior of silicate melt percolating through and interacting with a viscously deforming host rock. This requires 
the advection of a compositional field representing the volume fraction of melt present at any given time (the porosity $\phi$), 
and also a change of the mechanical part of the system. The latter is implemented using the approach of \cite{KMK2013} and changes 
the Stokes system to

\begin{align}
  \label{eq:stokes-1-melt}
  -\nabla \cdot \left[2\eta \left(\varepsilon(\mathbf{u}_s)
                                  - \frac{1}{3}(\nabla \cdot \mathbf{u}_s)\mathbf 1\right)
                \right] + \nabla p_f + \nabla p_c  &=
  \rho \mathbf g
  & \qquad
  & \textrm{in $\Omega$},
  \\
  \label{eq:stokes-2-melt}
  \nabla \cdot \mathbf{u}_s - \nabla \cdot K_D \nabla p_f 
  - K_D \nabla p_f \cdot \frac{\nabla \rho_f}{\rho_f}
  &= 
  - \nabla \cdot K_D \rho_f \mathbf g
  \notag
  \\
  &\quad
  + \Gamma \left( \frac{1}{\rho_f} - \frac{1}{\rho_s} \right)
  \\
  &\quad
  - \frac{\phi }{\rho_f} \mathbf{u}_s \cdot \nabla\rho_f 
  - \frac{1 - \phi }{\rho_s} \mathbf{u}_s \cdot \nabla\rho_s
  \notag
  \\
  &\quad
  - K_D \mathbf g \cdot \nabla \rho_f 
  & \qquad
  & \textrm{in $\Omega$},
  \notag
  \\
  \label{eq:stokes-3-melt}
  \nabla \cdot \mathbf{u}_s + \frac{p_c}{\xi} 
  &=
  0.
\end{align}

We use the indices $s$ to indicate properties of the solid and $f$ for the properties of the fluid. 
The equations are solved for the solid velocity $\mathbf{u}_s$, the fluid pressure $p_f$, and an additional 
variable, the compaction pressure $p_c$, which is related to the fluid and solid pressure through the relation 
$p_c = (1-\phi) (p_s-p_f)$. $K_D$ is the Darcy coefficient, which is defined as the quotient of the permeability 
and the fluid viscosity and $\Gamma$ is the melting rate. $\eta$ and $\xi$ are the shear and compaction viscosities 
and can depend on the porosity, temperature, pressure, strain rate and composition. However, there are various 
laws for these quantities and so they are implemented in the material model. Common formulations for the dependence 
on porosity are $\eta = (1-\phi) \eta_0 e^{-\alpha_\phi \phi}$ with $\alpha_\phi \approx 25...30$ and 
$\xi = \eta_0 \phi^{-n}$ with $n \approx 1$.

To avoid the density gradients in Equation~\eqref{eq:stokes-2-melt}, which would have to be specified individually 
for each material model by the user, we can use the same method as for the mass conservation (described in Section~\ref{sec:Boussinesq}) and assume the change in solid density is dominated by the change in static pressure, 
which can be written as
$\nabla p_s \approx \nabla p_{\text{static}} \approx \rho_s \textbf{g}$.
This finally allows us to write
\begin{equation*}
\frac{1}{\rho_s} \nabla \rho_s
\approx \frac{1}{\rho_s} \frac{\partial \rho_s}{\partial p_s} \nabla p_s
\approx \frac{1}{\rho_s} \frac{\partial \rho_s}{\partial p_s} \nabla p_s
\approx \frac{1}{\rho_s} \frac{\partial \rho_s}{\partial p_s} \rho_s \textbf{g}
\approx \kappa_s \rho_s \textbf{g}. 
\end{equation*}
For the fluid pressure, choosing a good approximation depends on the model parameters and setup (see \cite{dannberg_melt}). 
Hence, we make $\nabla \rho_{f}$ a model input parameter, which can be adapted based on the forces that are expected 
to be dominant in the model. 
We can then replace the second equation by
\begin{align*}
\nabla \cdot \mathbf{u}_s - \nabla \cdot K_D \nabla p_f 
  - K_D \nabla p_f \cdot \frac{\nabla \rho_f}{\rho_f}
  &= 
  - \nabla \cdot (K_D\rho_f \mathbf g)
  \\
  &\quad
  + \Gamma \left( \frac{1}{\rho_f} - \frac{1}{\rho_s} \right)
  \notag
  \\
  &\quad
  - \frac{\phi }{\rho_f} \mathbf{u}_s \cdot \nabla\rho_f
  - (\mathbf{u}_s \cdot \mathbf g ) (1 - \phi) \kappa_s \rho_s
  \notag
  \\
  &\quad
  - K_D \mathbf g \cdot \nabla \rho_f .
  \notag
\end{align*}
%
The melt velocity is computed as
\[
 \mathbf{u}_f =  \mathbf{u}_s - \frac{K_D}{\phi} (\nabla p_f - \rho_f g),
\]
but is only used for postprocessing purposes and for computing the time step length.  

\note{Here, we do not use the visco-elasto-plastic rheology of the \cite{KMK2013} formulation. 
Hence, we do not consider the elastic deformation terms that would appear on the right hand side of Equation 
\eqref{eq:stokes-1-melt} and Equation~\eqref{eq:stokes-3-melt} and that include the elastic and compaction stress 
evolution parameters $\xi_\tau$ and $\xi_p$. Moreover, our viscosity parameters $\eta$ and $\xi$ only cover viscous 
deformation instead of combining visco-elasticity and plastic failure. This would require a modification of the rheologic 
law using effective shear and compaction viscosities $\eta_{\text{eff}}$ and $\xi_{\text{eff}}$ combining a failure criterion 
and shear and compaction visco-elasticities.}

Moreover, melt transport requires an advection equation for the porosity field $\phi$:
\begin{align}
  \label{eq:porosity}
  \rho_s \frac{\partial (1 - \phi)}{\partial t} + \nabla \cdot \left[ \rho_s (1 - \phi) \mathbf{u}_s \right]
  &=
  - \Gamma
  & \quad
  & \textrm{in $\Omega$},
  i=1\ldots C
\end{align}

In order to solve this equation in the same way as the other advection equations, we replace the second term of the equation by: 

\begin{equation*}
\nabla \cdot \left[ \rho_s (1 - \phi) \mathbf{u}_s \right]
= \left( 1-\phi \right) \left( \rho_s \nabla \cdot \mathbf{u}_s 
+ \nabla \rho_s \cdot \mathbf{u}_s \right)
- \nabla \phi \cdot \rho_s \mathbf{u}_s 
\end{equation*}
Then we use the same method as described above and assume again that the change in density is dominated by the change in static pressure
\begin{equation*}
\frac{1}{\rho_s} \nabla \rho_s \cdot \mathbf{u}_s 
\approx \kappa_s \rho_s \textbf{g} \cdot \mathbf{u}_s 
\end{equation*}
so we get
\begin{equation*}
\frac{\partial \phi}{\partial t} + \mathbf{u}_s \cdot \nabla \phi
= \frac{\Gamma}{\rho_s}
+ (1 - \phi) (\nabla \cdot \mathbf{u}_s + \kappa_s \rho_s \textbf{g} \cdot \mathbf{u}_s ).
\end{equation*}

More details on the implementation can be found in \cite{dannberg_melt}. A benchmark case demonstrating the propagation of solitary waves can be found in Section~\ref{sec:benchmark-solitary_wave}.

\subsection{Nullspace removal}

The Stokes equation (\ref{eq:stokes-1}) only involves symmetric gradients of the velocity, and as such 
the velocity is determined only up to rigid-body motions (that is to say, translations and rotations).
For many simulations the boundary conditions will fully specify the velocity solution, but for some 
combinations of geometries and boundary conditions the solution will still be underdetermined.
In the language of linear algebra, the Stokes system may have a nullspace.

Usually the user will be able to determine beforehand whether their problem has a nullspace.  For instance, 
a model in a spherical shell geometry with free-slip boundary conditions at the top and bottom will 
have a rigid-body rotation in its nullspace (but not translations, as the boundary conditions do not 
allow flow through them).  That is to say, the solver may be able to come up with a solution to 
the Stokes operator, but that solution plus an arbitrary rotation is also an equally valid solution.

Another example is a model in a Cartesian box with periodic boundary conditions in the $x$-direction, 
and free slip boundaries on the top and bottom. This setup has arbitrary translations along the $x$-axis 
in its nullspace, so any solution plus an arbitrary $x$-translation is also a solution.

A solution with some small power in these nullspace modes should not affect the physics of the simulation. 
However, the timestepping of the model is based on evaluating the maximum velocities in the solution, 
and having unnecessary motions can severely shorten the time steps that \aspect{} takes. 
Furthermore, rigid body motions can make postprocessing calculations and visualization more 
difficult to interpret.  

\aspect{} allows the user to specify if their model has a nullspace. If so, any power in the nullspace 
is calculated and removed from the solution after every timestep.
There are two varieties of nullspace removal implemented: removing net linear/angular momentum, and 
removing net translations/rotations. 

For removing linear momentum we search for a constant velocity vector $\bf c$ such that 
\begin{equation*}
\int_\Omega \rho ({\bf u - c}) = 0
\end{equation*}

This may be solved by realizing that $\int_\Omega \rho {\bf u} = {\bf p}$, the linear momentum, and 
$\int_\Omega \rho = M$, the total mass of the model.  Then we find 
\begin{equation*}
{\bf c} = {\bf p}/M
\end{equation*}
which is subtracted off of the velocity solution.
 
Removing the angular momentum is similar, though a bit more complicated. 
We search for a rotation vector $\mathbf \omega$ such that 
\begin{equation*}
\int_\Omega \rho ( {\bf x \times (u - {\mathbf \omega} \times x) } ) = 0
\end{equation*}

Recognizing that $\int_\Omega \rho {\bf x \times u} = {\bf H}$, the angular momentum, 
and $\int_\Omega \rho {\bf x \times {\mathbf \omega} \times x} = {\bf I \cdot {\mathbf \omega} }$, 
the moment of inertia dotted into the sought-after vector, we can solve for ${\mathbf \omega}$: 
\begin{equation*}
{\mathbf \omega} = {\bf I^{-1} \cdot H}
\end{equation*}
A rotation about the rotation vector $\omega$ is then subtracted from the velocity solution.

Removing the net translations/rotations are identical to their momentum counterparts, but for those the 
density is dropped from the formulae. For most applications the density should not vary so wildly 
that there will be an appreciable difference between the two varieties, 
though removing linear/angular momentum is more physically motivated.

The user can flag the nullspace for removal by setting the \texttt{Remove nullspace} option,
as described in Section~\ref{parameters:Nullspace_20removal}.
Figure~\ref{fig:rigid_rotation} shows the result of removing angular momentum from a convection 
model in a 2D annulus with free-slip velocity boundary conditions. 

\begin{figure}[tbp]
  \centering
  \includegraphics[width=0.8\textwidth]{rigid_rotation.png}
  \caption{\it Example of nullspace removal. 
On the left the nullspace (a rigid rotation) is removed, and the velocity vectors accurately 
show the mantle flow. On the right there is a significant clockwise rotation to the velocity 
solution which is making the more interesting flow features difficult to see. }
  \label{fig:rigid_rotation}
\end{figure}


\subsection{Particles}
\label{sec:particles}

\aspect{} can, optionally, also deal with particles (sometimes called
``tracers''). Particles can be thought of as point-like objects that are simply
advected along with the flow. In other words, if $\mathbf u(\mathbf x,t)$ is the
flow field that results from solving equations
\eqref{eq:stokes-1}--\eqref{eq:stokes-2}, then the $k$th particle's position
satisfies the equations
\begin{align}
  \frac{\partial}{\partial t} \mathbf x_k(t)
  = \mathbf u(\mathbf x_k(t),t).
\end{align}
The initial positions of all particles also need to be given and are usually
either chosen randomly, based on a fixed pattern, or are read from a file.

Particles are typically used to track visually where material that starts
somewhere ends up after some time of a simulation. It can also be used to track
the \textit{history} of the volume of the fluid that surrounds a particle, for
example by tracking how much strain has accumulated, or what the minimal or maximal
temperature may have been in the medium along the trajectory of a particle. To
this end, particles can carry \textit{properties}. These are scalar-
or vector-valued quantities that are attached to each particle, that are
initialized at the beginning of a simulation, and that are then updated at each time step. In other words, if we
denote by $\mathbf p_{k,m}(t)$ the value of the $m$th property attached to
the $k$th particle, then $\mathbf p_{k,m}(t)$ will satisfy a differential
equation of the form
\begin{align*}
  \frac{\partial}{\partial t} \mathbf p_{k,m}(t)
  = \mathbf g_m\left(\mathbf p_{k,m}, 
  p(\mathbf x_k(t),t)), T(\mathbf x_k(t),t)), 
  \varepsilon(\mathbf u(\mathbf x_k(t),t)),
  \mathfrak c(\mathbf x_k(t),t)\right).
\end{align*}
The exact form of $\mathbf g_m$ of course depends on what exactly a particular
property represents. Like with compositional fields (see
Section~\ref{sec:compositional}), it is possible to describe the right hand side
$\mathbf g_m$ in ways that also allows for impulse (delta) functions in time.

How particles are used in practice is probably best explained using examples. To
this end, see in particular Section~\ref{sec:cookbooks-particles}. All
particle-related input parameters are listed in
Section~\ref{parameters:Postprocess/Particles}. The implementation of particles is
discussed in great detail in \cite{GHPB17}.

