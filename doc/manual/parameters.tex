\subsection{Global parameters}
\label{parameters:global}


\begin{itemize}
\item {\it Parameter name:} {\tt Additional shared libraries}


\index[prmindex]{Additional shared libraries}
\index[prmindexfull]{Additional shared libraries}
{\it Value:} 


{\it Default:} 


{\it Description:} A list of names of additional shared libraries that should be loaded upon starting up the program. The names of these files can contain absolute or relative paths (relative to the directory in which you call ASPECT). In fact, file names that are do not contain any directory information (i.e., only the name of a file such as <myplugin.so> will not be found if they are not located in one of the directories listed in the LD\_LIBRARY\_PATH environment variable. In order to load a library in the current directory, use <./myplugin.so> instead.

The typical use of this parameter is to so that you can implement additional plugins in your own directories, rather than in the ASPECT source directories. You can then simply compile these plugins into a shared library without having to re-compile all of ASPECT. See the section of the manual discussing writing extensions for more information on how to compile additional files into a shared library.


{\it Possible values:} [List list of <[FileName (Type: input)]> of length 0...4294967295 (inclusive)]
\item {\it Parameter name:} {\tt Adiabatic surface temperature}


\index[prmindex]{Adiabatic surface temperature}
\index[prmindexfull]{Adiabatic surface temperature}
{\it Value:} 0


{\it Default:} 0


{\it Description:} In order to make the problem in the first time step easier to solve, we need a reasonable guess for the temperature and pressure. To obtain it, we use an adiabatic pressure and temperature field. This parameter describes what the `adiabatic' temperature would be at the surface of the domain (i.e. at depth zero). Note that this value need not coincide with the boundary condition posed at this point. Rather, the boundary condition may differ significantly from the adiabatic value, and then typically induce a thermal boundary layer.
For more information, see the section in the manual that discusses the general mathematical model.


{\it Possible values:} [Double -1.79769e+308...1.79769e+308 (inclusive)]
\item {\it Parameter name:} {\tt CFL number}


\index[prmindex]{CFL number}
\index[prmindexfull]{CFL number}
{\it Value:} 1.0


{\it Default:} 1.0


{\it Description:} In computations, the time step $k$ is chosen according to $k = c \min_K \frac {h_K} {\|u\|_{\infty,K} p_T}$ where $h_K$ is the diameter of cell $K$, and the denominator is the maximal magnitude of the velocity on cell $K$ times the polynomial degree $p_T$ of the temperature discretization. The dimensionless constant $c$ is called the CFL number in this program. For time discretizations that have explicit components, $c$ must be less than a constant that depends on the details of the time discretization and that is no larger than one. On the other hand, for implicit discretizations such as the one chosen here, one can choose the time step as large as one wants (in particular, one can choose $c>1$) though a CFL number significantly larger than one will yield rather diffusive solutions. Units: None.


{\it Possible values:} [Double 0...1.79769e+308 (inclusive)]
\item {\it Parameter name:} {\tt Composition solver tolerance}


\index[prmindex]{Composition solver tolerance}
\index[prmindexfull]{Composition solver tolerance}
{\it Value:} 1e-12


{\it Default:} 1e-12


{\it Description:} The relative tolerance up to which the linear system for the composition system gets solved. See 'linear solver tolerance' for more details.


{\it Possible values:} [Double 0...1 (inclusive)]
\item {\it Parameter name:} {\tt Dimension}


\index[prmindex]{Dimension}
\index[prmindexfull]{Dimension}
{\it Value:} 2


{\it Default:} 2


{\it Description:} The number of space dimensions you want to run this program in. ASPECT can run in 2 and 3 space dimensions.


{\it Possible values:} [Integer range 2...4 (inclusive)]
\item {\it Parameter name:} {\tt End time}


\index[prmindex]{End time}
\index[prmindexfull]{End time}
{\it Value:} 1e300


{\it Default:} 1e300


{\it Description:} The end time of the simulation. Units: years if the 'Use years in output instead of seconds' parameter is set; seconds otherwise.


{\it Possible values:} [Double -1.79769e+308...1.79769e+308 (inclusive)]
\item {\it Parameter name:} {\tt Linear solver tolerance}


\index[prmindex]{Linear solver tolerance}
\index[prmindexfull]{Linear solver tolerance}
{\it Value:} 1e-7


{\it Default:} 1e-7


{\it Description:} A relative tolerance up to which the linear Stokes systems in each time or nonlinear step should be solved. The absolute tolerance will then be the norm of the right hand side of the equation times this tolerance. A given tolerance value of 1 would mean that a zero solution vector is an acceptable solution since in that case the norm of the residual of the linear system equals the norm of the right hand side. A given tolerance of 0 would mean that the linear system has to be solved exactly, since this is the only way to obtain a zero residual.

In practice, you should choose the value of this parameter to be so that if you make it smaller the results of your simulation do not change any more (qualitatively) whereas if you make it larger, they do. For most cases, the default value should be sufficient. However, for cases where the static pressure is much larger than the dynamic one, it may be necessary to choose a smaller value.


{\it Possible values:} [Double 0...1 (inclusive)]
\item {\it Parameter name:} {\tt Max nonlinear iterations}


\index[prmindex]{Max nonlinear iterations}
\index[prmindexfull]{Max nonlinear iterations}
{\it Value:} 10


{\it Default:} 10


{\it Description:} The maximal number of nonlinear iterations to be performed.


{\it Possible values:} [Integer range 0...2147483647 (inclusive)]
\item {\it Parameter name:} {\tt Nonlinear solver scheme}


\index[prmindex]{Nonlinear solver scheme}
\index[prmindexfull]{Nonlinear solver scheme}
{\it Value:} IMPES


{\it Default:} IMPES


{\it Description:} The kind of scheme used to resolve the nonlinearity in the system. 'IMPES' is the classical IMplicit Pressure Explicit Saturation scheme in which ones solves the temperatures and Stokes equations exactly once per time step, one after the other. The 'iterated IMPES' scheme iterates this decoupled approach by alternating the solution of the temperature and Stokes systems. The 'iterated Stokes' scheme solves the temperature equation once at the beginning of each time step and then iterates out the solution of the Stokes equation. The 'Stokes only' scheme only solves the Stokes system and ignores compositions and the temperature equation (careful, the material model must not depend on the temperature; mostly useful for Stokes benchmarks).


{\it Possible values:} [Selection IMPES|iterated IMPES|iterated Stokes|Stokes only ]
\item {\it Parameter name:} {\tt Number of cheap Stokes solver steps}


\index[prmindex]{Number of cheap Stokes solver steps}
\index[prmindexfull]{Number of cheap Stokes solver steps}
{\it Value:} 30


{\it Default:} 30


{\it Description:} As explained in the ASPECT paper (Kronbichler, Heister, and Bangerth, GJI 2012) we first try to solve the Stokes system in every time step using a GMRES iteration with a poor but cheap preconditioner. By default, we try whether we can converge the GMRES solver in 30 such iterations before deciding that we need a better preconditioner. This is sufficient for simple problems with constant viscosity and we never need the second phase with the more expensive preconditioner. On the other hand, for more complex problems, and in particular for problems with strongly varying viscosity, the 30 cheap iterations don't actually do very much good and one might skip this part right away. In that case, this parameter can be set to zero, i.e., we immediately start with the better but more expensive preconditioner.


{\it Possible values:} [Integer range 0...2147483647 (inclusive)]
\item {\it Parameter name:} {\tt Output directory}


\index[prmindex]{Output directory}
\index[prmindexfull]{Output directory}
{\it Value:} output


{\it Default:} output


{\it Description:} The name of the directory into which all output files should be placed. This may be an absolute or a relative path.


{\it Possible values:} [DirectoryName]
\item {\it Parameter name:} {\tt Pressure normalization}


\index[prmindex]{Pressure normalization}
\index[prmindexfull]{Pressure normalization}
{\it Value:} surface


{\it Default:} surface


{\it Description:} If and how to normalize the pressure after the solution step. This is necessary because depending on boundary conditions, in many cases the pressure is only determined by the model up to a constant. On the other hand, we often would like to have a well-determined pressure, for example for table lookups of material properties in models or for comparing solutions. If the given value is `surface', then normalization at the end of each time steps adds a constant value to the pressure in such a way that the average pressure at the surface of the domain is zero; the surface of the domain is determined by asking the geometry model whether a particular face of the geometry has a zero or small `depth'. If the value of this parameter is `volume' then the pressure is normalized so that the domain average is zero. If `no' is given, the no pressure normalization is performed.


{\it Possible values:} [Selection surface|volume|no ]
\item {\it Parameter name:} {\tt Resume computation}


\index[prmindex]{Resume computation}
\index[prmindexfull]{Resume computation}
{\it Value:} false


{\it Default:} false


{\it Description:} A flag indicating whether the computation should be resumed from a previously saved state (if true) or start from scratch (if false).


{\it Possible values:} [Bool]
\item {\it Parameter name:} {\tt Start time}


\index[prmindex]{Start time}
\index[prmindexfull]{Start time}
{\it Value:} 0


{\it Default:} 0


{\it Description:} The start time of the simulation. Units: years if the 'Use years in output instead of seconds' parameter is set; seconds otherwise.


{\it Possible values:} [Double -1.79769e+308...1.79769e+308 (inclusive)]
\item {\it Parameter name:} {\tt Surface pressure}


\index[prmindex]{Surface pressure}
\index[prmindexfull]{Surface pressure}
{\it Value:} 0


{\it Default:} 0


{\it Description:} The mathematical equations that describe thermal convection only determine the pressure up to an arbitrary constant. On the other hand, for comparison and for looking up material parameters it is important that the pressure be normalized somehow. We do this by enforcing a particular average pressure value at the surface of the domain, where the geometry model determines where the surface is. This parameter describes what this average surface pressure value is supposed to be. By default, it is set to zero, but one may want to choose a different value for example for simulating only the volume of the mantle below the lithosphere, in which case the surface pressure should be the lithostatic pressure at the bottom of the lithosphere.
For more information, see the section in the manual that discusses the general mathematical model.


{\it Possible values:} [Double -1.79769e+308...1.79769e+308 (inclusive)]
\item {\it Parameter name:} {\tt Temperature solver tolerance}


\index[prmindex]{Temperature solver tolerance}
\index[prmindexfull]{Temperature solver tolerance}
{\it Value:} 1e-12


{\it Default:} 1e-12


{\it Description:} The relative tolerance up to which the linear system for the temperature system gets solved. See 'linear solver tolerance' for more details.


{\it Possible values:} [Double 0...1 (inclusive)]
\item {\it Parameter name:} {\tt Timing output frequency}


\index[prmindex]{Timing output frequency}
\index[prmindexfull]{Timing output frequency}
{\it Value:} 100


{\it Default:} 100


{\it Description:} How frequently in timesteps to output timing information. This is generally adjusted only for debugging and timing purposes.


{\it Possible values:} [Integer range 0...2147483647 (inclusive)]
\item {\it Parameter name:} {\tt Use conduction timestep}


\index[prmindex]{Use conduction timestep}
\index[prmindexfull]{Use conduction timestep}
{\it Value:} false


{\it Default:} false


{\it Description:} Mantle convection simulations are often focused on convection dominated systems. However, these codes can also be used to investigate systems where heat conduction plays a dominant role. This parameter indicates whether the simulator should also use heat conduction in determining the length of each time step.


{\it Possible values:} [Bool]
\item {\it Parameter name:} {\tt Use years in output instead of seconds}


\index[prmindex]{Use years in output instead of seconds}
\index[prmindexfull]{Use years in output instead of seconds}
{\it Value:} true


{\it Default:} true


{\it Description:} When computing results for mantle convection simulations, it is often difficult to judge the order of magnitude of results when they are stated in MKS units involving seconds. Rather, some kinds of results such as velocities are often stated in terms of meters per year (or, sometimes, centimeters per year). On the other hand, for non-dimensional computations, one wants results in their natural unit system as used inside the code. If this flag is set to 'true' conversion to years happens; if it is 'false', no such conversion happens.


{\it Possible values:} [Bool]
\end{itemize}



\subsection{Parameters in section \tt Boundary temperature model}
\label{parameters:Boundary_20temperature_20model}

\begin{itemize}
\item {\it Parameter name:} {\tt Model name}


\index[prmindex]{Model name}
\index[prmindexfull]{Boundary temperature model!Model name}
{\it Value:} box


{\it Default:} 


{\it Description:} Select one of the following models:

`box': A model in which the temperature is chosen constant on all the sides of a box.

`spherical constant': A model in which the temperature is chosen constant on the inner and outer boundaries of a spherical shell. Parameters are read from subsection 'Sherical constant'.

`Tan Gurnis': A model for the Tan/Gurnis benchmark.


{\it Possible values:} [Selection box|spherical constant|Tan Gurnis ]
\end{itemize}



\subsection{Parameters in section \tt Boundary temperature model/Box}
\label{parameters:Boundary_20temperature_20model/Box}

\begin{itemize}
\item {\it Parameter name:} {\tt Bottom temperature}


\index[prmindex]{Bottom temperature}
\index[prmindexfull]{Boundary temperature model!Box!Bottom temperature}
{\it Value:} 0


{\it Default:} 0


{\it Description:} Temperature at the bottom boundary (at minimal z-value). Units: K.


{\it Possible values:} [Double -1.79769e+308...1.79769e+308 (inclusive)]
\item {\it Parameter name:} {\tt Left temperature}


\index[prmindex]{Left temperature}
\index[prmindexfull]{Boundary temperature model!Box!Left temperature}
{\it Value:} 1


{\it Default:} 1


{\it Description:} Temperature at the left boundary (at minimal x-value). Units: K.


{\it Possible values:} [Double -1.79769e+308...1.79769e+308 (inclusive)]
\item {\it Parameter name:} {\tt Right temperature}


\index[prmindex]{Right temperature}
\index[prmindexfull]{Boundary temperature model!Box!Right temperature}
{\it Value:} 0


{\it Default:} 0


{\it Description:} Temperature at the right boundary (at maximal x-value). Units: K.


{\it Possible values:} [Double -1.79769e+308...1.79769e+308 (inclusive)]
\item {\it Parameter name:} {\tt Top temperature}


\index[prmindex]{Top temperature}
\index[prmindexfull]{Boundary temperature model!Box!Top temperature}
{\it Value:} 0


{\it Default:} 0


{\it Description:} Temperature at the top boundary (at maximal x-value). Units: K.


{\it Possible values:} [Double -1.79769e+308...1.79769e+308 (inclusive)]
\end{itemize}

\subsection{Parameters in section \tt Boundary temperature model/Spherical constant}
\label{parameters:Boundary_20temperature_20model/Spherical_20constant}

\begin{itemize}
\item {\it Parameter name:} {\tt Inner temperature}


\index[prmindex]{Inner temperature}
\index[prmindexfull]{Boundary temperature model!Spherical constant!Inner temperature}
{\it Value:} 6000


{\it Default:} 6000


{\it Description:} Temperature at the inner boundary (core mantle boundary). Units: K.


{\it Possible values:} [Double -1.79769e+308...1.79769e+308 (inclusive)]
\item {\it Parameter name:} {\tt Outer temperature}


\index[prmindex]{Outer temperature}
\index[prmindexfull]{Boundary temperature model!Spherical constant!Outer temperature}
{\it Value:} 0


{\it Default:} 0


{\it Description:} Temperature at the outer boundary (lithosphere water/air). Units: K.


{\it Possible values:} [Double -1.79769e+308...1.79769e+308 (inclusive)]
\end{itemize}

\subsection{Parameters in section \tt Boundary velocity model}
\label{parameters:Boundary_20velocity_20model}


\subsection{Parameters in section \tt Boundary velocity model/Function}
\label{parameters:Boundary_20velocity_20model/Function}

\begin{itemize}
\item {\it Parameter name:} {\tt Function constants}


\index[prmindex]{Function constants}
\index[prmindexfull]{Boundary velocity model!Function!Function constants}
{\it Value:} 


{\it Default:} 


{\it Description:} Sometimes it is convenient to use symbolic constants in the expression that describes the function, rather than having to use its numeric value everywhere the constant appears. These values can be defined using this parameter, in the form `var1=value1, var2=value2, ...'.

A typical example would be to set this runtime parameter to `pi=3.1415926536' and then use `pi' in the expression of the actual formula. (That said, for convenience this class actually defines both `pi' and `Pi' by default, but you get the idea.)


{\it Possible values:} [Anything]
\item {\it Parameter name:} {\tt Function expression}


\index[prmindex]{Function expression}
\index[prmindexfull]{Boundary velocity model!Function!Function expression}
{\it Value:} 0; 0


{\it Default:} 0; 0


{\it Description:} The formula that denotes the function you want to evaluate for particular values of the independent variables. This expression may contain any of the usual operations such as addition or multiplication, as well as all of the common functions such as `sin' or `cos'. In addition, it may contain expressions like `if(x>0, 1, -1)' where the expression evaluates to the second argument if the first argument is true, and to the third argument otherwise. For a full overview of possible expressions accepted see the documentation of the fparser library.

If the function you are describing represents a vector-valued function with multiple components, then separate the expressions for individual components by a semicolon.


{\it Possible values:} [Anything]
\item {\it Parameter name:} {\tt Variable names}


\index[prmindex]{Variable names}
\index[prmindexfull]{Boundary velocity model!Function!Variable names}
{\it Value:} x,y,t


{\it Default:} x,y,t


{\it Description:} The name of the variables as they will be used in the function, separated by commas. By default, the names of variables at which the function will be evaluated is `x' (in 1d), `x,y' (in 2d) or `x,y,z' (in 3d) for spatial coordinates and `t' for time. You can then use these variable names in your function expression and they will be replaced by the values of these variables at which the function is currently evaluated. However, you can also choose a different set of names for the independent variables at which to evaluate your function expression. For example, if you work in spherical coordinates, you may wish to set this input parameter to `r,phi,theta,t' and then use these variable names in your function expression.


{\it Possible values:} [Anything]
\end{itemize}

\subsection{Parameters in section \tt Boundary velocity model/GPlates model}
\label{parameters:Boundary_20velocity_20model/GPlates_20model}

\begin{itemize}
\item {\it Parameter name:} {\tt Data directory}


\index[prmindex]{Data directory}
\index[prmindexfull]{Boundary velocity model!GPlates model!Data directory}
{\it Value:} data/velocity-boundary-conditions/gplates/


{\it Default:} data/velocity-boundary-conditions/gplates/


{\it Description:} The path to the model data.


{\it Possible values:} [DirectoryName]
\item {\it Parameter name:} {\tt Point one}


\index[prmindex]{Point one}
\index[prmindexfull]{Boundary velocity model!GPlates model!Point one}
{\it Value:} 1.570796,0.0


{\it Default:} 1.570796,0.0


{\it Description:} Point that determines the plane in which a 2D model lies in. Has to be in the format 'a,b' where a and b are theta (polar angle)  and phi in radians.


{\it Possible values:} [Anything]
\item {\it Parameter name:} {\tt Point two}


\index[prmindex]{Point two}
\index[prmindexfull]{Boundary velocity model!GPlates model!Point two}
{\it Value:} 1.570796,1.570796


{\it Default:} 1.570796,1.570796


{\it Description:} Point that determines the plane in which a 2D model lies in. Has to be in the format 'a,b' where a and b are theta (polar angle)  and phi in radians.


{\it Possible values:} [Anything]
\item {\it Parameter name:} {\tt Time step}


\index[prmindex]{Time step}
\index[prmindexfull]{Boundary velocity model!GPlates model!Time step}
{\it Value:} 3.1558e13


{\it Default:} 3.1558e13


{\it Description:} Time step between following velocity files. Default is one million years expressed in SI units.


{\it Possible values:} [Anything]
\item {\it Parameter name:} {\tt Velocity file name}


\index[prmindex]{Velocity file name}
\index[prmindexfull]{Boundary velocity model!GPlates model!Velocity file name}
{\it Value:} phi.%d


{\it Default:} phi.%d


{\it Description:} The file name of the material data. Provide file in format: (Velocity file name).%d.gpml where %d is any sprintf integer qualifier, specifying the format of the current file number.


{\it Possible values:} [Anything]
\item {\it Parameter name:} {\tt Velocity file start time}


\index[prmindex]{Velocity file start time}
\index[prmindexfull]{Boundary velocity model!GPlates model!Velocity file start time}
{\it Value:} 0.0


{\it Default:} 0.0


{\it Description:} Time at which the velocity file with number 0 shall be loaded.Previous to this time, a no-slip boundary condition is assumed.


{\it Possible values:} [Anything]
\end{itemize}

\subsection{Parameters in section \tt Checkpointing}
\label{parameters:Checkpointing}

\begin{itemize}
\item {\it Parameter name:} {\tt Steps between checkpoint}


\index[prmindex]{Steps between checkpoint}
\index[prmindexfull]{Checkpointing!Steps between checkpoint}
{\it Value:} 0


{\it Default:} 0


{\it Description:} The number of timesteps between performing checkpoints. If 0 and time between checkpoint is not specified, checkpointing will not be performed. Units: None.


{\it Possible values:} [Integer range 0...2147483647 (inclusive)]
\item {\it Parameter name:} {\tt Time between checkpoint}


\index[prmindex]{Time between checkpoint}
\index[prmindexfull]{Checkpointing!Time between checkpoint}
{\it Value:} 0


{\it Default:} 0


{\it Description:} The wall time between performing checkpoints. If 0, will use the checkpoint step frequency instead. Units: Seconds.


{\it Possible values:} [Integer range 0...2147483647 (inclusive)]
\end{itemize}

\subsection{Parameters in section \tt Compositional fields}
\label{parameters:Compositional_20fields}

\begin{itemize}
\item {\it Parameter name:} {\tt List of normalized fields}


\index[prmindex]{List of normalized fields}
\index[prmindexfull]{Compositional fields!List of normalized fields}
{\it Value:} 


{\it Default:} 


{\it Description:} A list of integers smaller than or equal to the number of compositional fields. All compositional fields in this list will be normalized before the first timestep. The normalization is implemented in the following way: First, the sum of the fields to be normalized is calculated at every point and the global maximum is determined. Second, the compositional fields to be normalized are divided by this maximum.


{\it Possible values:} [List list of <[Integer range 0...2147483647 (inclusive)]> of length 0...4294967295 (inclusive)]
\item {\it Parameter name:} {\tt Number of fields}


\index[prmindex]{Number of fields}
\index[prmindexfull]{Compositional fields!Number of fields}
{\it Value:} 0


{\it Default:} 0


{\it Description:} The number of fields that will be advected along with the flow field, excluding velocity, pressure and temperature.


{\it Possible values:} [Integer range 0...2147483647 (inclusive)]
\end{itemize}

\subsection{Parameters in section \tt Compositional initial conditions}
\label{parameters:Compositional_20initial_20conditions}

\begin{itemize}
\item {\it Parameter name:} {\tt Model name}


\index[prmindex]{Model name}
\index[prmindexfull]{Compositional initial conditions!Model name}
{\it Value:} function


{\it Default:} function


{\it Description:} Select one of the following models:

`function': Composition is given in terms of an explicit formula


{\it Possible values:} [Selection function ]
\end{itemize}



\subsection{Parameters in section \tt Compositional initial conditions/Function}
\label{parameters:Compositional_20initial_20conditions/Function}

\begin{itemize}
\item {\it Parameter name:} {\tt Function constants}


\index[prmindex]{Function constants}
\index[prmindexfull]{Compositional initial conditions!Function!Function constants}
{\it Value:} 


{\it Default:} 


{\it Description:} Sometimes it is convenient to use symbolic constants in the expression that describes the function, rather than having to use its numeric value everywhere the constant appears. These values can be defined using this parameter, in the form `var1=value1, var2=value2, ...'.

A typical example would be to set this runtime parameter to `pi=3.1415926536' and then use `pi' in the expression of the actual formula. (That said, for convenience this class actually defines both `pi' and `Pi' by default, but you get the idea.)


{\it Possible values:} [Anything]
\item {\it Parameter name:} {\tt Function expression}


\index[prmindex]{Function expression}
\index[prmindexfull]{Compositional initial conditions!Function!Function expression}
{\it Value:} 0


{\it Default:} 0


{\it Description:} The formula that denotes the function you want to evaluate for particular values of the independent variables. This expression may contain any of the usual operations such as addition or multiplication, as well as all of the common functions such as `sin' or `cos'. In addition, it may contain expressions like `if(x>0, 1, -1)' where the expression evaluates to the second argument if the first argument is true, and to the third argument otherwise. For a full overview of possible expressions accepted see the documentation of the fparser library.

If the function you are describing represents a vector-valued function with multiple components, then separate the expressions for individual components by a semicolon.


{\it Possible values:} [Anything]
\item {\it Parameter name:} {\tt Variable names}


\index[prmindex]{Variable names}
\index[prmindexfull]{Compositional initial conditions!Function!Variable names}
{\it Value:} x,y,t


{\it Default:} x,y,t


{\it Description:} The name of the variables as they will be used in the function, separated by commas. By default, the names of variables at which the function will be evaluated is `x' (in 1d), `x,y' (in 2d) or `x,y,z' (in 3d) for spatial coordinates and `t' for time. You can then use these variable names in your function expression and they will be replaced by the values of these variables at which the function is currently evaluated. However, you can also choose a different set of names for the independent variables at which to evaluate your function expression. For example, if you work in spherical coordinates, you may wish to set this input parameter to `r,phi,theta,t' and then use these variable names in your function expression.


{\it Possible values:} [Anything]
\end{itemize}

\subsection{Parameters in section \tt Discretization}
\label{parameters:Discretization}

\begin{itemize}
\item {\it Parameter name:} {\tt Composition polynomial degree}


\index[prmindex]{Composition polynomial degree}
\index[prmindexfull]{Discretization!Composition polynomial degree}
{\it Value:} 2


{\it Default:} 2


{\it Description:} The polynomial degree to use for the composition variable(s). Units: None.


{\it Possible values:} [Integer range 1...2147483647 (inclusive)]
\item {\it Parameter name:} {\tt Stokes velocity polynomial degree}


\index[prmindex]{Stokes velocity polynomial degree}
\index[prmindexfull]{Discretization!Stokes velocity polynomial degree}
{\it Value:} 2


{\it Default:} 2


{\it Description:} The polynomial degree to use for the velocity variables in the Stokes system. The polynomial degree for the pressure variable will then be one less in order to make the velocity/pressure pair conform with the usual LBB (Babuska-Brezzi) condition. In other words, we are using a Taylor-Hood element for the Stoeks equations and this parameter indicates the polynomial degree of it. Units: None.


{\it Possible values:} [Integer range 1...2147483647 (inclusive)]
\item {\it Parameter name:} {\tt Temperature polynomial degree}


\index[prmindex]{Temperature polynomial degree}
\index[prmindexfull]{Discretization!Temperature polynomial degree}
{\it Value:} 2


{\it Default:} 2


{\it Description:} The polynomial degree to use for the temperature variable. Units: None.


{\it Possible values:} [Integer range 1...2147483647 (inclusive)]
\item {\it Parameter name:} {\tt Use locally conservative discretization}


\index[prmindex]{Use locally conservative discretization}
\index[prmindexfull]{Discretization!Use locally conservative discretization}
{\it Value:} false


{\it Default:} false


{\it Description:} Whether to use a Stokes discretization that is locally conservative at the expense of a larger number of degrees of freedom (true), or to go with a cheaper discretization that does not locally conserve mass, although it is globally conservative (false).

When using a locally conservative discretization, the finite element space for the pressure is discontinuous between cells and is the polynomial space $P_ {-q}$ of polynomials of degree $q$ in each variable separately. Here, $q$ is one less than the value given in the parameter ``Stokes velocity polynomial degree''. As a consequence of choosing this element, it can be shown if the medium is considered incompressible that the computed discrete velocity field $\mathbf u_h$ satisfies the property $\int_ {\partial K} \mathbf u_h \cdot \mathbf n = 0$ for every cell $K$, i.e., for each cell inflow and outflow exactly balance each other as one would expect for an incompressible medium. In other words, the velocity field is locally conservative.

On the other hand, if this parameter is set to ``false'', then the finite element space is chosen as $Q_q$. This choice does not yield the local conservation property but has the advantage of requiring fewer degrees of freedom. Furthermore, the error is generally smaller with this choice.

For an in-depth discussion of these issues and a quantitative evaluation of the different choices, see \cite {KHB12} .


{\it Possible values:} [Bool]
\end{itemize}



\subsection{Parameters in section \tt Discretization/Stabilization parameters}
\label{parameters:Discretization/Stabilization_20parameters}

\begin{itemize}
\item {\it Parameter name:} {\tt alpha}


\index[prmindex]{alpha}
\index[prmindexfull]{Discretization!Stabilization parameters!alpha}
{\it Value:} 2


{\it Default:} 2


{\it Description:} The exponent $\alpha$ in the entropy viscosity stabilization. Valid options are 1 or 2. The recommended setting is 2. (This parameter does not correspond to any variable in the 2012 GJI paper by Kronbichler, Heister and Bangerth that describes ASPECT. Rather, the paper always uses 2 as the exponent in the definition of the entropy, following eq. (15).).Units: None.


{\it Possible values:} [Integer range 1...2 (inclusive)]
\item {\it Parameter name:} {\tt beta}


\index[prmindex]{beta}
\index[prmindexfull]{Discretization!Stabilization parameters!beta}
{\it Value:} 0.078


{\it Default:} 0.078


{\it Description:} The $\beta$ factor in the artificial viscosity stabilization. An appropriate value for 2d is 0.078 and 0.117 for 3d. (For historical reasons, the name used here is different from the one used in the 2012 GJI paper by Kronbichler, Heister and Bangerth that describes ASPECT. This parameter corresponds to the factor $\alpha_\text {max}$ in the formulas following equation (15) of the paper. After further experiments, we have also chosen to use a different value than described there: It can be chosen as stated there for uniformly refined meshes, but it needs to be chosen larger if the mesh has cells that are not squares or cubes.) Units: None.


{\it Possible values:} [Double 0...1.79769e+308 (inclusive)]
\item {\it Parameter name:} {\tt cR}


\index[prmindex]{cR}
\index[prmindexfull]{Discretization!Stabilization parameters!cR}
{\it Value:} 0.33


{\it Default:} 0.33


{\it Description:} The $c_R$ factor in the entropy viscosity stabilization. (For historical reasons, the name used here is different from the one used in the 2012 GJI paper by Kronbichler, Heister and Bangerth that describes ASPECT. This parameter corresponds to the factor $\alpha_E$ in the formulas following equation (15) of the paper. After further experiments, we have also chosen to use a different value than described there.) Units: None.


{\it Possible values:} [Double 0...1.79769e+308 (inclusive)]
\end{itemize}

\subsection{Parameters in section \tt Geometry model}
\label{parameters:Geometry_20model}

\begin{itemize}
\item {\it Parameter name:} {\tt Model name}


\index[prmindex]{Model name}
\index[prmindexfull]{Geometry model!Model name}
{\it Value:} box


{\it Default:} 


{\it Description:} Select one of the following models:

`box': A box geometry parallel to the coordinate directions. The extent of the box in each coordinate direction is set in the parameter file. The box geometry labels its 2*dim sides as follows: in 2d, boundary indicators 0 through 3 denote the left, right, bottom and top boundaries; in 3d, boundary indicators 0 through 5 indicate left, right, front, back, bottom and top boundaries. See also the documentation of the deal.II class ``GeometryInfo''.

`spherical shell': A geometry representing a spherical shell or a pice of it. Inner and outer radii are read from the parameter file in subsection 'Spherical shell'.


{\it Possible values:} [Selection box|spherical shell ]
\end{itemize}



\subsection{Parameters in section \tt Geometry model/Box}
\label{parameters:Geometry_20model/Box}

\begin{itemize}
\item {\it Parameter name:} {\tt X extent}


\index[prmindex]{X extent}
\index[prmindexfull]{Geometry model!Box!X extent}
{\it Value:} 1


{\it Default:} 1


{\it Description:} Extent of the box in x-direction. Units: m.


{\it Possible values:} [Double 0...1.79769e+308 (inclusive)]
\item {\it Parameter name:} {\tt Y extent}


\index[prmindex]{Y extent}
\index[prmindexfull]{Geometry model!Box!Y extent}
{\it Value:} 1


{\it Default:} 1


{\it Description:} Extent of the box in y-direction. Units: m.


{\it Possible values:} [Double 0...1.79769e+308 (inclusive)]
\item {\it Parameter name:} {\tt Z extent}


\index[prmindex]{Z extent}
\index[prmindexfull]{Geometry model!Box!Z extent}
{\it Value:} 1


{\it Default:} 1


{\it Description:} Extent of the box in z-direction. This value is ignored if the simulation is in 2d Units: m.


{\it Possible values:} [Double 0...1.79769e+308 (inclusive)]
\end{itemize}

\subsection{Parameters in section \tt Geometry model/Spherical shell}
\label{parameters:Geometry_20model/Spherical_20shell}

\begin{itemize}
\item {\it Parameter name:} {\tt Inner radius}


\index[prmindex]{Inner radius}
\index[prmindexfull]{Geometry model!Spherical shell!Inner radius}
{\it Value:} 3481000


{\it Default:} 3481000


{\it Description:} Inner radius of the spherical shell. Units: m.


{\it Possible values:} [Double 0...1.79769e+308 (inclusive)]
\item {\it Parameter name:} {\tt Opening angle}


\index[prmindex]{Opening angle}
\index[prmindexfull]{Geometry model!Spherical shell!Opening angle}
{\it Value:} 360


{\it Default:} 360


{\it Description:} Opening angle in degrees of the section of the shell that we want to build. Units: degrees.


{\it Possible values:} [Double 0...360 (inclusive)]
\item {\it Parameter name:} {\tt Outer radius}


\index[prmindex]{Outer radius}
\index[prmindexfull]{Geometry model!Spherical shell!Outer radius}
{\it Value:} 6336000


{\it Default:} 6336000


{\it Description:} Outer radius of the spherical shell. Units: m.


{\it Possible values:} [Double 0...1.79769e+308 (inclusive)]
\end{itemize}

\subsection{Parameters in section \tt Gravity model}
\label{parameters:Gravity_20model}

\begin{itemize}
\item {\it Parameter name:} {\tt Model name}


\index[prmindex]{Model name}
\index[prmindexfull]{Gravity model!Model name}
{\it Value:} vertical


{\it Default:} 


{\it Description:} Select one of the following models:

`radial constant': A gravity model in which the gravity direction is radially inward and at constant magnitude. The magnitude is read from the parameter file in subsection 'Radial constant'.

`radial earth-like': A gravity model in which the gravity direction is radially inward and with a magnitude that matches that of the earth at the core-mantle boundary as well as at the surface and in between is physically correct under the assumption of a constant density.

`vertical': A gravity model in which the gravity direction is vertically downward and at a constant magnitude by default equal to one.


{\it Possible values:} [Selection radial constant|radial earth-like|vertical ]
\end{itemize}



\subsection{Parameters in section \tt Gravity model/Radial constant}
\label{parameters:Gravity_20model/Radial_20constant}

\begin{itemize}
\item {\it Parameter name:} {\tt Magnitude}


\index[prmindex]{Magnitude}
\index[prmindexfull]{Gravity model!Radial constant!Magnitude}
{\it Value:} 30


{\it Default:} 30


{\it Description:} Magnitude of the gravity vector in $m/s^2$. The direction is always radially outward from the center of the earth.


{\it Possible values:} [Double 0...1.79769e+308 (inclusive)]
\end{itemize}

\subsection{Parameters in section \tt Gravity model/Vertical}
\label{parameters:Gravity_20model/Vertical}

\begin{itemize}
\item {\it Parameter name:} {\tt Magnitude}


\index[prmindex]{Magnitude}
\index[prmindexfull]{Gravity model!Vertical!Magnitude}
{\it Value:} 1


{\it Default:} 1


{\it Description:} Value of the gravity vector in $m/s^2$ directed along negative y (2D) or z (3D) axis.


{\it Possible values:} [Double 0...1.79769e+308 (inclusive)]
\end{itemize}

\subsection{Parameters in section \tt Initial conditions}
\label{parameters:Initial_20conditions}

\begin{itemize}
\item {\it Parameter name:} {\tt Model name}


\index[prmindex]{Model name}
\index[prmindexfull]{Initial conditions!Model name}
{\it Value:} perturbed box


{\it Default:} 


{\it Description:} Select one of the following models:

`adiabatic': Temperature is prescribed as an adiabatic profile with upper and lower thermal boundary layers, whose ages are given as input parameters.

`perturbed box': An initial temperature field in which the temperature is perturbed slightly from an otherwise constant value equal to one. The perturbation is chosen in such a way that the initial temperature is constant to one along the entire boundary.

`function': Temperature is given in terms of an explicit formula

`spherical hexagonal perturbation': An initial temperature field in which the temperature is perturbed following a six-fold pattern in angular direction from an otherwise spherically symmetric state.

`spherical gaussian perturbation': An initial temperature field in which the temperature is perturbed by a single Gaussian added to an otherwise spherically symmetric state. Additional parameters are read from the parameter file in subsection 'Spherical gaussian perturbation'.


{\it Possible values:} [Selection adiabatic|perturbed box|function|spherical hexagonal perturbation|spherical gaussian perturbation ]
\end{itemize}



\subsection{Parameters in section \tt Initial conditions/Adiabatic}
\label{parameters:Initial_20conditions/Adiabatic}

\begin{itemize}
\item {\it Parameter name:} {\tt Age bottom boundary layer}


\index[prmindex]{Age bottom boundary layer}
\index[prmindexfull]{Initial conditions!Adiabatic!Age bottom boundary layer}
{\it Value:} 0e0


{\it Default:} 0e0


{\it Description:} The age of the lower thermal boundary layer, used for the calculation of the half-space cooling model temperature. Units: years if the 'Use years in output instead of seconds' parameter is set; seconds otherwise.


{\it Possible values:} [Double 0...1.79769e+308 (inclusive)]
\item {\it Parameter name:} {\tt Age top boundary layer}


\index[prmindex]{Age top boundary layer}
\index[prmindexfull]{Initial conditions!Adiabatic!Age top boundary layer}
{\it Value:} 0e0


{\it Default:} 0e0


{\it Description:} The age of the upper thermal boundary layer, used for the calculation of the half-space cooling model temperature. Units: years if the 'Use years in output instead of seconds' parameter is set; seconds otherwise.


{\it Possible values:} [Double 0...1.79769e+308 (inclusive)]
\item {\it Parameter name:} {\tt Amplitude}


\index[prmindex]{Amplitude}
\index[prmindexfull]{Initial conditions!Adiabatic!Amplitude}
{\it Value:} 0e0


{\it Default:} 0e0


{\it Description:} The amplitude (in K) of the initial spherical temperature perturbation at the bottom of the model domain. This perturbation will be added to the adiabatic temperature profile, but not to the bottom thermal boundary layer. Instead, the maximum of the perturbation and the bottom boundary layer temperature will be used.


{\it Possible values:} [Double 0...1.79769e+308 (inclusive)]
\item {\it Parameter name:} {\tt Position}


\index[prmindex]{Position}
\index[prmindexfull]{Initial conditions!Adiabatic!Position}
{\it Value:} center


{\it Default:} center


{\it Description:} Where the initial temperature perturbation should be placed (in the center or at the boundary of the model domain).


{\it Possible values:} [Selection center|boundary ]
\item {\it Parameter name:} {\tt Radius}


\index[prmindex]{Radius}
\index[prmindexfull]{Initial conditions!Adiabatic!Radius}
{\it Value:} 0e0


{\it Default:} 0e0


{\it Description:} The Radius (in m) of the initial spherical temperature perturbation at the bottom of the model domain.


{\it Possible values:} [Double 0...1.79769e+308 (inclusive)]
\item {\it Parameter name:} {\tt Subadiabaticity}


\index[prmindex]{Subadiabaticity}
\index[prmindexfull]{Initial conditions!Adiabatic!Subadiabaticity}
{\it Value:} 0e0


{\it Default:} 0e0


{\it Description:} If this value is larger than 0, the initial temperature profile will not be adiabatic, but subadiabatic. This value gives the maximal deviation from adiabaticity. Set to 0 for an adiabatic temperature profile. Units: K.

The function object in the Function subsection represents the compositional fields that will be used as a reference profile for calculating the thermal diffusivity. The function depends only on depth.


{\it Possible values:} [Double 0...1.79769e+308 (inclusive)]
\end{itemize}



\subsection{Parameters in section \tt Initial conditions/Adiabatic/Function}
\label{parameters:Initial_20conditions/Adiabatic/Function}

\begin{itemize}
\item {\it Parameter name:} {\tt Function constants}


\index[prmindex]{Function constants}
\index[prmindexfull]{Initial conditions!Adiabatic!Function!Function constants}
{\it Value:} 


{\it Default:} 


{\it Description:} Sometimes it is convenient to use symbolic constants in the expression that describes the function, rather than having to use its numeric value everywhere the constant appears. These values can be defined using this parameter, in the form `var1=value1, var2=value2, ...'.

A typical example would be to set this runtime parameter to `pi=3.1415926536' and then use `pi' in the expression of the actual formula. (That said, for convenience this class actually defines both `pi' and `Pi' by default, but you get the idea.)


{\it Possible values:} [Anything]
\item {\it Parameter name:} {\tt Function expression}


\index[prmindex]{Function expression}
\index[prmindexfull]{Initial conditions!Adiabatic!Function!Function expression}
{\it Value:} 0


{\it Default:} 0


{\it Description:} The formula that denotes the function you want to evaluate for particular values of the independent variables. This expression may contain any of the usual operations such as addition or multiplication, as well as all of the common functions such as `sin' or `cos'. In addition, it may contain expressions like `if(x>0, 1, -1)' where the expression evaluates to the second argument if the first argument is true, and to the third argument otherwise. For a full overview of possible expressions accepted see the documentation of the fparser library.

If the function you are describing represents a vector-valued function with multiple components, then separate the expressions for individual components by a semicolon.


{\it Possible values:} [Anything]
\item {\it Parameter name:} {\tt Variable names}


\index[prmindex]{Variable names}
\index[prmindexfull]{Initial conditions!Adiabatic!Function!Variable names}
{\it Value:} x,t


{\it Default:} x,t


{\it Description:} The name of the variables as they will be used in the function, separated by commas. By default, the names of variables at which the function will be evaluated is `x' (in 1d), `x,y' (in 2d) or `x,y,z' (in 3d) for spatial coordinates and `t' for time. You can then use these variable names in your function expression and they will be replaced by the values of these variables at which the function is currently evaluated. However, you can also choose a different set of names for the independent variables at which to evaluate your function expression. For example, if you work in spherical coordinates, you may wish to set this input parameter to `r,phi,theta,t' and then use these variable names in your function expression.


{\it Possible values:} [Anything]
\end{itemize}

\subsection{Parameters in section \tt Initial conditions/Function}
\label{parameters:Initial_20conditions/Function}

\begin{itemize}
\item {\it Parameter name:} {\tt Function constants}


\index[prmindex]{Function constants}
\index[prmindexfull]{Initial conditions!Function!Function constants}
{\it Value:} 


{\it Default:} 


{\it Description:} Sometimes it is convenient to use symbolic constants in the expression that describes the function, rather than having to use its numeric value everywhere the constant appears. These values can be defined using this parameter, in the form `var1=value1, var2=value2, ...'.

A typical example would be to set this runtime parameter to `pi=3.1415926536' and then use `pi' in the expression of the actual formula. (That said, for convenience this class actually defines both `pi' and `Pi' by default, but you get the idea.)


{\it Possible values:} [Anything]
\item {\it Parameter name:} {\tt Function expression}


\index[prmindex]{Function expression}
\index[prmindexfull]{Initial conditions!Function!Function expression}
{\it Value:} 0


{\it Default:} 0


{\it Description:} The formula that denotes the function you want to evaluate for particular values of the independent variables. This expression may contain any of the usual operations such as addition or multiplication, as well as all of the common functions such as `sin' or `cos'. In addition, it may contain expressions like `if(x>0, 1, -1)' where the expression evaluates to the second argument if the first argument is true, and to the third argument otherwise. For a full overview of possible expressions accepted see the documentation of the fparser library.

If the function you are describing represents a vector-valued function with multiple components, then separate the expressions for individual components by a semicolon.


{\it Possible values:} [Anything]
\item {\it Parameter name:} {\tt Variable names}


\index[prmindex]{Variable names}
\index[prmindexfull]{Initial conditions!Function!Variable names}
{\it Value:} x,y,t


{\it Default:} x,y,t


{\it Description:} The name of the variables as they will be used in the function, separated by commas. By default, the names of variables at which the function will be evaluated is `x' (in 1d), `x,y' (in 2d) or `x,y,z' (in 3d) for spatial coordinates and `t' for time. You can then use these variable names in your function expression and they will be replaced by the values of these variables at which the function is currently evaluated. However, you can also choose a different set of names for the independent variables at which to evaluate your function expression. For example, if you work in spherical coordinates, you may wish to set this input parameter to `r,phi,theta,t' and then use these variable names in your function expression.


{\it Possible values:} [Anything]
\end{itemize}

\subsection{Parameters in section \tt Initial conditions/Spherical gaussian perturbation}
\label{parameters:Initial_20conditions/Spherical_20gaussian_20perturbation}

\begin{itemize}
\item {\it Parameter name:} {\tt Amplitude}


\index[prmindex]{Amplitude}
\index[prmindexfull]{Initial conditions!Spherical gaussian perturbation!Amplitude}
{\it Value:} 0.01


{\it Default:} 0.01


{\it Description:} The amplitude of the perturbation.


{\it Possible values:} [Double 0...1.79769e+308 (inclusive)]
\item {\it Parameter name:} {\tt Angle}


\index[prmindex]{Angle}
\index[prmindexfull]{Initial conditions!Spherical gaussian perturbation!Angle}
{\it Value:} 0e0


{\it Default:} 0e0


{\it Description:} The angle where the center of the perturbation is placed.


{\it Possible values:} [Double 0...1.79769e+308 (inclusive)]
\item {\it Parameter name:} {\tt Filename for initial geotherm table}


\index[prmindex]{Filename for initial geotherm table}
\index[prmindexfull]{Initial conditions!Spherical gaussian perturbation!Filename for initial geotherm table}
{\it Value:} initial-geotherm-table


{\it Default:} initial-geotherm-table


{\it Description:} The file from which the initial geotherm table is to be read. The format of the file is defined by what is read in source/initial\_conditions/spherical\_shell.cc.


{\it Possible values:} [FileName (Type: input)]
\item {\it Parameter name:} {\tt Non-dimensional depth}


\index[prmindex]{Non-dimensional depth}
\index[prmindexfull]{Initial conditions!Spherical gaussian perturbation!Non-dimensional depth}
{\it Value:} 0.7


{\it Default:} 0.7


{\it Description:} The non-dimensional radial distance where the center of the perturbation is placed.


{\it Possible values:} [Double 0...1.79769e+308 (inclusive)]
\item {\it Parameter name:} {\tt Sigma}


\index[prmindex]{Sigma}
\index[prmindexfull]{Initial conditions!Spherical gaussian perturbation!Sigma}
{\it Value:} 0.2


{\it Default:} 0.2


{\it Description:} The standard deviation of the Gaussian perturbation.


{\it Possible values:} [Double 0...1.79769e+308 (inclusive)]
\item {\it Parameter name:} {\tt Sign}


\index[prmindex]{Sign}
\index[prmindexfull]{Initial conditions!Spherical gaussian perturbation!Sign}
{\it Value:} 1


{\it Default:} 1


{\it Description:} The sign of the perturbation.


{\it Possible values:} [Double -1.79769e+308...1.79769e+308 (inclusive)]
\end{itemize}

\subsection{Parameters in section \tt Material model}
\label{parameters:Material_20model}

\begin{itemize}
\item {\it Parameter name:} {\tt Model name}


\index[prmindex]{Model name}
\index[prmindexfull]{Material model!Model name}
{\it Value:} simple


{\it Default:} 


{\it Description:} Select one of the following models:

`SolCx': A material model that corresponds to the 'SolCx' benchmark defined in Duretz et al., G-Cubed, 2011.

`SolKz': A material model that corresponds to the 'SolKz' benchmark defined in Duretz et al., G-Cubed, 2011.

`Inclusion': A material model that corresponds to the 'Inclusion' benchmark defined in Duretz et al., G-Cubed, 2011.

`simple': A simple material model that has constant values for all coefficients but the density and viscosity. This model uses the formulation that assumes an incompressible medium despite the fact that the density follows the law $\rho(T)=\rho_0(1-\beta(T-T_{\text{ref}})$. The temperature dependency of viscosity is  switched off by default and follows the formula$\eta(T)=\eta_0*e^{\eta_T*\Delta T / T_{\text{ref}})}$.The value for the components of this formula and additional parameters are read from the parameter file in subsection 'Simple model'.

`Steinberger': lookup viscosity from the paper of Steinberger/Calderwood2006 and material data from a database generated by Perplex. The database builds upon the thermodynamic database by Stixrude 2011 and assumes a pyrolitic composition by Ringwood 1988. 

`table': A material model that reads tables of pressure and temperature dependent material coefficients from files. The default values for this model's runtime parameters use a material description taken from the paper \textit{Complex phase distribution and seismic velocity structure of the transition zone: Convection model predictions for a magnesium-endmember olivine-pyroxene mantle} by Michael H.G. Jacobs and Arie P. van den Berg, Physics of the Earth and Planetary Interiors, Volume 186, Issues 1-2, May 2011, Pages 36--48. See \url{http://www.sciencedirect.com/science/article/pii/S0031920111000422}.

`Tan Gurnis': A simple compressible material model based on a benchmark from the paper of Tan/Gurnis (2007). This does not use the temperature equation, but has a hardcoded temperature.

`timo': A simple material model that has constant values for all coefficients but the density and viscosity. This model uses the formulation that assumes an incompressible medium despite the fact that the density follows the law $\rho(T)=\rho_0(1-\beta(T-T_{\text{ref}})$. The temperature dependency of viscosity is  switched off by default and follows the formula$\eta(T)=\eta_0*e^{\eta_T*\Delta T / T_{\text{ref}})}$.The value for the components of this formula and additional parameters are read from the parameter file in subsection 'Simple model'.


{\it Possible values:} [Selection SolCx|SolKz|Inclusion|simple|Steinberger|table|Tan Gurnis|timo ]
\end{itemize}



\subsection{Parameters in section \tt Material model/Inclusion}
\label{parameters:Material_20model/Inclusion}

\begin{itemize}
\item {\it Parameter name:} {\tt Viscosity jump}


\index[prmindex]{Viscosity jump}
\index[prmindexfull]{Material model!Inclusion!Viscosity jump}
{\it Value:} 1e3


{\it Default:} 1e3


{\it Description:} Viscosity in the Inclusion.


{\it Possible values:} [Double 0...1.79769e+308 (inclusive)]
\end{itemize}

\subsection{Parameters in section \tt Material model/Simple model}
\label{parameters:Material_20model/Simple_20model}

\begin{itemize}
\item {\it Parameter name:} {\tt Composition viscosity prefactor}


\index[prmindex]{Composition viscosity prefactor}
\index[prmindexfull]{Material model!Simple model!Composition viscosity prefactor}
{\it Value:} 1.0


{\it Default:} 1.0


{\it Description:} A linear dependency of viscosity on composition. Dimensionless prefactor.


{\it Possible values:} [Double 0...1.79769e+308 (inclusive)]
\item {\it Parameter name:} {\tt Density differential for compositional field 1}


\index[prmindex]{Density differential for compositional field 1}
\index[prmindexfull]{Material model!Simple model!Density differential for compositional field 1}
{\it Value:} 0


{\it Default:} 0


{\it Description:} If compositional fields are used, then one would frequently want to make the density depend on these fields. In this simple material model, we make the following assumptions: if no compositional fields are used in the current simulation, then the density is simply the usual one with its linear dependence on the temperature. If there are compositional fields, then the density only depends on the first one in such a way that the density has an additional term of the kind $+\Delta \rho \; c_1(\mathbf x)$. This parameter describes the value of $\Delta \rho$. Units: $kg/m^3/\textrm{unit change in composition}$.


{\it Possible values:} [Double -1.79769e+308...1.79769e+308 (inclusive)]
\item {\it Parameter name:} {\tt Reference density}


\index[prmindex]{Reference density}
\index[prmindexfull]{Material model!Simple model!Reference density}
{\it Value:} 3300


{\it Default:} 3300


{\it Description:} Reference density $\rho_0$. Units: $kg/m^3$.


{\it Possible values:} [Double 0...1.79769e+308 (inclusive)]
\item {\it Parameter name:} {\tt Reference specific heat}


\index[prmindex]{Reference specific heat}
\index[prmindexfull]{Material model!Simple model!Reference specific heat}
{\it Value:} 1250


{\it Default:} 1250


{\it Description:} The value of the specific heat $cp$. Units: $J/kg/K$.


{\it Possible values:} [Double 0...1.79769e+308 (inclusive)]
\item {\it Parameter name:} {\tt Reference temperature}


\index[prmindex]{Reference temperature}
\index[prmindexfull]{Material model!Simple model!Reference temperature}
{\it Value:} 293


{\it Default:} 293


{\it Description:} The reference temperature $T_0$. Units: $K$.


{\it Possible values:} [Double 0...1.79769e+308 (inclusive)]
\item {\it Parameter name:} {\tt Thermal conductivity}


\index[prmindex]{Thermal conductivity}
\index[prmindexfull]{Material model!Simple model!Thermal conductivity}
{\it Value:} 4.7


{\it Default:} 4.7


{\it Description:} The value of the thermal conductivity $k$. Units: $W/m/K$.


{\it Possible values:} [Double 0...1.79769e+308 (inclusive)]
\item {\it Parameter name:} {\tt Thermal expansion coefficient}


\index[prmindex]{Thermal expansion coefficient}
\index[prmindexfull]{Material model!Simple model!Thermal expansion coefficient}
{\it Value:} 2e-5


{\it Default:} 2e-5


{\it Description:} The value of the thermal expansion coefficient $\beta$. Units: $1/K$.


{\it Possible values:} [Double 0...1.79769e+308 (inclusive)]
\item {\it Parameter name:} {\tt Thermal viscosity exponent}


\index[prmindex]{Thermal viscosity exponent}
\index[prmindexfull]{Material model!Simple model!Thermal viscosity exponent}
{\it Value:} 0.0


{\it Default:} 0.0


{\it Description:} The temperature dependence of viscosity. Dimensionless exponent.


{\it Possible values:} [Double 0...1.79769e+308 (inclusive)]
\item {\it Parameter name:} {\tt Viscosity}


\index[prmindex]{Viscosity}
\index[prmindexfull]{Material model!Simple model!Viscosity}
{\it Value:} 5e24


{\it Default:} 5e24


{\it Description:} The value of the constant viscosity. Units: $kg/m/s$.


{\it Possible values:} [Double 0...1.79769e+308 (inclusive)]
\end{itemize}

\subsection{Parameters in section \tt Material model/SolCx}
\label{parameters:Material_20model/SolCx}

\begin{itemize}
\item {\it Parameter name:} {\tt Background density}


\index[prmindex]{Background density}
\index[prmindexfull]{Material model!SolCx!Background density}
{\it Value:} 0


{\it Default:} 0


{\it Description:} Density value upon which the variation of this testcase is overlaid. Since this background density is constant it does not affect the flow pattern but it adds to the total pressure since it produces a nonzero adiabatic pressure if set to a nonzero value.


{\it Possible values:} [Double 0...1.79769e+308 (inclusive)]
\item {\it Parameter name:} {\tt Viscosity jump}


\index[prmindex]{Viscosity jump}
\index[prmindexfull]{Material model!SolCx!Viscosity jump}
{\it Value:} 1e6


{\it Default:} 1e6


{\it Description:} Viscosity in the right half of the domain.


{\it Possible values:} [Double 0...1.79769e+308 (inclusive)]
\end{itemize}

\subsection{Parameters in section \tt Material model/Steinberger model}
\label{parameters:Material_20model/Steinberger_20model}

\begin{itemize}
\item {\it Parameter name:} {\tt Bilinear interpolation}


\index[prmindex]{Bilinear interpolation}
\index[prmindexfull]{Material model!Steinberger model!Bilinear interpolation}
{\it Value:} true


{\it Default:} true


{\it Description:} whether to use bilinear interpolation to compute material properties (slower but more accurate).


{\it Possible values:} [Bool]
\item {\it Parameter name:} {\tt Data directory}


\index[prmindex]{Data directory}
\index[prmindexfull]{Material model!Steinberger model!Data directory}
{\it Value:} data/material-model/steinberger/


{\it Default:} data/material-model/steinberger/


{\it Description:} The path to the model data. 


{\it Possible values:} [DirectoryName]
\item {\it Parameter name:} {\tt Latent heat}


\index[prmindex]{Latent heat}
\index[prmindexfull]{Material model!Steinberger model!Latent heat}
{\it Value:} false


{\it Default:} false


{\it Description:} whether to include latent heat effects in thecalculation of thermal expansivity and specific heat.Following the approach of Nakagawa et al. 2009.


{\it Possible values:} [Bool]
\item {\it Parameter name:} {\tt Lateral viscosity file name}


\index[prmindex]{Lateral viscosity file name}
\index[prmindexfull]{Material model!Steinberger model!Lateral viscosity file name}
{\it Value:} temp-viscosity-prefactor.txt


{\it Default:} temp-viscosity-prefactor.txt


{\it Description:} The file name of the lateral viscosity data. 


{\it Possible values:} [Anything]
\item {\it Parameter name:} {\tt Material file names}


\index[prmindex]{Material file names}
\index[prmindexfull]{Material model!Steinberger model!Material file names}
{\it Value:} pyr-ringwood88.txt


{\it Default:} pyr-ringwood88.txt


{\it Description:} The file names of the material data. List with as many components as activecompositional fields (material data is assumed tobe in order with the ordering of the fields). 


{\it Possible values:} [List list of <[Anything]> of length 0...4294967295 (inclusive)]
\item {\it Parameter name:} {\tt Radial viscosity file name}


\index[prmindex]{Radial viscosity file name}
\index[prmindexfull]{Material model!Steinberger model!Radial viscosity file name}
{\it Value:} radial-visc.txt


{\it Default:} radial-visc.txt


{\it Description:} The file name of the radial viscosity data. 


{\it Possible values:} [Anything]
\end{itemize}

\subsection{Parameters in section \tt Material model/Table model}
\label{parameters:Material_20model/Table_20model}

\begin{itemize}
\item {\it Parameter name:} {\tt Composition}


\index[prmindex]{Composition}
\index[prmindexfull]{Material model!Table model!Composition}
{\it Value:} standard


{\it Default:} standard


{\it Description:} The Composition of the model. 


{\it Possible values:} [Anything]
\item {\it Parameter name:} {\tt Compressible}


\index[prmindex]{Compressible}
\index[prmindexfull]{Material model!Table model!Compressible}
{\it Value:} true


{\it Default:} true


{\it Description:} whether the model is compressible. 


{\it Possible values:} [Bool]
\item {\it Parameter name:} {\tt ComputePhases}


\index[prmindex]{ComputePhases}
\index[prmindexfull]{Material model!Table model!ComputePhases}
{\it Value:} false


{\it Default:} false


{\it Description:} whether to compute phases. 


{\it Possible values:} [Bool]
\item {\it Parameter name:} {\tt Gravity}


\index[prmindex]{Gravity}
\index[prmindexfull]{Material model!Table model!Gravity}
{\it Value:} 30


{\it Default:} 30


{\it Description:} The value of the gravity constant.Units: $m/s^2$.


{\it Possible values:} [Double 0...1.79769e+308 (inclusive)]
\item {\it Parameter name:} {\tt Path to model data}


\index[prmindex]{Path to model data}
\index[prmindexfull]{Material model!Table model!Path to model data}
{\it Value:} data/material-model/table/


{\it Default:} data/material-model/table/


{\it Description:} The path to the model data. 


{\it Possible values:} [DirectoryName]
\item {\it Parameter name:} {\tt Reference density}


\index[prmindex]{Reference density}
\index[prmindexfull]{Material model!Table model!Reference density}
{\it Value:} 3300


{\it Default:} 3300


{\it Description:} Reference density $\rho_0$. Units: $kg/m^3$.


{\it Possible values:} [Double 0...1.79769e+308 (inclusive)]
\item {\it Parameter name:} {\tt Reference specific heat}


\index[prmindex]{Reference specific heat}
\index[prmindexfull]{Material model!Table model!Reference specific heat}
{\it Value:} 1250


{\it Default:} 1250


{\it Description:} The value of the specific heat $cp$. Units: $J/kg/K$.


{\it Possible values:} [Double 0...1.79769e+308 (inclusive)]
\item {\it Parameter name:} {\tt Reference temperature}


\index[prmindex]{Reference temperature}
\index[prmindexfull]{Material model!Table model!Reference temperature}
{\it Value:} 293


{\it Default:} 293


{\it Description:} The reference temperature $T_0$. Units: $K$.


{\it Possible values:} [Double 0...1.79769e+308 (inclusive)]
\item {\it Parameter name:} {\tt Thermal conductivity}


\index[prmindex]{Thermal conductivity}
\index[prmindexfull]{Material model!Table model!Thermal conductivity}
{\it Value:} 4.7


{\it Default:} 4.7


{\it Description:} The value of the thermal conductivity $k$. Units: $W/m/K$.


{\it Possible values:} [Double 0...1.79769e+308 (inclusive)]
\item {\it Parameter name:} {\tt Thermal expansion coefficient}


\index[prmindex]{Thermal expansion coefficient}
\index[prmindexfull]{Material model!Table model!Thermal expansion coefficient}
{\it Value:} 2e-5


{\it Default:} 2e-5


{\it Description:} The value of the thermal expansion coefficient $\beta$. Units: $1/K$.


{\it Possible values:} [Double 0...1.79769e+308 (inclusive)]
\end{itemize}



\subsection{Parameters in section \tt Material model/Table model/Viscosity}
\label{parameters:Material_20model/Table_20model/Viscosity}

\begin{itemize}
\item {\it Parameter name:} {\tt Reference Viscosity}


\index[prmindex]{Reference Viscosity}
\index[prmindexfull]{Material model!Table model!Viscosity!Reference Viscosity}
{\it Value:} 5e24


{\it Default:} 5e24


{\it Description:} The value of the constant viscosity. Units: $kg/m/s$.


{\it Possible values:} [Double 0...1.79769e+308 (inclusive)]
\item {\it Parameter name:} {\tt Viscosity Model}


\index[prmindex]{Viscosity Model}
\index[prmindexfull]{Material model!Table model!Viscosity!Viscosity Model}
{\it Value:} Exponential


{\it Default:} Exponential


{\it Description:} Viscosity Model


{\it Possible values:} [Anything]
\item {\it Parameter name:} {\tt Viscosity increase lower mantle}


\index[prmindex]{Viscosity increase lower mantle}
\index[prmindexfull]{Material model!Table model!Viscosity!Viscosity increase lower mantle}
{\it Value:} 1e0


{\it Default:} 1e0


{\it Description:} The Viscosity increase (jump) in the lower mantle.


{\it Possible values:} [Double 0...1.79769e+308 (inclusive)]
\end{itemize}



\subsection{Parameters in section \tt Material model/Table model/Viscosity/Composite}
\label{parameters:Material_20model/Table_20model/Viscosity/Composite}

\begin{itemize}
\item {\it Parameter name:} {\tt Activation energy diffusion}


\index[prmindex]{Activation energy diffusion}
\index[prmindexfull]{Material model!Table model!Viscosity!Composite!Activation energy diffusion}
{\it Value:} 335e3


{\it Default:} 335e3


{\it Description:} activation energy for diffusion creep


{\it Possible values:} [Double 0...1.79769e+308 (inclusive)]
\item {\it Parameter name:} {\tt Activation energy dislocation}


\index[prmindex]{Activation energy dislocation}
\index[prmindexfull]{Material model!Table model!Viscosity!Composite!Activation energy dislocation}
{\it Value:} 540e3


{\it Default:} 540e3


{\it Description:} activation energy for dislocation creep


{\it Possible values:} [Double 0...1.79769e+308 (inclusive)]
\item {\it Parameter name:} {\tt Activation volume diffusion}


\index[prmindex]{Activation volume diffusion}
\index[prmindexfull]{Material model!Table model!Viscosity!Composite!Activation volume diffusion}
{\it Value:} 4.0e-6


{\it Default:} 4.0e-6


{\it Description:} activation volume for diffusion creep


{\it Possible values:} [Double 0...1.79769e+308 (inclusive)]
\item {\it Parameter name:} {\tt Activation volume dislocation}


\index[prmindex]{Activation volume dislocation}
\index[prmindexfull]{Material model!Table model!Viscosity!Composite!Activation volume dislocation}
{\it Value:} 14.0e-6


{\it Default:} 14.0e-6


{\it Description:} activation volume for dislocation creep


{\it Possible values:} [Double 0...1.79769e+308 (inclusive)]
\item {\it Parameter name:} {\tt Prefactor diffusion}


\index[prmindex]{Prefactor diffusion}
\index[prmindexfull]{Material model!Table model!Viscosity!Composite!Prefactor diffusion}
{\it Value:} 1.92e-11


{\it Default:} 1.92e-11


{\it Description:} prefactor for diffusion creep (1e0/prefactor)*exp((activation\_energy+activation\_volume*pressure)/(R*temperature))


{\it Possible values:} [Double 0...1.79769e+308 (inclusive)]
\item {\it Parameter name:} {\tt Prefactor dislocation}


\index[prmindex]{Prefactor dislocation}
\index[prmindexfull]{Material model!Table model!Viscosity!Composite!Prefactor dislocation}
{\it Value:} 2.42e-10


{\it Default:} 2.42e-10


{\it Description:} prefactor for dislocation creep (1e0/prefactor)*exp((activation\_energy+activation\_volume*pressure)/(R*temperature))


{\it Possible values:} [Double 0...1.79769e+308 (inclusive)]
\item {\it Parameter name:} {\tt Stress exponent}


\index[prmindex]{Stress exponent}
\index[prmindexfull]{Material model!Table model!Viscosity!Composite!Stress exponent}
{\it Value:} 3.5


{\it Default:} 3.5


{\it Description:} stress exponent for dislocation creep


{\it Possible values:} [Double 0...1.79769e+308 (inclusive)]
\end{itemize}

\subsection{Parameters in section \tt Material model/Table model/Viscosity/Diffusion}
\label{parameters:Material_20model/Table_20model/Viscosity/Diffusion}

\begin{itemize}
\item {\it Parameter name:} {\tt Activation energy diffusion}


\index[prmindex]{Activation energy diffusion}
\index[prmindexfull]{Material model!Table model!Viscosity!Diffusion!Activation energy diffusion}
{\it Value:} 335e3


{\it Default:} 335e3


{\it Description:} activation energy for diffusion creep


{\it Possible values:} [Double 0...1.79769e+308 (inclusive)]
\item {\it Parameter name:} {\tt Activation volume diffusion}


\index[prmindex]{Activation volume diffusion}
\index[prmindexfull]{Material model!Table model!Viscosity!Diffusion!Activation volume diffusion}
{\it Value:} 4.0e-6


{\it Default:} 4.0e-6


{\it Description:} activation volume for diffusion creep


{\it Possible values:} [Double 0...1.79769e+308 (inclusive)]
\item {\it Parameter name:} {\tt Prefactor diffusion}


\index[prmindex]{Prefactor diffusion}
\index[prmindexfull]{Material model!Table model!Viscosity!Diffusion!Prefactor diffusion}
{\it Value:} 1.92e-11


{\it Default:} 1.92e-11


{\it Description:} prefactor for diffusion creep (1e0/prefactor)*exp((activation\_energy+activation\_volume*pressure)/(R*temperature))


{\it Possible values:} [Double 0...1.79769e+308 (inclusive)]
\end{itemize}

\subsection{Parameters in section \tt Material model/Table model/Viscosity/Dislocation}
\label{parameters:Material_20model/Table_20model/Viscosity/Dislocation}

\begin{itemize}
\item {\it Parameter name:} {\tt Activation energy dislocation}


\index[prmindex]{Activation energy dislocation}
\index[prmindexfull]{Material model!Table model!Viscosity!Dislocation!Activation energy dislocation}
{\it Value:} 335e3


{\it Default:} 335e3


{\it Description:} activation energy for dislocation creep


{\it Possible values:} [Double 0...1.79769e+308 (inclusive)]
\item {\it Parameter name:} {\tt Activation volume dislocation}


\index[prmindex]{Activation volume dislocation}
\index[prmindexfull]{Material model!Table model!Viscosity!Dislocation!Activation volume dislocation}
{\it Value:} 4.0e-6


{\it Default:} 4.0e-6


{\it Description:} activation volume for dislocation creep


{\it Possible values:} [Double 0...1.79769e+308 (inclusive)]
\item {\it Parameter name:} {\tt Prefactor dislocation}


\index[prmindex]{Prefactor dislocation}
\index[prmindexfull]{Material model!Table model!Viscosity!Dislocation!Prefactor dislocation}
{\it Value:} 1.92e-11


{\it Default:} 1.92e-11


{\it Description:} prefactor for dislocation creep (1e0/prefactor)*exp((activation\_energy+activation\_volume*pressure)/(R*temperature))


{\it Possible values:} [Double 0...1.79769e+308 (inclusive)]
\item {\it Parameter name:} {\tt Stress exponent}


\index[prmindex]{Stress exponent}
\index[prmindexfull]{Material model!Table model!Viscosity!Dislocation!Stress exponent}
{\it Value:} 3.5


{\it Default:} 3.5


{\it Description:} stress exponent for dislocation creep


{\it Possible values:} [Double 0...1.79769e+308 (inclusive)]
\end{itemize}

\subsection{Parameters in section \tt Material model/Table model/Viscosity/Exponential}
\label{parameters:Material_20model/Table_20model/Viscosity/Exponential}

\begin{itemize}
\item {\it Parameter name:} {\tt Exponential P}


\index[prmindex]{Exponential P}
\index[prmindexfull]{Material model!Table model!Viscosity!Exponential!Exponential P}
{\it Value:} 1


{\it Default:} 1


{\it Description:} multiplication factor or Pressure exponent


{\it Possible values:} [Double 0...1.79769e+308 (inclusive)]
\item {\it Parameter name:} {\tt Exponential T}


\index[prmindex]{Exponential T}
\index[prmindexfull]{Material model!Table model!Viscosity!Exponential!Exponential T}
{\it Value:} 1


{\it Default:} 1


{\it Description:} multiplication factor or Temperature exponent


{\it Possible values:} [Double 0...1.79769e+308 (inclusive)]
\end{itemize}

\subsection{Parameters in section \tt Material model/Tan Gurnis model}
\label{parameters:Material_20model/Tan_20Gurnis_20model}

\begin{itemize}
\item {\it Parameter name:} {\tt Di}


\index[prmindex]{Di}
\index[prmindexfull]{Material model!Tan Gurnis model!Di}
{\it Value:} 0.5


{\it Default:} 0.5


{\it Possible values:} [Double 0...1.79769e+308 (inclusive)]
\item {\it Parameter name:} {\tt Reference density}


\index[prmindex]{Reference density}
\index[prmindexfull]{Material model!Tan Gurnis model!Reference density}
{\it Value:} 3300


{\it Default:} 3300


{\it Description:} Reference density $\rho_0$. Units: $kg/m^3$.


{\it Possible values:} [Double 0...1.79769e+308 (inclusive)]
\item {\it Parameter name:} {\tt Reference specific heat}


\index[prmindex]{Reference specific heat}
\index[prmindexfull]{Material model!Tan Gurnis model!Reference specific heat}
{\it Value:} 1250


{\it Default:} 1250


{\it Description:} The value of the specific heat $cp$. Units: $J/kg/K$.


{\it Possible values:} [Double 0...1.79769e+308 (inclusive)]
\item {\it Parameter name:} {\tt Reference temperature}


\index[prmindex]{Reference temperature}
\index[prmindexfull]{Material model!Tan Gurnis model!Reference temperature}
{\it Value:} 293


{\it Default:} 293


{\it Description:} The reference temperature $T_0$. Units: $K$.


{\it Possible values:} [Double 0...1.79769e+308 (inclusive)]
\item {\it Parameter name:} {\tt Thermal conductivity}


\index[prmindex]{Thermal conductivity}
\index[prmindexfull]{Material model!Tan Gurnis model!Thermal conductivity}
{\it Value:} 4.7


{\it Default:} 4.7


{\it Description:} The value of the thermal conductivity $k$. Units: $W/m/K$.


{\it Possible values:} [Double 0...1.79769e+308 (inclusive)]
\item {\it Parameter name:} {\tt Thermal expansion coefficient}


\index[prmindex]{Thermal expansion coefficient}
\index[prmindexfull]{Material model!Tan Gurnis model!Thermal expansion coefficient}
{\it Value:} 2e-5


{\it Default:} 2e-5


{\it Description:} The value of the thermal expansion coefficient $\beta$. Units: $1/K$.


{\it Possible values:} [Double 0...1.79769e+308 (inclusive)]
\item {\it Parameter name:} {\tt Viscosity}


\index[prmindex]{Viscosity}
\index[prmindexfull]{Material model!Tan Gurnis model!Viscosity}
{\it Value:} 5e24


{\it Default:} 5e24


{\it Description:} The value of the constant viscosity. Units: $kg/m/s$.


{\it Possible values:} [Double 0...1.79769e+308 (inclusive)]
\item {\it Parameter name:} {\tt a}


\index[prmindex]{a}
\index[prmindexfull]{Material model!Tan Gurnis model!a}
{\it Value:} 0


{\it Default:} 0


{\it Possible values:} [Double 0...1.79769e+308 (inclusive)]
\item {\it Parameter name:} {\tt gamma}


\index[prmindex]{gamma}
\index[prmindexfull]{Material model!Tan Gurnis model!gamma}
{\it Value:} 1


{\it Default:} 1


{\it Possible values:} [Double 0...1.79769e+308 (inclusive)]
\item {\it Parameter name:} {\tt wavenumber}


\index[prmindex]{wavenumber}
\index[prmindexfull]{Material model!Tan Gurnis model!wavenumber}
{\it Value:} 1


{\it Default:} 1


{\it Possible values:} [Double 0...1.79769e+308 (inclusive)]
\end{itemize}

\subsection{Parameters in section \tt Material model/Timo model}
\label{parameters:Material_20model/Timo_20model}

\begin{itemize}
\item {\it Parameter name:} {\tt Composition viscosity prefactor}


\index[prmindex]{Composition viscosity prefactor}
\index[prmindexfull]{Material model!Timo model!Composition viscosity prefactor}
{\it Value:} 1.0


{\it Default:} 1.0


{\it Description:} A linear dependency of viscosity on composition. Dimensionless prefactor.


{\it Possible values:} [Double 0...1.79769e+308 (inclusive)]
\item {\it Parameter name:} {\tt Density differential for compositional field 1}


\index[prmindex]{Density differential for compositional field 1}
\index[prmindexfull]{Material model!Timo model!Density differential for compositional field 1}
{\it Value:} 0


{\it Default:} 0


{\it Description:} If compositional fields are used, then one would frequently want to make the density depend on these fields. In this simple material model, we make the following assumptions: if no compositional fields are used in the current simulation, then the density is simply the usual one with its linear dependence on the temperature. If there are compositional fields, then the density only depends on the first one in such a way that the density has an additional term of the kind $+\Delta \rho \; c_1(\mathbf x)$. This parameter describes the value of $\Delta \rho$. Units: $kg/m^3/\textrm{unit change in composition}$.


{\it Possible values:} [Double -1.79769e+308...1.79769e+308 (inclusive)]
\item {\it Parameter name:} {\tt Reference density}


\index[prmindex]{Reference density}
\index[prmindexfull]{Material model!Timo model!Reference density}
{\it Value:} 3300


{\it Default:} 3300


{\it Description:} Reference density $\rho_0$. Units: $kg/m^3$.


{\it Possible values:} [Double 0...1.79769e+308 (inclusive)]
\item {\it Parameter name:} {\tt Reference specific heat}


\index[prmindex]{Reference specific heat}
\index[prmindexfull]{Material model!Timo model!Reference specific heat}
{\it Value:} 1250


{\it Default:} 1250


{\it Description:} The value of the specific heat $cp$. Units: $J/kg/K$.


{\it Possible values:} [Double 0...1.79769e+308 (inclusive)]
\item {\it Parameter name:} {\tt Reference temperature}


\index[prmindex]{Reference temperature}
\index[prmindexfull]{Material model!Timo model!Reference temperature}
{\it Value:} 293


{\it Default:} 293


{\it Description:} The reference temperature $T_0$. Units: $K$.


{\it Possible values:} [Double 0...1.79769e+308 (inclusive)]
\item {\it Parameter name:} {\tt Thermal conductivity}


\index[prmindex]{Thermal conductivity}
\index[prmindexfull]{Material model!Timo model!Thermal conductivity}
{\it Value:} 4.7


{\it Default:} 4.7


{\it Description:} The value of the thermal conductivity $k$. Units: $W/m/K$.


{\it Possible values:} [Double 0...1.79769e+308 (inclusive)]
\item {\it Parameter name:} {\tt Thermal expansion coefficient}


\index[prmindex]{Thermal expansion coefficient}
\index[prmindexfull]{Material model!Timo model!Thermal expansion coefficient}
{\it Value:} 2e-5


{\it Default:} 2e-5


{\it Description:} The value of the thermal expansion coefficient $\beta$. Units: $1/K$.


{\it Possible values:} [Double 0...1.79769e+308 (inclusive)]
\item {\it Parameter name:} {\tt Thermal viscosity exponent}


\index[prmindex]{Thermal viscosity exponent}
\index[prmindexfull]{Material model!Timo model!Thermal viscosity exponent}
{\it Value:} 0.0


{\it Default:} 0.0


{\it Description:} The temperature dependence of viscosity. Dimensionless exponent.


{\it Possible values:} [Double 0...1.79769e+308 (inclusive)]
\item {\it Parameter name:} {\tt Viscosity}


\index[prmindex]{Viscosity}
\index[prmindexfull]{Material model!Timo model!Viscosity}
{\it Value:} 5e24


{\it Default:} 5e24


{\it Description:} The value of the constant viscosity. Units: $kg/m/s$.


{\it Possible values:} [Double 0...1.79769e+308 (inclusive)]
\end{itemize}

\subsection{Parameters in section \tt Mesh refinement}
\label{parameters:Mesh_20refinement}

\begin{itemize}
\item {\it Parameter name:} {\tt Additional refinement times}


\index[prmindex]{Additional refinement times}
\index[prmindexfull]{Mesh refinement!Additional refinement times}
{\it Value:} 


{\it Default:} 


{\it Description:} A list of times so that if the end time of a time step is beyond this time, an additional round of mesh refinement is triggered. This is mostly useful to make sure we can get through the initial transient phase of a simulation on a relatively coarse mesh, and then refine again when we are in a time range that we are interested in and where we would like to use a finer mesh. Units: each element of the list has units years if the 'Use years in output instead of seconds' parameter is set; seconds otherwise.


{\it Possible values:} [List list of <[Double 0...1.79769e+308 (inclusive)]> of length 0...4294967295 (inclusive)]
\item {\it Parameter name:} {\tt Coarsening fraction}


\index[prmindex]{Coarsening fraction}
\index[prmindexfull]{Mesh refinement!Coarsening fraction}
{\it Value:} 0.05


{\it Default:} 0.05


{\it Description:} The fraction of cells with the smallest error that should be flagged for coarsening.


{\it Possible values:} [Double 0...1 (inclusive)]
\item {\it Parameter name:} {\tt Initial adaptive refinement}


\index[prmindex]{Initial adaptive refinement}
\index[prmindexfull]{Mesh refinement!Initial adaptive refinement}
{\it Value:} 2


{\it Default:} 2


{\it Description:} The number of adaptive refinement steps performed after initial global refinement but while still within the first time step.


{\it Possible values:} [Integer range 0...2147483647 (inclusive)]
\item {\it Parameter name:} {\tt Initial global refinement}


\index[prmindex]{Initial global refinement}
\index[prmindexfull]{Mesh refinement!Initial global refinement}
{\it Value:} 2


{\it Default:} 2


{\it Description:} The number of global refinement steps performed on the initial coarse mesh, before the problem is first solved there.


{\it Possible values:} [Integer range 0...2147483647 (inclusive)]
\item {\it Parameter name:} {\tt Normalize individual refinement criteria}


\index[prmindex]{Normalize individual refinement criteria}
\index[prmindexfull]{Mesh refinement!Normalize individual refinement criteria}
{\it Value:} true


{\it Default:} true


{\it Description:} If multiple refinement criteria are specified in the ``Strategy'' parameter, then they need to be combined somehow to form the final refinement indicators. This is done using the method described by the ``Refinement criteria merge operation'' parameter which can either operate on the raw refinement indicators returned by each strategy (i.e., dimensional quantities) or using normalized values where the indicators of each strategy are first normalized to the interval $[0,1]$ (which also makes them non-dimensional). This parameter determines whether this normalization will happen.


{\it Possible values:} [Bool]
\item {\it Parameter name:} {\tt Refinement criteria merge operation}


\index[prmindex]{Refinement criteria merge operation}
\index[prmindexfull]{Mesh refinement!Refinement criteria merge operation}
{\it Value:} max


{\it Default:} max


{\it Description:} If multiple mesh refinement criteria are computed for each cell (by passing a list of more than element to the \texttt{Strategy} parameter in this section of the input file) then one will have to decide which one should win when deciding which cell to refine. The operation that selects from these competing criteria is the one that is selected here. The options are:

\begin{itemize}
\item \texttt{plus}: Add the various error indicators together and refine those cells on which the sum of indicators is largest.
\item \texttt{max}: Take the maximum of the various error indicators and refine those cells on which the maximal indicators is largest.
\end{itemize}The refinement indicators computed by each strategy are modified by the ``Normalize individual refinement criteria'' and ``Refinement criteria scale factors'' parameters.


{\it Possible values:} [Selection plus|max ]
\item {\it Parameter name:} {\tt Refinement criteria scaling factors}


\index[prmindex]{Refinement criteria scaling factors}
\index[prmindexfull]{Mesh refinement!Refinement criteria scaling factors}
{\it Value:} 


{\it Default:} 


{\it Description:} A list of scaling factors by which every individual refinement criterion will be multiplied by. If only a single refinement criterion is selected (using the ``Strategy'' parameter, then this parameter has no particular meaning. On the other hand, if multiple criteria are chosen, then these factors are used to weigh the various indicators relative to each other. 

If ``Normalize individual refinement criteria'' is set to true, then the criteria will first be normalized to the interval $[0,1]$ and then multiplied by the factors specified here. You will likely want to choose the factors to be not too far from 1 in that case, say between 1 and 10, to avoid essentially disabling those criteria with small weights. On the other hand, if the criteria are not normalized to $[0,1]$ using the parameter mentioned above, then the factors you specify here need to take into account the relative numerical size of refinement indicators (which in that case carry physical units).

You can experimentally play with these scaling factors by choosing to output the refinement indicators into the graphical output of a run.

If the list of indicators given in this parameter is empty, then this indicates that they should all be chosen equal to one. If the list is not empty then it needs to have as many entries as there are indicators chosen in the ``Strategy'' parameter.


{\it Possible values:} [List list of <[Double 0...1.79769e+308 (inclusive)]> of length 0...4294967295 (inclusive)]
\item {\it Parameter name:} {\tt Refinement fraction}


\index[prmindex]{Refinement fraction}
\index[prmindexfull]{Mesh refinement!Refinement fraction}
{\it Value:} 0.3


{\it Default:} 0.3


{\it Description:} The fraction of cells with the largest error that should be flagged for refinement.


{\it Possible values:} [Double 0...1 (inclusive)]
\item {\it Parameter name:} {\tt Run postprocessors on initial refinement}


\index[prmindex]{Run postprocessors on initial refinement}
\index[prmindexfull]{Mesh refinement!Run postprocessors on initial refinement}
{\it Value:} false


{\it Default:} false


{\it Description:} Whether or not the postproccessors should be run at the end of each of ths initial adaptive refinement cycles at the of the simulation start.


{\it Possible values:} [Bool]
\item {\it Parameter name:} {\tt Strategy}


\index[prmindex]{Strategy}
\index[prmindexfull]{Mesh refinement!Strategy}
{\it Value:} thermal energy density


{\it Default:} thermal energy density


{\it Description:} A comma separated list of mesh refinement criteria that will be run whenever mesh refinement is required. The results of each of these criteria will, i.e., the refinement indicators they produce for all the cells of the mesh will then be normalized to a range between zero and one and the results of different criteria will then be merged through the operation selected in this section.

The following criteria are available:

`composition': A mesh refinement criterion that computes refinement indicators from the compositional fields. If there is more than one compositional field, then it simply takes the sum of the indicators computed from each of the compositional field.

`density': A mesh refinement criterion that computes refinement indicators from a field that describes the spatial variability of the density, $\rho$. Because this quantity may not be a continuous function ($\rho$ and $C_p$ may be discontinuous functions along discontinuities in the medium, for example due to phase changes), we approximate the gradient of this quantity to refine the mesh. The error indicator defined here takes the magnitude of the approximate gradient and scales it by $h_K^{1+d/2}$ where $h_K$ is the diameter of each cell and $d$ is the dimension. This scaling ensures that the error indicators converge to zero as $h_K\rightarrow 0$ even if the energy density is discontinuous, since the gradient of a discontinuous function grows like $1/h_K$.

`temperature': A mesh refinement criterion that computes refinement indicators from the temperature field.

`thermal energy density': A mesh refinement criterion that computes refinement indicators from a field that describes the spatial variability of the thermal energy density, $\rho C_p T$. Because this quantity may not be a continuous function ($\rho$ and $C_p$ may be discontinuous functions along discontinuities in the medium, for example due to phase changes), we approximate the gradient of this quantity to refine the mesh. The error indicator defined here takes the magnitude of the approximate gradient and scales it by $h_K^{1.5}$ where $h_K$ is the diameter of each cell. This scaling ensures that the error indicators converge to zero as $h_K\rightarrow 0$ even if the energy density is discontinuous, since the gradient of a discontinuous function grows like $1/h_K$.

`velocity': A mesh refinement criterion that computes refinement indicators from the velocity field.


{\it Possible values:} [MultipleSelection composition|density|temperature|thermal energy density|velocity ]
\item {\it Parameter name:} {\tt Time steps between mesh refinement}


\index[prmindex]{Time steps between mesh refinement}
\index[prmindexfull]{Mesh refinement!Time steps between mesh refinement}
{\it Value:} 10


{\it Default:} 10


{\it Description:} The number of time steps after which the mesh is to be adapted again based on computed error indicators. If 0 then the mesh will never be changed.


{\it Possible values:} [Integer range 0...2147483647 (inclusive)]
\end{itemize}

\subsection{Parameters in section \tt Model settings}
\label{parameters:Model_20settings}

\begin{itemize}
\item {\it Parameter name:} {\tt Fixed temperature boundary indicators}


\index[prmindex]{Fixed temperature boundary indicators}
\index[prmindexfull]{Model settings!Fixed temperature boundary indicators}
{\it Value:} 


{\it Default:} 


{\it Description:} A comma separated list of integers denoting those boundaries on which the temperature is fixed and described by the boundary temperature object selected in its own section of this input file. All boundary indicators used by the geometry but not explicitly listed here will end up with no-flux (insulating) boundary conditions.

This parameter only describes which boundaries have a fixed temperature, but not what temperature should hold on these boundaries. The latter piece of information needs to be implemented in a plugin in the BoundaryTemperature group, unless an existing implementation in this group already provides what you want.


{\it Possible values:} [List list of <[Integer range 0...2147483647 (inclusive)]> of length 0...4294967295 (inclusive)]
\item {\it Parameter name:} {\tt Include adiabatic heating}


\index[prmindex]{Include adiabatic heating}
\index[prmindexfull]{Model settings!Include adiabatic heating}
{\it Value:} false


{\it Default:} false


{\it Description:} Whether to include adiabatic heating into the model or not. From a physical viewpoint, adiabatic heating should always be used but may be undesirable when comparing results with known benchmarks that do not include this term in the temperature equation.


{\it Possible values:} [Bool]
\item {\it Parameter name:} {\tt Include shear heating}


\index[prmindex]{Include shear heating}
\index[prmindexfull]{Model settings!Include shear heating}
{\it Value:} true


{\it Default:} true


{\it Description:} Whether to include shear heating into the model or not. From a physical viewpoint, shear heating should always be used but may be undesirable when comparing results with known benchmarks that do not include this term in the temperature equation.


{\it Possible values:} [Bool]
\item {\it Parameter name:} {\tt Prescribed velocity boundary indicators}


\index[prmindex]{Prescribed velocity boundary indicators}
\index[prmindexfull]{Model settings!Prescribed velocity boundary indicators}
{\it Value:} 


{\it Default:} 


{\it Description:} A comma separated list denoting those boundaries on which the velocity is tangential but prescribed, i.e., where external forces act to prescribe a particular velocity. This is often used to prescribe a velocity that equals that of overlying plates.

The format of valid entries for this parameter is that of a map given as ``key1: value1, key2: value2, key3: value3, ...'' where each key must be a valid boundary indicator and each value must be one of the currently implemented boundary velocity models.

Note that the no-slip boundary condition is a special case of the current one where the prescribed velocity happens to be zero. It can thus be implemented by indicating that a particular boundary is part of the ones selected using the current parameter and using ``zero velocity'' as the boundary values. Alternatively, you can simply list the part of the boundary on which the velocity is to be zero with the parameter ``Zero velocity boundary indicator'' in the current parameter section.


{\it Possible values:} [Map map of <[Integer range 0...255 (inclusive)]:[Selection inclusion|function|gplates|zero velocity ]> of length 0...4294967295 (inclusive)]
\item {\it Parameter name:} {\tt Radiogenic heating rate}


\index[prmindex]{Radiogenic heating rate}
\index[prmindexfull]{Model settings!Radiogenic heating rate}
{\it Value:} 0e0


{\it Default:} 0e0


{\it Description:} H0


{\it Possible values:} [Double -1.79769e+308...1.79769e+308 (inclusive)]
\item {\it Parameter name:} {\tt Tangential velocity boundary indicators}


\index[prmindex]{Tangential velocity boundary indicators}
\index[prmindexfull]{Model settings!Tangential velocity boundary indicators}
{\it Value:} 


{\it Default:} 


{\it Description:} A comma separated list of integers denoting those boundaries on which the velocity is tangential and unrestrained, i.e., free-slip where no external forces act to prescribe a particular tangential velocity (although there is a force that requires the flow to be tangential).


{\it Possible values:} [List list of <[Integer range 0...255 (inclusive)]> of length 0...4294967295 (inclusive)]
\item {\it Parameter name:} {\tt Zero velocity boundary indicators}


\index[prmindex]{Zero velocity boundary indicators}
\index[prmindexfull]{Model settings!Zero velocity boundary indicators}
{\it Value:} 


{\it Default:} 


{\it Description:} A comma separated list of integers denoting those boundaries on which the velocity is zero.


{\it Possible values:} [List list of <[Integer range 0...255 (inclusive)]> of length 0...4294967295 (inclusive)]
\end{itemize}

\subsection{Parameters in section \tt Postprocess}
\label{parameters:Postprocess}

\begin{itemize}
\item {\it Parameter name:} {\tt List of postprocessors}


\index[prmindex]{List of postprocessors}
\index[prmindexfull]{Postprocess!List of postprocessors}
{\it Value:} all


{\it Default:} all


{\it Description:} A comma separated list of postprocessor objects that should be run at the end of each time step. Some of these postprocessors will declare their own parameters which may, for example, include that they will actually do something only every so many time steps or years. Alternatively, the text 'all' indicates that all available postprocessors should be run after each time step.

The following postprocessors are available:

`composition statistics': A postprocessor that computes some statistics about the compositional fields, if present in this simulation. In particular, it computes maximal and minimal values of each field, as well as the total mass contained in this field as defined by the integral $m_i(t) = \int_\Omega c_i(\mathbf x,t) \; dx$.

`depth average': A postprocessor that computes depth averaged quantities and writes them out.

`DuretzEtAl error': A postprocessor that compares the solution of the benchmarks from the Duretz et al., G-Cubed, 2011, paper with the one computed by ASPECT and reports the error. Specifically, it can compute the errors for the SolCx, SolKz and inclusion benchmarks. The postprocessor inquires which material model is currently being used and adjusts which exact solution to use accordingly.

`heat flux statistics': A postprocessor that computes some statistics about the heat flux across boundaries.

`heat flux statistics for the table model': A postprocessor that computes some statistics about the heat flux across boundaries.

`velocity statistics for the table model': A postprocessor that computes some statistics about the velocity field.

`Tan Gurnis error': A postprocessor that compares the solution of the benchmarks from the Tan/Gurnis (2007) paper with the one computed by ASPECT by outputing data that is compared using a matlab script.

`temperature statistics': A postprocessor that computes some statistics about the temperature field.

`tracers': Postprocessor that propagates passive tracer particles based on the velocity field.

`velocity statistics': A postprocessor that computes some statistics about the velocity field.

`visualization': A postprocessor that takes the solution and writes it into files that can be read by a graphical visualization program. Additional run time parameters are read from the parameter subsection 'Visualization'.


{\it Possible values:} [MultipleSelection composition statistics|depth average|DuretzEtAl error|heat flux statistics|heat flux statistics for the table model|velocity statistics for the table model|Tan Gurnis error|temperature statistics|tracers|velocity statistics|visualization|all ]
\end{itemize}



\subsection{Parameters in section \tt Postprocess/Depth average}
\label{parameters:Postprocess/Depth_20average}

\begin{itemize}
\item {\it Parameter name:} {\tt Time between graphical output}


\index[prmindex]{Time between graphical output}
\index[prmindexfull]{Postprocess!Depth average!Time between graphical output}
{\it Value:} 1e8


{\it Default:} 1e8


{\it Description:} The time interval between each generation of graphical output files. A value of zero indicates that output should be generated in each time step. Units: years if the 'Use years in output instead of seconds' parameter is set; seconds otherwise.


{\it Possible values:} [Double 0...1.79769e+308 (inclusive)]
\end{itemize}

\subsection{Parameters in section \tt Postprocess/Tracers}
\label{parameters:Postprocess/Tracers}

\begin{itemize}
\item {\it Parameter name:} {\tt Data output format}


\index[prmindex]{Data output format}
\index[prmindexfull]{Postprocess!Tracers!Data output format}
{\it Value:} none


{\it Default:} none


{\it Description:} File format to output raw particle data in.


{\it Possible values:} [Selection none|ascii|vtu|hdf5 ]
\item {\it Parameter name:} {\tt Integration scheme}


\index[prmindex]{Integration scheme}
\index[prmindexfull]{Postprocess!Tracers!Integration scheme}
{\it Value:} rk2


{\it Default:} rk2


{\it Description:} Integration scheme to move particles.


{\it Possible values:} [Selection euler|rk2|rk4|hybrid ]
\item {\it Parameter name:} {\tt Number of tracers}


\index[prmindex]{Number of tracers}
\index[prmindexfull]{Postprocess!Tracers!Number of tracers}
{\it Value:} 1e3


{\it Default:} 1e3


{\it Description:} Total number of tracers to create (not per processor or per element).


{\it Possible values:} [Double 0...1.79769e+308 (inclusive)]
\item {\it Parameter name:} {\tt Time between data output}


\index[prmindex]{Time between data output}
\index[prmindexfull]{Postprocess!Tracers!Time between data output}
{\it Value:} 1e8


{\it Default:} 1e8


{\it Description:} The time interval between each generation of output files. A value of zero indicates that output should be generated every time step. Units: years if the 'Use years in output instead of seconds' parameter is set; seconds otherwise.


{\it Possible values:} [Double 0...1.79769e+308 (inclusive)]
\end{itemize}

\subsection{Parameters in section \tt Postprocess/Visualization}
\label{parameters:Postprocess/Visualization}

\begin{itemize}
\item {\it Parameter name:} {\tt List of output variables}


\index[prmindex]{List of output variables}
\index[prmindexfull]{Postprocess!Visualization!List of output variables}
{\it Value:} 


{\it Default:} 


{\it Description:} A comma separated list of visualization objects that should be run whenever writing graphical output. By default, the graphical output files will always contain the primary variables velocity, pressure, and temperature. However, one frequently wants to also visualize derived quantities, such as the thermodynamic phase that corresponds to a given temperature-pressure value, or the corresponding seismic wave speeds. The visualization objects do exactly this: they compute such derived quantities and place them into the output file. The current parameter is the place where you decide which of these additional output variables you want to have in your output file.

The following postprocessors are available:

`density': A visualization output object that generates output for the density.

`error indicator': A visualization output object that generates output showing the estimated error or other mesh refinement indicator as a spatially variable function with one value per cell.

`friction heating': A visualization output object that generates output for the amount of friction heating often referred to as $\tau:\epsilon$. More concisely, in the incompressible case, the quantity that is output is defined as $\eta \varepsilon(\mathbf u):\varepsilon(\mathbf u)$ where $\eta$ is itself a function of temperature, pressure and strain rate. In the compressible case, the quantity that's computed is $\eta [\varepsilon(\mathbf u)-\tfrac 13(\textrm{tr}\;\varepsilon(\mathbf u))\mathbf I]:[\varepsilon(\mathbf u)-\tfrac 13(\textrm{tr}\;\varepsilon(\mathbf u))\mathbf I]$.

`nonadiabatic pressure': A visualization output object that generates output for the non-adiabatic component of the pressure.

`nonadiabatic temperature': A visualization output object that generates output for the non-adiabatic component of the pressure.

`partition': A visualization output object that generates output for the parallel partition that every cell of the mesh is associated with.

`Vs anomaly': A visualization output object that generates output showing the anomaly in the seismic shear wave speed $V_s$ as a spatially variable function with one value per cell. This anomaly is shown as a percentage change relative to the average value of $V_s$ at the depth of this cell.

`Vp anomaly': A visualization output object that generates output showing the anomaly in the seismic compression wave speed $V_p$ as a spatially variable function with one value per cell. This anomaly is shown as a percentage change relative to the average value of $V_p$ at the depth of this cell.

`seismic vp': A visualization output object that generates output for the seismic P-wave speed.

`seismic vs': A visualization output object that generates output for the seismic S-wave speed.

`specific heat': A visualization output object that generates output for the specific heat $C_p$.

`strain rate': A visualization output object that generates output for the norm of the strain rate, i.e., for the quantity $\sqrt{\varepsilon(\mathbf u):\varepsilon(\mathbf u)}$ in the incompressible case and $\sqrt{[\varepsilon(\mathbf u)-\tfrac 13(\textrm{tr}\;\varepsilon(\mathbf u))\mathbf I]:[\varepsilon(\mathbf u)-\tfrac 13(\textrm{tr}\;\varepsilon(\mathbf u))\mathbf I]}$ in the compressible case.

`thermal expansivity': A visualization output object that generates output for the thermal expansivity.

`thermodynamic phase': A visualization output object that generates output for the integer number of the phase that is thermodynamically stable at the temperature and pressure of the current point.

`viscosity': A visualization output object that generates output for the viscosity.

`viscosity ratio': A visualization output object that generates output for the ratio between dislocation viscosity and diffusion viscosity.


{\it Possible values:} [MultipleSelection density|error indicator|friction heating|nonadiabatic pressure|nonadiabatic temperature|partition|Vs anomaly|Vp anomaly|seismic vp|seismic vs|specific heat|strain rate|thermal expansivity|thermodynamic phase|viscosity|viscosity ratio|all ]
\item {\it Parameter name:} {\tt Number of grouped files}


\index[prmindex]{Number of grouped files}
\index[prmindexfull]{Postprocess!Visualization!Number of grouped files}
{\it Value:} 0


{\it Default:} 0


{\it Description:} VTU file output supports grouping files from several CPUs into one file using MPI I/O when writing on a parallel filesystem. Select 0 for no grouping. This will disable parallel file output and instead write one file per processor in a background thread. A value of 1 will generate one big file containing the whole solution.


{\it Possible values:} [Integer range 0...2147483647 (inclusive)]
\item {\it Parameter name:} {\tt Output format}


\index[prmindex]{Output format}
\index[prmindexfull]{Postprocess!Visualization!Output format}
{\it Value:} vtu


{\it Default:} vtu


{\it Description:} The file format to be used for graphical output.


{\it Possible values:} [Selection none|dx|ucd|gnuplot|povray|eps|gmv|tecplot|tecplot\_binary|vtk|vtu|hdf5|deal.II intermediate ]
\item {\it Parameter name:} {\tt Time between graphical output}


\index[prmindex]{Time between graphical output}
\index[prmindexfull]{Postprocess!Visualization!Time between graphical output}
{\it Value:} 1e8


{\it Default:} 1e8


{\it Description:} The time interval between each generation of graphical output files. A value of zero indicates that output should be generated in each time step. Units: years if the 'Use years in output instead of seconds' parameter is set; seconds otherwise.


{\it Possible values:} [Double 0...1.79769e+308 (inclusive)]
\end{itemize}

\subsection{Parameters in section \tt Termination criteria}
\label{parameters:Termination_20criteria}

\begin{itemize}
\item {\it Parameter name:} {\tt Checkpoint on termination}


\index[prmindex]{Checkpoint on termination}
\index[prmindexfull]{Termination criteria!Checkpoint on termination}
{\it Value:} false


{\it Default:} false


{\it Description:} Whether to checkpoint the simulation right before termination.


{\it Possible values:} [Bool]
\item {\it Parameter name:} {\tt Termination criteria}


\index[prmindex]{Termination criteria}
\index[prmindexfull]{Termination criteria!Termination criteria}
{\it Value:} end time


{\it Default:} end time


{\it Description:} A comma separated list of termination criteria that will determine when the simulation should end. Whether explicitly stated or not, the ``end time'' termination criterion will always be used.The following termination criteria are available:

`end time': Terminate the simulation once the end time specified in the input file has been reached. Unlike all other termination criteria, this criterion is \textit{always} active, whether it has been explicitly selected or not in the input file (this is done to preserve historical behavior of \aspect{}, but it also likely does not inconvenience anyone since it is what would be selected in most cases anyway).

`steady state velocity': A criterion that terminates the simulation when the RMS of the velocity field stays within a certain range for a specified period of time.

`user request': Terminate the simulation gracefully when a file with a specified name appears in the output directory. This allows the user to gracefully exit the simulation at any time by simply creating such a file using, for example, \texttt{touch output/terminate}. The file's location is chosen to be in the output directory, rather than in a generic location such as the Aspect directory, so that one can run multiple simulations at the same time (which presumably write to different output directories) and can selectively terminate a particular one.


{\it Possible values:} [MultipleSelection end time|steady state velocity|user request|all ]
\end{itemize}



\subsection{Parameters in section \tt Termination criteria/Steady state velocity}
\label{parameters:Termination_20criteria/Steady_20state_20velocity}

\begin{itemize}
\item {\it Parameter name:} {\tt Maximum relative deviation}


\index[prmindex]{Maximum relative deviation}
\index[prmindexfull]{Termination criteria!Steady state velocity!Maximum relative deviation}
{\it Value:} 0.05


{\it Default:} 0.05


{\it Description:} The maximum relative deviation of the RMS in recent simulation time for the system to be considered in steady state. If the actual deviation is smaller than this number, then the simulation will be terminated.


{\it Possible values:} [Double 0...1.79769e+308 (inclusive)]
\item {\it Parameter name:} {\tt Time in steady state}


\index[prmindex]{Time in steady state}
\index[prmindexfull]{Termination criteria!Steady state velocity!Time in steady state}
{\it Value:} 1e7


{\it Default:} 1e7


{\it Description:} The minimum length of simulation time that the system should be in steady state before termination.Units: years if the 'Use years in output instead of seconds' parameter is set; seconds otherwise.


{\it Possible values:} [Double 0...1.79769e+308 (inclusive)]
\end{itemize}

\subsection{Parameters in section \tt Termination criteria/User request}
\label{parameters:Termination_20criteria/User_20request}

\begin{itemize}
\item {\it Parameter name:} {\tt File name}


\index[prmindex]{File name}
\index[prmindexfull]{Termination criteria!User request!File name}
{\it Value:} terminate-aspect


{\it Default:} terminate-aspect


{\it Description:} The name of a file that, if it exists in the output directory (whose name is also specified in the input file) will lead to termination of the simulation. The file's location is chosen to be in the output directory, rather than in a generic location such as the Aspect directory, so that one can run multiple simulations at the same time (which presumably write to different output directories) and can selectively terminate a particular one.


{\it Possible values:} [FileName (Type: input)]
\end{itemize}
